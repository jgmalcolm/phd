\documentclass[letterpaper,hyperref,12pt]{gatech-thesis}
\usepackage{amsfonts,amsmath,amssymb,latexsym}
\usepackage{fixltx2e} % maintain order of floats
\usepackage{cite,mdwlist}
\usepackage{color}
\usepackage{graphicx} \graphicspath{{/Users/malcolm/junk/writeups/fibers/figs/ukf/}
                                    {/Users/malcolm/junk/writeups/fibers/figs/watson/}
                                    {/Users/malcolm/junk/writeups/fibers/figs/watson_response/}}

\usepackage[draft]{hyperref}
\usepackage[all]{hypcap} % hyperref/caption fix -- subfig broken
\usepackage[font=footnotesize]{subfig}
\renewcommand{\equationautorefname}{Equation}
\newcommand{\subfigureautorefname}{Figure}
\renewcommand{\figureautorefname}{Figure}
\renewcommand{\sectionautorefname}{Section}
\renewcommand{\subsectionautorefname}{Section}
\newcommand{\algorithmautorefname}{Algorithm}
\newcommand{\citet}[1]{\cite{#1}} % since no natbib

\usepackage{algorithm,algorithmic}
\floatname{algorithm}{Algorithm}

\usepackage{xspace}
\renewcommand{\deg}{\ensuremath{^\circ}\xspace}
\renewcommand{\v}[1]{\ensuremath{\mathbf #1}\xspace}
\newcommand{\rv}[1]{\ensuremath{[\, #1 \,]}\xspace} % row vector

\makeatletter
\DeclareRobustCommand\onedot{\futurelet\@let@token\@onedot}
\def\@onedot{\ifx\@let@token.\else.\null\fi\xspace}
\newcommand{\ie}{\textit{i.e}\onedot}
\newcommand{\eg}{\textit{e.g}\onedot}
\newcommand{\etal}{et al\onedot}
\makeatother

\newcommand{\R}{\ensuremath{\mathbb R}}
\renewcommand{\S}{\ensuremath{\mathbb S}}

\newcommand{\s}{\v s}
\newcommand{\bn}[1]{\ensuremath{b\!=\!#1}\xspace}
\newcommand{\bone}{\bn{1000}}
\newcommand{\bthree}{\bn{3000}}
\newcommand{\snr}[1]{SNR $\approx$ #1 dB}
\renewcommand{\u}{\v u}

% tensor
\newcommand{\m}{\v m}
\newcommand{\lx}{{\ensuremath{\lambda_1}} \xspace}
\newcommand{\ly}{{\ensuremath{\lambda_2}} \xspace}

% UKF
\newcommand{\norm}[1]{\ensuremath{\|#1\|}}
\newcommand{\x}{\v x}
\newcommand{\y}{\v y}
\newcommand{\X}{\v X}
\newcommand{\Y}{\v Y}

\title{Filtered Tractography}
\author{James G. Malcolm}
\department{Electrical and Computer Engineering}

%% Can have up to six readers, plus principaladvisor and
%% committeechair. All have the form
%%
%%  \reader{Name}[Department][Institution]
%%
%% The second and third arguments are optional, but if you wish to
%% supply the third, you must supply the second. Department defaults
%% to the department defined above and Institution defaults to Georgia
%% Institute of Technology.

\principaladvisor{Yogesh Rathi}
\committeechair{Anthony Yezzi}
\firstreader{Allen Tannenbaum}
\secondreader{Patricio Vela}

\degree{Doctor of Philosophy}

\copyrightyear{2009}
\submitdate{9 October 2009}

\approveddate{Not Yet Approved}

\bibfiles{references}

\figurespagefalse

\tablespagefalse
\dedicationheadingfalse
\thesisproposaltrue

\begin{document}



\bibliographystyle{gatech-thesis}
\begin{preliminary}
\contents


\begin{summary}
  Computer vision encompasses a host of computational techniques to process
  visual information.
  %
  Medical imagery is one particular area of application where data comes in
  various forms: X-rays, ultrasound probes, MRI volumes, EEG recordings, NMR
  spectroscopy, etc.
  %
  This thesis proposal is concerned with techniques for accurate
  reconstruction of neural pathways from diffusion magnetic resonance imagery
  (dMRI).

  This proposal introduces a filtered approach to neural tractography.
  %
  Existing methods independently estimate the diffusion model at each voxel so
  there is no running knowledge of confidence in the estimation process.
  %
  We propose using tractography to drive estimation of the local diffusion
  model.
  %
  Toward this end, we formulate fiber tracking as recursive estimation: at
  each step of tracing the fiber, the current estimate is guided by those
  previous.

  We show preliminary results using a mixture of Watson directional functions
  to model the diffusion signal.  Compared to conventional techniques, this
  filtered approach significantly improves the angular resolution at crossings
  and branchings.  Further, we confirm its ability to trace through regions
  known to contain such crossing and branching while providing inherent path
  regularization.

  Future work will extend the technique to various alternative diffusion
  models and methods of tracing, as well as techniques for the analysis of
  tracts and validation of connectivity.
\end{summary}

\end{preliminary}

\chapter{Research Topic}

\section{Introduction}
The advent of diffusion weighted magnetic resonance imaging has provided the
opportunity for non-invasive investigation of neural architecture.  Using this
imaging technique, neuroscientists can determine how neurons originating from
one region connect to other regions and how well-defined those connections may
be.  For such studies, the quality of the results relies heavily on the chosen
fiber representation and the method of reconstructing pathways.

To begin studying the microstructure of fibers, we need a model to interpret
the diffusion weighted signal.  Such models fall broadly into two categories:
parametric and nonparametric.
%
One of the simplest parametric models is the diffusion tensor which describes
a Gaussian estimate of the diffusion orientation and strength at each voxel.
While robust, this model can be inadequate in cases of mixed fiber presence or
more complex orientations \cite{Alexander2002,Frank2002}.
%
To handle more complex diffusion patterns, various parametric models have been
introduced including mixtures of tensors
\cite{Alexander2001,Tuch2002,Parker2005,Kreher2005,Peled2006} and directional
functions \cite{McGraw2006,Kaden2007,Rathi2009mia_w}.  To demonstrate the
difference that arises, \autoref{fig:sweetness} shows a tracing from the
center of the corpus callosum.  While a single-tensor model finds only the
dominant U-shaped structure, multi-fiber methods reveal many of the lateral
pathways known to exist.

\begin{figure}[t]
  \centering
  \subfloat[Single-tensor streamline]{
    \includegraphics[width=.48\columnwidth]{case01045_1T_tc_sweet}}%
  \subfloat[Two-component model]{
    \includegraphics[width=.48\columnwidth]{case01045_2W_tc_sweet}}%
  \caption{Comparison of tractography using a single-tensor model and the
    proposed two-component model with filtering.  While the single-tensor
    model misses many of the lateral branches from the corpus callosum, the
    filter provides a stable estimate of the two-component model capable of
    revealing the lateral transcallosal pathways.  Seed region indicated with
    yellow.}
  \label{fig:sweetness}
\end{figure}
Nonparametric models often provide more information about the diffusion
pattern.  Instead of estimating a discrete number of fibers as in parametric
models, nonparametric techniques estimate an oriented distribution function
(ODF) describing an arbitrary configuration of fibers.
%
For this estimation, \citet{Tuch2004} introduced Q-ball imaging to numerically
compute the ODF via the Funk-Radon transform, and subsequently, the use of
spherical harmonics simplified the computation with an analytic form
\cite{Anderson2005,Hess2006,Descoteaux2007mrm}.  Recently, linear Kalman
filtering was demonstrated for online direct estimation of single-tensor and
harmonic coefficients \cite{Poupon2008,Deriche2009}.
%
Another approach to producing an ODF is to assume a model for the signal
response of a single-fiber and use spherical deconvolution
\cite{Jian2007dot,Jansons2003,Tournier2004,Kaden2007,Kumar2008}.  Diffusion
oriented transforms offer even another alternative \cite{Ozarslan2006}.  A
good review of both parametric and nonparametric models can be found in
\cite{Alexander2005nyas,Descoteaux2007}.


Based on these models, several techniques attempt to reconstruct pathways.
%
Deterministic tractography involves directly following the diffusion pathways.
In the single tensor model, this means simply following the principal
diffusion direction \cite{Basser2000}, while multi-fiber models often include
techniques for determining the number of fibers present or when pathways
branch \cite{Hagmann2004,Kreher2005,Guo2006}.  Kalman and particle filters
have been used with single tensor streamline tractography
\cite{Gossl2002,Bjornemo2002,Zhang2009}, but these are used for path
regularization and not to estimate the underlying fiber model.  Another
approach to regularizing single tensor tractography uses a moving least
squares estimate weighted with the previous tensor \cite{Zhukov2002}.
%
In this present study we focus on deterministic techniques, but probabilistic
methods have also been developed to form connections by sampling
\cite{Parker2003,Hosey2005,Behrens2007}.

Several studies have specific relevance to this present work because of their
use of a filtering strategy in either orientation estimation or tractography.
%
In extending standard streamline tractography to enhance path regularization,
\cite{Gossl2002} move curve integration into a linear Kalman filter while
\cite{Zhukov2002} incorporate a moving least squares estimator.
%
Alternatively, one could use a particle filter to place a prior on the
direction of propagation \cite{Zhang2009}.
%
Since these methods model only the position of the fiber, not the local fiber
model, they are inherently focused on path regularization rather than
estimating the underlying fiber structure.
%
Finally, \cite{Poupon2008} proposed using a linear Kalman filter for online,
direct estimation of either single-tensor or harmonic coefficients while
successive diffusion image slices are acquired, while \cite{Deriche2009}
revisited the technique to account for proper regularization and proposed a
method to quickly determine optimal gradient set orderings.



\section{Filtered Tractography}

Of the approaches listed above, nearly all fit the model at each voxel
independent of other voxels; however, tractography is a causal process: we
arrive at each new position along the fiber based upon the diffusion found at
the previous position.

We propose to we treat model estimation and tractography as such by placing
this process within a causal filter.
%
As we examine the signal at each new position, the filter recursively updates
the underlying local model parameters, provides the variance of that estimate,
and indicates the direction in which to propagate tractography.
\autoref{fig:overview} provides an overview of this recursive process.
\begin{figure}[t]
  \centering
  \resizebox{!}{1.8in}{\input{/Users/malcolm/junk/writeups/fibers/figs/ukf/ukf.pstex_t}}
  \caption{System overview illustrating relation between the neural fibers,
    the measured scanner signal, and the unscented Kalman filter as it is used
    to estimate the local model.  At each step, the filter uses its current
    model state ($\hat{\x}_t$) to predict the observed scanner signal
    ($\bar{\y}_{t+1|t}$) and then compares that against the actual measured
    signal ($\y_t$) in order to update its internal model state
    ($\hat{\x}_{t+1}$).}
  \label{fig:overview}
\end{figure}

To begin estimating within a finite dimensional filter, we model the diffusion
signal using a mixture of Watson functions.  We choose the Watson function
since it provides a compact representation of the signal parameterized by the
principal diffusion direction and a scaling parameter describing anisotropy,
and further allows analytic reconstruction of the oriented diffusion function
from those parameters \cite{Rathi2009mia_w}.  This enables estimation directly
from the raw signal without separate preprocessing or regularization.
%
Because the signal reconstruction is nonlinear, we use the unscented Kalman
filter to estimate the model parameters and then propagate in the most
consistent direction.
%
Using causal estimation in this way yields inherent path regularization and
accurate fiber resolution at crossing angles not found with independent
optimization.  In a loop, the filter estimates the model at the current
position, moves a step in the most consistent direction, and then begins
estimation again.
%
Since each iteration begins with a near-optimal solution provided by the
previous estimation, the convergence of model fitting is improved and many
local minima are naturally avoided.
%
This approach generalizes to arbitrary fiber model with finite dimensional
parameter space.

\autoref{sec:model} provides the necessary background on modeling the
measurement signal using directional functions and defines the specific fiber
model employed in this proposal.  Then, \autoref{sec:estimation} describes how
this model may be estimated using an unscented Kalman filter.


\subsection{Modeling local fiber orientations} \label{sec:model}

In diffusion weighted imaging, image contrast is related to the strength of
water diffusion, and our goal is to accurately relate these signals to an
underlying model of fiber orientation.  At each image voxel, diffusion is
measured along a set of distinct gradients, $\u_1,...,\u_n\in\S^2$ (on the
unit sphere), producing the corresponding signal,
$\s=\rv{s_1,...,s_n}^T\in\R^n$.  For voxels containing a mixed diffusion
pattern, a general weighted formulation may be written as,
\begin{equation} \label{eq:general_model}
  s_i = s_0 \sum_j w_j e^{ -b \u_i^T D_j \u_i },
\end{equation}
where $s_0$ is a baseline signal intensity, $b$ is an acquisition-specific
constant, $w_j$ are convex weights, and $D_j$ is a tensor matrix describing a
diffusion pattern.

Considering a single tensor, we now follow the formulation of Rathi
\etal\cite{Rathi2009mia_w} to define the Watson directional function which
approximates the apparent diffusion pattern.  We begin by noting that any
diffusion tensor $D$ can be decomposed as $D = U \Lambda U^T$, where $U$ is a
rotation matrix and $\Lambda$ is a diagonal matrix with eigenvalues
$\{\lambda_1, \lambda_2, \lambda_3\}$.  These eigenvalues determine the shape
of the tensor:  ellipsoidal, planar, and spherical.  For example, if
$\lambda_1 > \lambda_2 > \lambda_3$, then the shape is ellipsoidal with the
major axis of the ellipsoid pointing to the eigenvector corresponding to
$\lambda_1$.  Intuitively, it represents strong diffusion along that
particular direction.  When $\lambda_1 = \lambda_2 > \lambda_3$, the shape is
planar indicating diffusion along orthogonal directions, and finally, when
$\lambda_1 = \lambda_2 = \lambda_3$, the diffusion is spherical (isotropic).

The most common of these configurations is ellipsoidal with principal
diffusion direction $\m$ and eigenvalue $\lambda_1$, and hence the first step
in introducing directional functions is to approximate the tensor by its first
eigenvector expansion: $D \approx \lambda_1 \m \m^T$.  Using this, each
exponent in \autoref{eq:general_model} may then be rewritten,
\begin{eqnarray} \label{eq:approx}
  -b \u_i^T D \u_i
  &\approx& -b\lambda_1\u_i^T\left(\m\m^T\right)\u_i \\
  &=& -b \lambda_1 \left(\u_i^T \m\right)^2 \\
  &=& -k \left(\u_i^T \m\right)^2 ,
\end{eqnarray}
where the scalar $k$ concentration parameter determines the degree of
anisotropy.  Finally, the general model may be restated:
\begin{equation} \label{eq:watson_model}
  s_i = A \sum_j w_j e^{-k_j (\u_i^T \m_j)^2} ,
\end{equation}
where $A$ is a normalization constant such that $\norm{\s}=1$.  For purposes
of comparison, this normalization will also be done to signals obtained from
scanner.  Note that, while the diffusion tensor requires six parameters, these
Watson functions require four parameters: three for the orientation vector
$\m$ and one concentration parameter $k$.  Employing a spherical
representation can further reduce the unit vector \m to two spherical
coordinates.  \autoref{fig:watson_k} demonstrates how adjusting the $k$-value
produces different diffusion patterns, and \autoref{fig:watson_multi}
illustrates two multi-fiber configurations.
\begin{figure}[t]
  \centering
  \subfloat[Reconstructed signals for strong diffusivity ($k=2$), weak
  diffusivity ($k=0.5$), and isotropic diffusion patterns ($k=0.01$).]{
    \label{fig:watson_k}
    \includegraphics[width=.19\columnwidth]{watson_strong}%
    \includegraphics[width=.19\columnwidth]{watson_weak}%
    \includegraphics[width=.19\columnwidth]{watson_iso}}%
  \hspace{1em}%
  \subfloat[Signals for two-fiber and three-fiber mixtures.]{
    \label{fig:watson_multi}
    \includegraphics[width=.19\columnwidth]{watson_two}%
    \includegraphics[width=.19\columnwidth]{watson_three}}
  \caption{Watson directional functions are capable of representing various
    diffusion patterns and fiber orientations.}
  \label{fig:watson}
\end{figure}

From this general mixture model, we choose to start with a restricted form
involving two equally-weighted Watson functions.  This choice is guided by
several previous studies.
%
Behrens \etal\cite{Behrens2007} showed that at a $b$-value of 1000 ms/mm$^2$
the maximum number of detectable fibers is two.  Several other studies have
also found two-fiber models to be sufficient
\cite{Tuch2002,Kreher2005,Zhan2006,Peled2006}.  Using this as a practical
guideline, we started with a mixture of two Watson functions as our local
fiber model.
%
\comment{Also, we assume the shape of each tensor to be ellipsoidal, \ie there
  is one dominant principal diffusion direction \m with eigenvalue \lx and the
  remaining orthonormal directions have equal eigenvalues $\ly=\lambda_3$
  \cite{Friman2006}.}
%
Further, following the study of \cite{Zhan2006}, we assume an equal
combination (50\%-50\%) of the two Watson functions.  While the effect of this
second choice appears to have little to no effect on experiments, we have yet
to quantify any potential loss in accuracy.  These assumptions leave us with
the following two-fiber model used in this study:
\begin{equation} \label{eq:model_2W}
  s_i = \frac{A}{2} \left(e^{ -k_1 (\u_i^T \m_1)^2 } + e^{ -k_2 (\u_i^T \m_2)^2 } \right) .
\end{equation}
where $k_1$ and $\m_1$ parameterize the first Watson function, $k_2$ and
$\m_2$ parameterize the second, and $A$ is again a normalization constant such
that $\norm{\s}=1$.  Thus, the equally-weighted two-fiber model is fully
described by the following parameters: $k_1$, $\m_1$, $k_2$, $\m_2$.
\comment{Extending off the two-Watson model, we can directly formulate the
  equally-weighted three-Watson model:
\begin{equation}   \label{eq:model_3W}
  s_i = \frac{A}{3} \sum_{j=1}^3 e^{-k_j (\u_i^T \m_j)^2} ,
\end{equation}
with the additional parameters $k_3$ and $\m_3$.}

Finally, from such parameters, Rathi \etal\cite{Rathi2009mia_w} describe how
one may compute the ODF analytically by applying the Funk-Radon transform
directly to \autoref{eq:watson_model}.  The ODF can be reconstructed directly
from the same parameters describing the signal without a separate estimation
process.  For the two-Watson model (\autoref{eq:model_2W}) the ODF is
approximated by,
\begin{equation} \label{eq:model_2W_odf}
  f_i = \frac{B}{2} \left(e^{ -\tfrac{k_1}{2} (1-(\u_i^T \m_1)^2) }
                        + e^{ -\tfrac{k_2}{2} (1-(\u_i^T \m_2)^2) } \right) ,
\end{equation}
\comment{and for the three-Watson model (\autoref{eq:model_3W}) this becomes,
\begin{equation}   \label{eq:model_3W_odf}
  f_i = \frac{B}{3} \sum_{j=1}^3 e^{-\tfrac{k_j}{2} (1-(\u_i^T \m_j)^2) } ,
\end{equation}}
where $B$ is a normalization factor such that $\sum_i f_i = 1$.



\subsection{Estimating the fiber model} \label{sec:estimation}

Given the measured scanner signal at a particular voxel, we want to estimate
the underlying model parameters that explain this signal.  As in streamline
tractography, we treat the fiber as the trajectory of a particle which we
trace out.  At each step, we examine the measured signal at that position,
estimate the underlying model parameters, and propagate forward in the most
consistent direction.  \autoref{fig:overview} illustrates this filtering
process.

To use a state-space filter for estimating the model parameters, we need the
application-specific definition of four filter components:
\begin{enumerate*}
\item The system state \x: the model parameters
\item The state transition $f[\cdot]$: how the model changes as we trace the fiber
\item The observation $h[\cdot]$: how the signal appears given a particular state
\item The measurement \y: the actual signal obtained from the scanner
\end{enumerate*}
For our state, we directly use the model parameters, thus the two-fiber model
in \autoref{eq:model_2W} has the following state vector:
\begin{equation} \label{eq:state}
  \x = \rv{\m_1 \;\; k_1 \;\; \m_2 \;\; k_2 }^T,
  \quad
  \m \in \S^2, k \in \R .
\end{equation}
While each $\m$ could be represented in a reduced spherical form, the
antipodes of the spherical parameterization would then introduce
nonlinearities which complicate estimation.
%
For the state transition we assume identity dynamics; the local fiber
configuration does not undergo drastic change from one position to the next.
Our observation is the signal reconstruction, $\y=\s=\rv{s_1,...,s_n}^T$ using
$s_i$ from \autoref{eq:model_2W}, and our measurement is the actual signal
interpolated directly from the diffusion weighted images at the current
position.

Since the signal reconstruction is a nonlinear process, we employ an unscented
Kalman filter to perform nonlinear estimation.  Similar to classical linear
Kalman filtering, the unscented version seeks to reconcile the predicted state
of the system with the measured state and addresses the fact that those two
processes (prediction and measurement) may be nonlinear or unknown.  It does
this in two phases:  first it uses the system transition model to predict the
next state and observation, and then it uses the new measurement to correct
that state estimate.  In what follows, we present the algorithmic application
of the filter.  For more thorough treatments, see \cite{Julier2004,Merwe2003}.

It is important to note two alternative techniques for nonlinear estimation.
%
First, particle filters are commonly used to provide a multi-modal estimate of
unknown systems.  With respect to an $n$-dimensional state space, particle
filters require the number of particles to be exponential to properly explore
the state space.  In contrast, the unscented filter requires only $2n+1$
particles (sigma points) for a Gaussian estimate that space.  Further, for
many slowly varying systems, the multi-modal estimate is unnecessary:  from
one voxel to the next, fibers tend not to change direction drastically.
%
Second, an extended Kalman filter may also be used to provide a Gaussian
estimate after linearizing the system; however, the unscented Kalman filter
provides a more accurate estimate with equivalent computational cost and
altogether avoids the attempt at linearization
\cite{Julier2004,Merwe2003,Lefebvre2004}.


Suppose the system of interest is at time $t$ and we have a Gaussian estimate
of its current state with mean, $\x_t \in \R^n$, and covariance, $P_t \in
\R^{n \times n}$.  Prediction begins with the formation of a set
$\X_t=\{\x_i\} \subset \R^n$ of $2n+1$ \textit{sigma point} states with
associated convex weights, $w_i \in \R$, each a perturbed version of the
current state.  We use the covariance, $P_t$, to distribute this set:
\begin{equation*}
  \x_0 = \x_t
  \qquad
  w_0 = \kappa/(n+\kappa)
  \qquad
  w_i = w_{i+n} = \tfrac{1}{2(n+\kappa)}
\end{equation*}
\begin{equation}   \label{eq:sigma_points}
  \x_{i}   = \x_t + \left[\sqrt{(n+\kappa)P_t}\right]_i
  \quad
  \x_{i+n} = \x_t - \left[\sqrt{(n+\kappa)P_t}\right]_i
\end{equation}
where $[A]_i$ denotes the $i^\text{th}$ column of matrix $A$ and $\kappa$ is
an adjustable scaling parameter ($\kappa = 0.01$ in all experiments).  Next,
this set is propagated through the state transition function, $\hat{\x}=f[\x]
\in \R^n$, to obtain a new predicted sigma point set:
$\X_{t+1|t}=\{f[\x_i]\}=\{\hat{\x}_i\}$.  Since in this study we assume the
fiber configuration does not change drastically as we follow it from one voxel
to the next, we may write this identity transition as, $\x_{t+1|t} = f[\x_t] =
\x_t $.  These are then used to calculate the predicted system mean state and
covariance,
\begin{equation*}
  \bar{\x}_{t+1|t} = \sum_i w_i ~ \hat{\x}_i ,
\end{equation*}
\begin{equation} \label{eq:Pxx}
  P_{xx} = \sum_i w_i \left(\hat{\x}_i - \bar{\x}_{t+1|t}\right)
                     \left(\hat{\x}_i - \bar{\x}_{t+1|t}\right)^T
           + Q ,
\end{equation}
where $Q$ is the injected process noise bias used to ensure a non-null spread
of sigma points and a positive-definite covariance.  This procedure comprises
the \textit{unscented transform} used to estimate the behavior of a nonlinear
function: spread sigma points based on your current uncertainty, propagate
those using your transform function, and measure their spread.

To obtain the predicted observation, we again apply the unscented transform
this time using the predicted states, $\X_{t+1|t}$, to estimate what we expect
to observe from the hypothetical measurement of each state:
$\y=h[\hat{\x}] \in \R^m$.  Keep in mind that, for this study, our
observation is the signal reconstruction from \autoref{eq:model_2W}, and the
measurement itself is the diffusion-weighted signal, $\s$, interpolated at the
current position.  From these, we obtain the predicted set of observations,
$\Y_{t+1|t} = \{ h[\hat{\x}_i] \} = \{ \y_i \}$, and may calculate its
weighted mean and covariance,
\begin{equation*}
  \bar{\y}_{t+1|t} = \sum_i w_i ~ \hat{\y}_i ,
\end{equation*}
\begin{equation} \label{eq:Pyy}
  P_{yy} = \sum_i w_i \left(\hat{\y}_i - \bar{\y}_{t+1|t}\right)
                     \left(\hat{\y}_i - \bar{\y}_{t+1|t}\right)^T
           + R ,
\end{equation}
where $R$ is the injected measurement noise bias again used to ensure a
positive-definite covariance.  The cross correlation between the estimated
state and measurement may also be calculated:
\begin{equation} \label{eq:Pxy}
  P_{xy} = \sum_i w_i \left(\hat{\x}_i - \bar{\x}_{t+1|t}\right)
                     \left(\hat{\y}_i - \bar{\y}_{t+1|t}\right)^T .
\end{equation}

\begin{algorithm}[t]
  \setlength\abovedisplayskip{2pt}
  \setlength\belowdisplayskip{2pt}
  \caption{Unscented Kalman Filter}
  \label{alg:ukf}
  \begin{algorithmic}[1]
    \STATE Form weighted sigma points $\X_t=\{w_i, \x_i\}_{i=0}^{2n}$ around
    current mean $\x_t$ and covariance $P_t$ with scaling factor $\zeta$
    \begin{equation*}
      \x_0 = \x_t
      \qquad
      \x_i    = \x_t + [\sqrt{\zeta P_t}]_i
      \qquad
      \x_{i+n} = \x_t - [\sqrt{\zeta P_t}]_i
    \end{equation*}
    \STATE Predict the new sigma points and observations
    \begin{equation*}
      \X_{t+1|t} = f[\X_t]   \qquad   \Y_{t+1|t} = h[\X_{t+1|t}]
    \end{equation*}
    \STATE Compute weighted means and covariances, \eg
    \begin{equation*}
      \bar{\x}_{t+1|t} = \sum_i w_i ~ \x_i
      \qquad
      P_{xy} = \sum_i w_i (\x_i - \bar{\x}_{t+1|t})(\y_i - \bar{\y}_{t+1|t})^T
    \end{equation*}
    \STATE Update estimate using Kalman gain $K$ and scanner measurement
    $\y_t$
    \begin{equation*}
      K = P_{xy}P_{yy}^{-1}
      \qquad
      \x_{t+1} = \bar{\x}_{t+1|t} + K(\y_t - \bar{\y}_{t+1|t})
      \qquad
      P_{t+1} = P_{xx} - K P_{yy} K^T
    \end{equation*}
  \end{algorithmic}
\end{algorithm}
As is done in the classic linear Kalman filter, the final step is to use the
Kalman gain, $K = P_{xy}P_{yy}^{-1}$, to correct our prediction and provide us
with the final estimated system mean and covariance,
\begin{equation} \label{eq:x_}
  \x_{t+1} = \bar{\x}_{t+1|t} + K(\y_t - \bar{\y}_{t+1|t})
\end{equation}
\begin{equation} \label{eq:P_}
  P_{t+1} = P_{xx} - K P_{yy} K^T ,
\end{equation}
where $\y_t \in \R^m$ is the actual signal measurement taken at this time.
\autoref{alg:ukf} summarizes this algorithm for unscented Kalman filtering.



\subsection{The algorithm} \label{sec:alg}

To summarize the proposed technique, we are using the unscented Kalman filter
to estimate the local model parameters as we trace out each fiber.  For each
fiber, we maintain the position at which we are currently tracing it and the
current estimate of its model parameters (mean and covariance).  At each
iteration of the algorithm, we predict the new state, which in this case is
simply identity:  $\x_{t+1|t}=\x_t$.  Our actual measurement $\y_t$ in
\autoref{eq:x_} is the diffusion-weighted signal, \s, recorded by the scanner
at this position.  At subvoxel positions we interpolate directly on the
diffusion-weighted images.  With these, we step through the equations above to
find the new estimated model parameters, $\x_{t+1}$.  Last, we use
second-order Runge-Kutta integration to move a small step in the most
consistent of principal diffusion directions, and then we repeat these steps
from that new location.  \autoref{alg:loop} outlines the integration of
filtering and tractography.
\begin{algorithm}[t]
  \setlength\abovedisplayskip{2pt}
  \setlength\belowdisplayskip{2pt}
  \caption{Main loop repeated for each fiber}
  \label{alg:loop}
  \begin{algorithmic}[1]
    \REPEAT
      \STATE Form the sigma points $\X_t$ around $\x_t$
      \STATE Predict the new sigma points $\X_{t+1|t}$ and observations
      $\Y_{t+1|t}$
      \STATE Compute weighted means and covariances, \eg $\bar{\x}_{t+1|t}$, $P_{xy}$
      \STATE Update estimate ($\x_{t+1}$, $P_{t+1}$) using scanner
      measurement ($\y_t$)
      \STATE Proceed in most consistent direction $\m_j$
    \UNTIL{estimated model appears isotropic}
  \end{algorithmic}
\end{algorithm}






\chapter{Preliminary Results}

We first use experiments with synthetic data to validate our technique against
ground truth.  We confirm that our approach accurately estimates the true
underlying signal and reliably recognizes crossing fibers over a broad range
of angles.  Comparing against two alternative multi-fiber optimization
techniques, we find the filtered approach gives consistently superior results
in both respects (\autoref{sec:mse_angle}).
%
Next, we perform tractography through crossing fiber fields and examine the
underlying orientations and branchings (\autoref{sec:tracts}).
%
Lastly, we examine a real dataset to demonstrate how causal estimation is able
to pick up fibers and branchings known to exist \textit{in vivo} yet absent
using other techniques (\autoref{sec:real}).

\begin{figure}[t]
  \centering
  \subfloat[Ground truth signal and ODF (\bone, \bthree).]{
    \label{fig:samples_pure}
    \includegraphics[width=.24\columnwidth]{samples_b1000_pure}%
   \includegraphics[width=.24\columnwidth]{samples_b3000_pure}}%
  \subfloat[Low- and high-noise signals (\bone, \bthree).]{
    \label{fig:samples_noisy}
    \includegraphics[width=.24\columnwidth]{samples_b1000_noisy}%
   \includegraphics[width=.24\columnwidth]{samples_b3000_noisy}}
  \caption{Synthetic two-fiber voxel signals at a 60\deg angle \textit{(black
      wires indicate principal diffusion directions)}.  Each column shows the
    same surface from two viewpoints.  \subref{fig:samples_pure} shows the
    ground truth signal and corresponding true ODF \textit{(left to right)}.
    \subref{fig:samples_noisy} shows the corrupted versions of the ground
    truth signal \textit{(left to right)}.}
  \label{fig:samples}
\end{figure}
Following the experimental method of generating synthetic data found in
\cite{Tournier2004,Descoteaux2009tmi,Schultz2008}, we pull from our real data
set the 300 voxels with highest fractional anisotropy (FA) and compute the
average eigenvalues among these voxels:  $\{1200, 100, 100\}\mu$m$^2$/msec
(FA=0.91).  We generated synthetic MR signals according to
\autoref{eq:general_model} using these eigenvalues to form an anisotropic
tensor at both \bone and \bthree, using 81 gradient directions uniformly
spread on the hemisphere, and assume $s_0=1$.  We generate two separate data
sets, each with a different level of Rician noise: low-noise ($\sigma=0.1$,
\snr{10}) and high-noise ($\sigma=0.2$, \snr{5}).  To get an idea of this
level of noise, \autoref{fig:samples} visualizes a sample voxel with two
fibers at a 60\deg angle.

\begin{figure}[t]
  \centering
  \subfloat[\bone]{
    \includegraphics[width=.24\columnwidth]{mse_2W_b1000_clean}%
    \includegraphics[width=.24\columnwidth]{mse_2W_b1000_dirty}}%
  \subfloat[\bthree]{
    \includegraphics[width=.24\columnwidth]{mse_2W_b3000_clean}%
    \includegraphics[width=.24\columnwidth]{mse_2W_b3000_dirty}}
  \caption{Mean squared error (MSE) between reconstructed signal and ground
    truth signal at various crossing angles \textit{(low-noise on left,
      high-noise on right)}.  Notice how the increased noise has little effect
    on the filter \textit{(black)} compared to using matching pursuit
    \textit{(blue)} or sharpened spherical harmonics \textit{(red)}.}
  \label{fig:mse}
\end{figure}
Throughout the experiments, we draw comparison to three alternative
techniques.
%
First, we use the same two-Watson model from \autoref{sec:model} with a
variant of matching pursuit for brute force, dictionary-based optimization
\cite{Mallat1993}.  In our implementation, we construct a finite dictionary of
two-Watson signals at a range of various $k$-values, essentially discretizing
the search space across orientations and $k$-values.  Given a new measured
signal, the signal from the dictionary with highest inner product provides an
estimate of orientation and concentration.  While our signal is produced at 81
gradient directions, we use 341 directions to construct the dictionary, thus
any error is due to the method's sensitivity to noise and discretization.
Note that by using 341 orientation directions there is roughly an 8\deg
angular difference between offset orientations, hence we see that the angular
error is often at most 8\deg.  This approach highlights the effect of using
the same model but changing the optimization technique to one that treats each
voxel independently.
%
Second, we use spherical harmonics for modeling \cite{Tournier2004} and
fiber-ODF sharpening for peak detection as described in \cite{Descoteaux2009tmi}
(order $l=8$, regularization $L=0.006$).  This provides a comparison with an
independently estimated, model-free representation.  Note that this technique
is very similar to spherical deconvolution.
%
Last, when performing tractography on real data, we use single-tensor
streamline tractography as a baseline\footnote{Using the freely available
  Slicer 2.7 (\url{http://www.slicer.org}).}.

\begin{figure*}[t]
  \centering
  \subfloat[Matching pursuit, \bone]{ \label{fig:angle_2W_MP_1000}
    \includegraphics[width=.24\columnwidth]{angle_2W_b1000_PP_clean}
    \includegraphics[width=.24\columnwidth]{angle_2W_b1000_PP_dirty}}%
  \subfloat[Matching pursuit, \bthree]{ \label{fig:angle_2W_MP_3000}
    \includegraphics[width=.24\columnwidth]{angle_2W_b3000_PP_clean}
    \includegraphics[width=.24\columnwidth]{angle_2W_b3000_PP_dirty}}
  \subfloat[Spherical harmonics, \bone]{ \label{fig:angle_2W_SH_1000}
    \includegraphics[width=.24\columnwidth]{angle_2W_b1000_SH_clean}
    \includegraphics[width=.24\columnwidth]{angle_2W_b1000_SH_dirty}}%
  \subfloat[Spherical harmonics, \bthree]{ \label{fig:angle_2W_SH_3000}
    \includegraphics[width=.24\columnwidth]{angle_2W_b3000_SH_clean}
    \includegraphics[width=.24\columnwidth]{angle_2W_b3000_SH_dirty}}
  \caption{Average angle error at various crossing angles comparing all three
    techniques: matching pursuit \textit{(blue)}, sharpened spherical
    harmonics \textit{(red)}, and the proposed filter \textit{(black)}.  The
    filter provides stable and consistent estimation compared to either
    alternative technique.  Each subfigure shows both the low-noise and
    high-noise experiments \textit{(left, right)}.}
  \label{fig:angle}
\end{figure*}
The unscented Kalman filter conveniently requires few parameters.
Specifically, of importance are the matrices for injecting model noise $Q$ and
injecting measurement noise $R$ (see \autoref{eq:Pxx} and \autoref{eq:Pyy}).
Fortunately, the relative magnitude of each can be determined off-line from
the data itself.  We found that values on the order of $q_\m = 0.001$ (roughly
2\deg), $q_\lambda = 10$, and $r_s=0.02$ were quite robust for the appropriate
diagonal entries of $Q$ and $R$.  Off-diagonal entries were left at zero.



\section{Signal reconstruction and angular resolution}  \label{sec:mse_angle}

While the independent optimization techniques can be run on individually
generated voxels, care must be taken in constructing reasonable scenarios to
test the causal filter.  For this purpose, we constructed an actual 2D field
through which to navigate (see \autoref{fig:single_fiber} and
\autoref{fig:crossing}).  In the middle is one long fiber pathway where the
filter begins estimating a single component but then runs into a field of
voxels with two crossed fibers at a fixed angle.  In this crossing region we
calculated error statistics.  Similarly, we computed the angular error over
this region using both sharpened spherical harmonics and matching pursuit.  We
generated several similar fields, each at a different fixed angle.  By varying
the size of the crossing region or the number of fibers run, we ensured that
each technique performed estimation on at least 500 voxels.

In the first experiment, we look at signal reconstruction error.  We calculate
the mean squared error of the reconstructed signal, \s, against the ground
truth signal, $\hat{\s}$ (pure, no noise): $ \norm{\s - \hat{\s}}^2 /
\norm{\hat{\s}}^2$.  In essence, this is exactly what the filter is trying to
minimize: the error between the reconstructed signal and the measured signal.
\autoref{fig:mse} shows the results of using the proposed filter, matching
pursuit, and spherical harmonics.  Over each technique's series of
estimations, the trendlines indicate the mean error while the bars indicate
one standard deviation.  Spherical harmonics \textit{(red)} appear to produce
a smooth fit to the given noisy data, while matching pursuit \textit{(blue)}
shows the effect of discretization and sensitivity to noise.  The two raised
areas are a result of the dictionary being constructed with an 8\deg minimum
separation between any pair of orientations.  This experiment demonstrates
that the proposed filter \textit{(black)} accurately and reliably estimates
the true underlying signal.

In the second experiment, we looked at the error in angular resolution by
comparing the filtered approach to matching pursuit and sharpened spherical
harmonics.  \autoref{fig:angle_2W_MP_1000} and \autoref{fig:angle_2W_MP_3000}
show the sensitivity of matching pursuit.  Consistent with the results
reported in \cite{Descoteaux2009tmi,Descoteaux2007mrm}, spherical harmonics are
generally unable to detect and resolve angles below 50\deg for \bone or below
40\deg for \bthree.  \autoref{fig:angle_2W_SH_1000} and
\autoref{fig:angle_2W_SH_3000} confirm this, respectively.  This experiment
demonstrates that for \bone, the filtered approach consistently resolves
angles down to 20-30\deg with 5\deg error compared to independent optimization
which fails to reliably resolve below 60\deg with as much as 15\deg error.
For \bthree, the filtered approach consistently resolves down to 20-30\deg
with 2-3\deg error compared to independent optimization which cannot resolve
below 50\deg with 5\deg error.





\begin{figure}[t]
  \centering
  \includegraphics[width=.5\columnwidth]{compare_b3000_2W}
  \caption{Fiber passing through 40\deg, 50\deg, and 60\deg synthetic
    crossings (\bthree, noisy).  Blue dashes represent the orientation of the
    second fiber when detected.}
  \label{fig:single_fiber}
\end{figure}
\begin{figure}[t]
  \centering
  \subfloat[Fiber passing through 60\deg synthetic crossing.]{
    \label{fig:crossing}
    \includegraphics[width=.3\columnwidth]{compare_tract_b1000_2W}}%
  \subfloat[Estimated ODFs along fiber.]{
    \label{fig:zooms}
    \includegraphics[width=.3\columnwidth]{compare_tract_b1000_2W_zoom}}
  \caption{One fiber passing through an example synthetic field (\bone, noisy)
    and the estimated ODFs using spherical harmonics and the filter as it
    passes through the crossing region \textit{(blue box)}.  The filter
    provides consistent angular resolution along the fiber while independent
    spherical harmonic modeling at those same locations misses the second
    fiber in two voxels.}
\end{figure}


\section{Synthetic tractography}  \label{sec:tracts}

\begin{figure*}[t]
  \centering
  \subfloat[Single-tensor]{ \label{fig:1T_tc}%
    \includegraphics[width=.5\columnwidth]{case01045_1T_tc}}%
  \subfloat[Spherical harmonics]{ \label{fig:SH_tc}%
    \includegraphics[width=.5\columnwidth]{case01045_SH_tc}}

  \subfloat[Filtered two-Watson]{ \label{fig:2W_tc}%
    \includegraphics[width=.5\columnwidth]{case01045_2W_tc}}%
  \subfloat[Closeup of upper right in \subref{fig:2W_tc}.]{ \label{fig:2W_zoom_tc}%
    \includegraphics[width=.5\columnwidth]{case01045_2W_tc_zoom}}
  \caption{Filtered tractography picks up many fiber paths consistent with the
    underlying structures.  Both single-tensor streamline and sharpened
    spherical harmonics are unable to find the majority of these pathways.
    Fibers existing $\pm22\,\text{mm}$ around the mid-sagittal plane are
    indicated in blue.  Seed region indicated in yellow.}
  \label{fig:tc}
\end{figure*}
Having verified the technique's accuracy, we now turn to the resulting
tractography.  \autoref{fig:single_fiber} provides examples of synthetic
crossing fiber fields each at different fixed angles: 40\deg, 50\deg, 60\deg
(\bthree, noisy).  In our experiments, we start fibers from the bottom and
propagate upward where they encounter the crossing region.  Here we show one
such fiber and use blue glyphs to indicate the second component detected as
the it passes through the crossing region.  In general, we found that in
regions with only one true fiber present (those outside the crossing), the
second component either aligned with the first or adjusted its concentration
parameter to fill out the isotropic component of the signal.  Further, we
found the filtering strategy to be robust with respect to initial
configuration and choice of injected noise matrices $Q$ and $R$
(\autoref{eq:Pxx} and \autoref{eq:Pyy}).

In \autoref{fig:crossing} we show another 60\deg field (\bone, noisy) but take
a closer look at several points along a single fiber as it passes through the
crossing region.  We also examine the corresponding ODFs reconstructed using
sharpened spherical harmonics and the proposed filter.  As expected, the
sharpened spherical harmonics often do not detect the crossing but result in a
single angle as seen in the middle two samples in \autoref{fig:zooms}.  A
close examination of the reported axes shows the bias toward a single averaged
axis as reported in \cite{Zhan2006,Tournier2007,Schultz2008}.  In contrast,
the filtered results are consistent and accurate.



\section{\textit{In vivo} tractography}  \label{sec:real}

We tested our approach on a real human brain scan acquired on a 3-Tesla GE
system using an echo planar imaging (EPI) diffusion weighted image sequence.
A double echo option was used to reduce eddy-current related distortions.  To
reduce impact of EPI spatial distortion, an eight channel coil was used to
perform parallel imaging using Array Spatial Sensitivity Encoding Techniques
(GE) with a SENSE-factor (speed-up) of 2.  Acquisitions have 51 gradient
directions with \bn{900} and eight baseline scans with \bn{0}.  The original
GE sequence was modified to increase spatial resolution, and to further
minimize image artifacts.  The following scan parameters were used: TR 17000
ms, TE 78 ms, FOV 24 cm, 144x144 encoding steps, 1.7 mm slice thickness. All
scans had 85 axial slices parallel to the AC-PC line covering the whole brain.
In addition, \bn{0} field inhomogeneity maps were collected and calculated.

We first focused on fibers originating in the corpus callosum.  Specifically,
we sought to trace out the lateral transcallosal fibers that run through the
corpus callosum out to the lateral gyri.  It is known that single-tensor
streamline tractography only traces out the dominant pathways forming the
U-shaped callosal radiation (\autoref{fig:1T_tc} and \autoref{fig:1T_cc}).
Several studies document this phenomena, among them the works of Descoteaux
\etal\cite{Descoteaux2009tmi} and Schultz and Seidel \cite{Schultz2008} have
side-by-side comparisons.  These fibers have been reported in using diffusion
spectrum imaging \cite{Hagmann2004}, probabilistic tractography
\cite{Kaden2007,Anwander2007,Descoteaux2009tmi}, and more recently with tensor
decomposition \cite{Schultz2008}.


We start with two basic experiments:  first examining the tracts surrounding a
single coronal slice and second looking at all tracts passing through the
corpus callosum.  We seed each algorithm multiple times in voxels at the
intersection of the mid-sagital plane and the corpus callosum.  To explore
branchings found using the proposed technique, we considered a component to be
branching if it was separated from the primary component by less than 40\deg
with $k\ge0.6$.  Similarly, with sharpened spherical harmonics, we considered
it a branch if we found additional maxima over the same range.  We terminated
fibers when the general fractional anisotropy of the estimated signal
(std/rms) fell below 0.1.  While such heuristics are somewhat arbitrary, we
found little qualitative difference in adjusting these values.

\begin{figure*}[t]
  \centering
  \subfloat[Single-tensor]{ \label{fig:1T_cc}
    \begin{minipage}[b]{0.33\linewidth}
      \includegraphics[width=\textwidth]{case01045_1T_cc_top} \\
      \includegraphics[width=\textwidth]{case01045_1T_cc_front}
    \end{minipage}}%
  \subfloat[Spherical harmonics]{ \label{fig:SH_cc}
    \begin{minipage}[b]{0.33\linewidth}
      \includegraphics[width=\textwidth]{case01045_SH_cc_top}
      \includegraphics[width=\textwidth]{case01045_SH_cc_front}%
    \end{minipage}}%
  \subfloat[Filtered Watson]{ \label{fig:2W_cc}
    \begin{minipage}[b]{0.33\linewidth}
      \includegraphics[width=\textwidth]{case01045_2W_cc_top}
      \includegraphics[width=\textwidth]{case01045_2W_cc_front}
    \end{minipage}}
 \caption{Tracing fibers originating from the center of the entire corpus
    callosum with views from above \textit{(top rows)} and front-to-back
    \textit{(bottom rows)}.  The proposed filtered tractography is able to
    find many of the lateral projections \textit{(blue)} while single-tensor
    is unable to find any and few are found with sharpened spherical
    harmonics.  Seed region indicated in yellow.}
  \label{fig:cc}
\end{figure*}

\begin{figure}[t]
  \centering
  \includegraphics[width=.8\columnwidth]{case01045_2W_cc_zoom}
  \caption{Closeup of frontal fibers in \autoref{fig:2W_cc} viewed from
    above.}
  \label{fig:frontal}
\end{figure}
\begin{figure}[t]
  \centering
  \subfloat[Single-tensor]{ \label{fig:1T_ic_front}%
    \includegraphics[width=.33\columnwidth]{case01045_1T_ic_front}}%
  \subfloat[Spherical harmonics]{ \label{fig:SH_ic_front}%
    \includegraphics[width=.33\columnwidth]{case01045_SH_ic_front}}%
  \subfloat[Filtered Watson]{ \label{fig:2W_ic_front}%
    \includegraphics[width=.33\columnwidth]{case01045_2W_ic_front}}%
 \caption{Frontal view with seeding in the internal capsule \textit{(yellow)}.
    While both single-tensor and spherical harmonics tend to follow the
    dominant corticospinal tract to the primary motor cortex, the filtered
    approach follows many more pathways.  Seed region indicated in yellow.}
  \label{fig:ic_front}
\end{figure}
\begin{figure}[t]
  \centering
  \subfloat[Single-tensor]{ \label{fig:1T_ic_top}%
    \includegraphics[width=.33\columnwidth]{case01045_1T_ic_top}}%
  \subfloat[Spherical harmonics]{ \label{fig:SH_ic_top}%
    \includegraphics[width=.33\columnwidth]{case01045_SH_ic_top}}%
  \subfloat[Filtered Watson]{ \label{fig:2W_ic_top}%
    \includegraphics[width=.33\columnwidth]{case01045_2W_ic_top}}%
 \caption{View from above showing cortical insertion points for each method.
    FA backdrop is taken near the top of the brain.  The filtered approach
    shows more lateral insertions compared to single-tensor and spherical
    harmonic tracts.}
  \label{fig:ic_top}
\end{figure}
For the first experiment, \autoref{fig:tc} shows tracts originating from
within a few voxels intersecting a particular coronal slice.  For a reference
backdrop, we use a coronal slice showing the intensity of fractional
anisotropy (FA) placed a few voxels behind the seeded coronal position.
Keeping in mind that these fibers are intersecting or are in front of the
image plane, this roughly shows how the fibers navigate the areas of high
anisotropy (bright regions).  Similar to the results in
\cite{Descoteaux2009tmi,Schultz2008}, \autoref{fig:SH_tc} shows that sharpened
spherical harmonics only pick up a few fibers intersecting the U-shaped
callosal radiata.  In contrast, our proposed method traces out many pathways
consistent with the apparent anatomy.  To emphasize transcallosal tracts, we
color as blue those fibers exiting a corridor of $\pm22\,\text{mm}$ around the
mid-sagittal plane.  \autoref{fig:2W_zoom_tc} provides a closer inspection of
\autoref{fig:2W_tc} where, to emphasize the underlying anatomy influencing the
fibers, we use as a backdrop the actual coronal slice passing through the
voxels used to seed this run.  Such results are obtained in minutes in our
current MATLAB implementation.  At each step, the cost of reconstructing the
signal for few sigma points approaches the cost of a few iterations of
weighted least-squares estimation of a single tensor.

For the second experiment, \autoref{fig:cc} shows a view of the whole brain to
see the overall difference between the different methods.  Here again we
emphasize with blue the transcallosal fibers found using the proposed filter.
To show the various pathways infiltrating the gyri, \autoref{fig:frontal}
provides a closeup of the frontal lobe from above (without blue emphasis).

Next we examined fibers passing through the internal capsule to trace out the
pathways reaching up into the primary motor cortex at the top of the brain as
well as down into the hippocampal regions near the brain stem.
\autoref{fig:ic_front} shows frontal views for each technique with seeding
near the cerebral peduncles \textit{(yellow)}.  \autoref{fig:ic_side} shows
this same result from a side view where we can see that the filtered approach
picks up the corticospinal pathways.  As reported elsewhere
\cite{Behrens2007}, single-tensor tractography follows the dominant
corticospinal tract to the primary motor cortex.  The same pathways were also
found with spherical harmonics.
%
\autoref{fig:ic_top} shows a view from above where we use a transverse FA
image slice near the top of the brain as a backdrop so we can focus on the
fiber endpoints.  From this we can see how each method infiltrates the sulci
grooves, and specifically we see that the filtered method is able to
infiltrate sulci more lateral compared to single-tensor tractography.

\begin{figure*}[t]
  \centering
  \subfloat[Single-tensor]{ \label{fig:1T_ic_side}%
    \includegraphics[width=.33\textwidth]{case01045_1T_ic_left}}%
  \subfloat[Spherical harmonics]{ \label{fig:SH_ic_side}%
    \includegraphics[width=.33\textwidth]{case01045_SH_ic_left}}%
  \subfloat[Filtered Watson]{ \label{fig:2W_ic_side}%
    \includegraphics[width=.33\textwidth]{case01045_2W_ic_left}}%
 \caption{Side view with seeding in the internal capsule \textit{(yellow)}.
    Filtered tractography finds many insertions into cortical regions of the
    parietal and occipital lobes.  Seed region indicated in yellow.}
  \label{fig:ic_side}
\end{figure*}

Note that in the region of intersection between the transcallosal fibers, the
corticospinal, and the superior longitudinal fasciculus, the partial voluming
of each of these pathways leads the filter to report several end-to-end
connections that are not necessarily present, \eg fibers originating in the
left internal capsule do not pass through this region, through the corpus
callosum and then insert into the right motor cortex.  Many of the lateral
extensions are callosal fibers that are picked up while passing through this
juncture.  It is our hope that such connections may be avoided with the
introduction of weighted mixtures, alternative filter formulations, or
different heuristic choices in the algorithm.


\chapter{Proposed Research}

The goal of this thesis is to develop an accurate and reliable method of
tracing neural pathways.  This entails development of the technique itself and
comparison against relevant alternatives, as well as the demonstration of
various applications.

\section{Work Completed}
The preliminary results in this proposal are contained in
\cite{Malcolm2010watson,Rathi2009ismrm}, where we introduced this method of
model-based filtering for tractography using compact directional functions.
Our preliminary results indicate that these filter-based approaches to
tractography provide superior results compared to independent streamline
estimation.


\section{Work Remaining}

There are several phases defining the remainder of this thesis:

\begin{enumerate}
\item Efforts thus far have focused on using Watson directional functions to
  model the local diffusion.  To show the flexibility and stability of this
  filtering framework, we will apply this same methodology to incorporate
  Gaussian tensor models with will require additional parameters introducing
  further nonlinearities.

\item The next phase of this thesis will then focus on weighted mixtures for
  further flexibility in modeling partial volumes.

\item We have introduced a framework for fitting a diffusion model at a
  particular voxel based on the local signal.  However, our experience
  suggests that it is important to take into account information beyond the
  local signal.  Along those lines, various global techniques have been
  proposed to estimate entire paths at once
  \cite{Fillard2009,Reisert2009,Kreher2008}; although, these focus on full
  connecting path and not accurate estimation of the local diffusion model.
  Our goal here is to develop a hybrid solution:  local filtering for model
  estimation combined with global filtering for path estimation.  For example,
  instead of estimating a local principle diffusion direction at each point,
  one could estimate an underlying curve to match the fiber geometry beyond a
  single voxel \cite{Savadjiev2007}.

\item We begin validation on a physical phantom \cite{Poupon2008phantom}.

\item We will deliver a version of this algorithm as a Slicer module
  (\url{http://www.slicer.org}).

\item The final phase of this thesis will entail applying these techniques in
  the context of a population study where we study the effect of local model
  choice, \eg single- or two-tensor models.  Borrowing from our earlier work
  on nonparametric density estimation of manifold data \cite{Malcolm2007tc},
  we will explore the suitability of this same modeling technique to the
  analysis of tractography data among individuals and groups.
\end{enumerate}

The completion of this thesis requires the use of a computer, a MATLAB
license, 3D Slicer, and access to diffusion MRI from patients--all currently
available through Georgia Tech and our collaborations.  We expect this work to
be ready for presentation in Summer 2010.

\begin{postliminary}
\references
\end{postliminary}

\end{document}
