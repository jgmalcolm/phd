% 1 Sep 2010 -- 2h -- apartment
% 8 Sep 2010 -- 1hr -- SBUX
% 9 Sep 2010 -- 30m -- office
% 14 Sep 2010 -- 20m -- ChikFila w Rome

\documentclass[final,hyperref]{gatech-thesis}
\usepackage{amsfonts,amsmath,amssymb,latexsym}
\usepackage{fixltx2e} % maintain order of floats
\usepackage{cite,mdwlist}
\usepackage{color}
\usepackage{graphicx}
\graphicspath{{/Users/malcolm/junk/writeups/fibers/figs/ukf/}
              {/Users/malcolm/junk/writeups/fibers/figs/tensor_TMI/}
              {/Users/malcolm/junk/writeups/fibers/figs/tensor_2t/}
              {/Users/malcolm/junk/writeups/fibers/figs/tensor_NI/}
              {/Users/malcolm/junk/writeups/fibers/figs/miccai_2TW/}
              {/Users/malcolm/junk/writeups/fibers/videos/2T_cc_ic_zoom/}
              {/Users/malcolm/junk/writeups/fibers/figs/watson/}
              {/Users/malcolm/junk/writeups/fibers/figs/watson_response/}
              {/Users/malcolm/Dropbox/malcolm-rathi-westin/chapter/figures/}
              {/Users/malcolm/junk/writeups/fibers/figs/miccai_study/}
              {/Users/malcolm/junk/writeups/fibers/figs/fc/}}

\usepackage[draft] {hyperref}
%\definecolor{dblue}{rgb}{0,0,.5}
%\usepackage[colorlinks=true,citecolor=dblue,linkcolor=dblue,urlcolor=blue,pdfborder={0 0 0},pagebackref=true]{hyperref}
\usepackage[all]{hypcap} % hyperref/caption fix -- subfig broken
\usepackage[font=footnotesize]{subfig}
\renewcommand{\equationautorefname}{Equation}
\newcommand{\subfigureautorefname}{Figure}
\renewcommand{\figureautorefname}{Figure}
\renewcommand{\sectionautorefname}{Section}
\renewcommand{\subsectionautorefname}{Section}
\newcommand{\algorithmautorefname}{Algorithm}
\newcommand{\citet}[1]{\cite{#1}} % since no natbib

\newcommand{\region}[1]{\textit{#1}\xspace}
\newcommand{\Green}{\region{Caudalmiddle frontal}}
\newcommand{\Blue}{\region{Precentral}}
\newcommand{\Red}{\region{Superiorfrontal}}
\newcommand{\green}{\region{caudalmiddle frontal}}
\newcommand{\blue}{\region{precentral}}
\newcommand{\red}{\region{superiorfrontal}}

%makes figures labeled ChapterNumber.FigureNumber instead of FigureNumber
\renewcommand{\theequation}{\thechapter.\arabic{equation}}
\renewcommand{\thefigure}{\thechapter.\arabic{figure}}
\renewcommand{\thetable}{\thechapter.\arabic{table}}
\newcommand{\thealgorithm}{\thechapter.\arabic{algorithm}}

%makes autoref show Equation (#) instead of Equation #
\makeatletter
\def\tagform@#1{\maketag@@@{\ignorespaces#1\unskip\@@italiccorr}}
\let\orgtheequation\theequation
\def\theequation{(\orgtheequation)}
\makeatother

\hypersetup{
    bookmarks=true,
    pdftitle={Filtered Tractography},
    pdfauthor={James G. Malcolm},
    pdfsubject={Computers},
    pdfkeywords={},
}

\usepackage{algorithm,algorithmic}
\floatname{algorithm}{Algorithm}

\usepackage{xspace}
\renewcommand{\deg}{\ensuremath{^\circ}\xspace}
\renewcommand{\v}[1]{\ensuremath{\mathbf #1}\xspace}
\newcommand{\rv}[1]{\ensuremath{[\, #1 \,]}\xspace} % row vector

\makeatletter
\DeclareRobustCommand\onedot{\futurelet\@let@token\@onedot}
\def\@onedot{\ifx\@let@token.\else.\null\fi\xspace}
\newcommand{\ie}{\textit{i.e}\onedot}
\newcommand{\eg}{\textit{e.g}\onedot}
\newcommand{\etal}{et al\onedot}
\makeatother

\newcommand{\R}{\ensuremath{\mathbb R}}
\renewcommand{\S}{\ensuremath{\mathbb S}}

\newcommand{\s}{\v s}
\newcommand{\bn}[1]{\ensuremath{b\!=\!#1}\xspace}
\newcommand{\bone}{\bn{1000}}
\newcommand{\bthree}{\bn{3000}}
\newcommand{\snr}[1]{SNR $\approx$ #1 dB}
\renewcommand{\u}{\v u}
\newcommand{\invivo}{\textit{in vivo}\xspace}
\newcommand{\Invivo}{\textit{In vivo}\xspace}

% tensor
\newcommand{\m}{\v m}
\newcommand{\lx}{{\ensuremath{\lambda_1}} \xspace}
\newcommand{\ly}{{\ensuremath{\lambda_2}} \xspace}

% UKF
\newcommand{\norm}[1]{\ensuremath{\|#1\|}}
\newcommand{\x}{\v x}
\newcommand{\y}{\v y}
\newcommand{\X}{\v X}
\newcommand{\Y}{\v Y}

\title{Filtered Tractography}
\author{James G. Malcolm}
\department{Electrical and Computer Engineering}

%% Can have up to six readers, plus principaladvisor and
%% committeechair. All have the form
%%
%%  \reader{Name}[Department][Institution]
%%
%% The second and third arguments are optional, but if you wish to
%% supply the third, you must supply the second. Department defaults
%% to the department defined above and Institution defaults to Georgia
%% Institute of Technology.

\principaladvisor{Yogesh Rathi}[Department of Psychiatry][Harvard Medical School]
\firstreader{Allen Tannenbaum}
\secondreader{Anthony Yezzi}
\thirdreader{Patricio Vela}
\fourthreader{Hongwei Wu}
\fifthreader{Sylvain Bouix}[Department of Psychiatry][Harvard Medical School]

\degree{Doctor of Philosophy}

\copyrightyear{2010}
\approveddate{21 Oct 2010}
\submitdate{May 2011}

\bibfiles{references}

\dedicationheadingfalse
\thesisproposalfalse

\begin{document}



\bibliographystyle{gatech-thesis}
\begin{preliminary}
\contents


\begin{summary}
  Computer vision encompasses a host of computational techniques to process
  visual information.
  %
  Medical imagery is one particular area of application where data comes in
  various forms: X-rays, ultrasound probes, MRI volumes, EEG recordings, NMR
  spectroscopy, etc.
  %
  This dissertation is concerned with techniques for accurate reconstruction
  of neural pathways from diffusion magnetic resonance imagery (dMRI).

  This dissertation describes a filtered approach to neural tractography.
  %
  Existing methods independently estimate the diffusion model at each voxel so
  there is no running knowledge of confidence in the estimation process.
  %
  We propose using tractography to drive estimation of the local diffusion
  model.
  %
  Toward this end, we formulate fiber tracking as recursive estimation: at
  each step of tracing the fiber, the current estimate is guided by those
  previous.

  We argue that this approach is more accurate than conventional techniques.
  Experiments demonstrate that this filtered approach significantly improves
  the angular resolution at crossings and branchings.  Further, we confirm its
  ability to trace through regions known to contain such crossing and
  branching while providing inherent path regularization.

  We also argue that this approach is flexible.  Experiments demonstrate using
  various models in the estimation process, specifically combinations of
  Watson directional functions and rank-2 tensors.  Further, this dissertation
  includes an extension of the technique to weighted mixtures using a
  constrained filter.
\end{summary}

\end{preliminary}

\chapter{Background}

\section{Overview of Diffusion Imaging}

\begin{figure}[t]
  \centering
  \subfloat[Slice indicating region of interest]{ \label{fig:roi}
    \includegraphics[width=.45\columnwidth]{roi}}%

  \subfloat[Original signal]{ \label{fig:signal} %
   \includegraphics[width=.50\textwidth]{signal}}%
  \subfloat[Single-tensor with axis]{ \label{fig:1t} %
   \includegraphics[width=.50\textwidth]{odf_1t}}%

  \subfloat[Two-tensor axes]{ \label{fig:2t} %
   \includegraphics[width=.50\textwidth]{odf_2t_axes_cp}}%
  \subfloat[Spherical harmonic signal]{ \label{fig:sh} %
   \includegraphics[width=.50\textwidth]{odf_sh_signal}}%
 \caption{Comparison of various models within a coronal slice \subref{fig:roi}
   passing through the corpus callosum.  In \subref{fig:signal} the original
   signal appears noisy.  In \subref{fig:1t} a single tensor fit provides a
   robust estimate of the principal diffusion direction.  In \subref{fig:2t} a
   two-tensor model is fit to planar voxels and the two axes are reported
   \cite{Peled2006}.  In \subref{fig:sh} spherical harmonics provide a
   smoothed non-parametric estimate of the signal surface eliminating much of
   the noise seen in \subref{fig:signal} \cite{Descoteaux2007mrm}.}
\end{figure}
The advent of diffusion magnetic resonance imaging (dMRI) has provided the
opportunity for non-invasive investigation of neural architecture.
%
While structural MRI has long been used to image soft tissue and bone, dMRI
provides additional insight into tissue microstructure by measuring its
microscopic diffusion characteristics.
%
To accomplish this, the magnetic field induces the movement of water while the
presence of cell membranes, fibers, or other macromolecules hinder this
movement.  By varying the direction and strength of the magnetic fields, we
essentially use the water molecules as a probe to get a sense of the local
tissue structure.

At the lowest level this diffusion pattern provides several insights.  For
example, in fibrous tissue the dominant direction of allowed diffusion
corresponds the underlying direction of fibers.  In addition, quantifying the
anisotropy of the diffusion pattern can also provide useful biomarkers.
%
Several models have been proposed to interpret scanner measurements, ranging
from geometric abstractions to those with biological motivation.
%
In \autoref{sec:modeling} we will introduce various models and methods for
interpreting the diffusion measurements.


By connecting these local orientation models, tractography attempts to
reconstruct the neural pathways.  Tracing out these pathways, we begin to see
how neurons originating from one region connect to other regions and how
well-defined those connections may be.  Not only can we examine properties of
the local tissue but we begin to see the global functional architecture of the
brain, but for such studies, the quality of the results relies heavily on the
chosen fiber representation and the method of reconstructing pathways.
%
In \autoref{sec:tractography} we will describe several techniques for tracing
out pathways.

\begin{figure}[t]
  \centering
  \includegraphics[width=.8\columnwidth]{case01045_2T_half_fa}%
 \caption{Cutaway showing tractography throughout the left hemisphere colored
   by FA to indicate diffusion strength \cite{Malcolm2009ipmi}.  From this
   view, the fornix and cingulum bundle are visible near the center.}
  \label{fig:rh_fa}
\end{figure}




At the highest level, neuroscientists can use the results of local modeling
and tractography to examine individuals or groups of individuals.
%
In \autoref{sec:applications} we will survey approaches to segment tissue with
boundaries indistinguishable with structural MRI, apply network analysis to
characterize the macroscopic neural architecture, reconstruct fiber bundles
from individual fiber traces, and analyze groups of individuals.




\section{Modeling} \label{sec:modeling}

\subsection{Imaging the tissue}

The overall signal observed in an dMRI image voxel (millimetric) is the
superposition of signals from many underlying molecules probing the tissue
(micrometric).
%
Thus, the image contrast is related to the strength of water diffusion. At
each image voxel, diffusion is measured along a set of distinct gradients,
$\u_1,...,\u_n\in\R^3$, producing the corresponding signal,
$\s=\rv{s_1,...,s_n}^T\in\R^n$. A general weighted formulation that relates
the measured diffusion signal to the underlying fiber architecture may be
written as,
\begin{equation} \label{eq:general_model}
  s_i = s_0 \sum_j w_j e^{ -b_j \u_i^T D_j \u_i },
\end{equation}
where $s_0$ is a baseline signal intensity, $b_j$ is the b-value, an
acquisition-specific constant, $w_j$ are convex weights, and $D_j$ is a tensor
describing a diffusion pattern.
\comment{While, this mixture model is the most popular
generative model, other models that take into account the restricted geometry
of the neural cells have also been used \cite{Barmpoutis2009}.}
%
One of the first acquisition schemes developed, diffusion tensor imaging (DTI)
utilizes these measurements to compute a Gaussian estimate of the diffusion
orientation and strength at each voxel \cite{Basser1994jmr,LeBihan2003}.

Going beyond this macroscopic description of diffusion, various higher
resolution acquisition techniques have been developed to capture more
information about the diffusion pattern.  One of the first techniques,
Diffusion Spectrum Imaging (DSI) \comment{Q-spacing imaging} measures the
diffusion process at various scales by sampling densely throughout the voxel
\cite{Wedeen2005,Hagmann2005phd}.  From this, the Fourier transform is used to
convert the signal to a diffusion probability distribution.  Due to the large
number of samples acquired (usually more than 256 gradient directions), this
scheme provides a much more accurate description of the diffusion process.
However, on account of the large acquisition time (of the order of 1-2 hours
per subject), this technique is not typically used in clinical scans and its
use is restricted to few research applications.

Instead of spatially sampling the diffusion in a lattice throughout the voxel,
a spherical shell sampling could be used \cite{Tuch2002phd}. Using this
sampling technique, the authors in \cite{Tuch2004} demonstrated that the shape
of the diffusion probability could be recovered from the images acquired on a
single spherical shell.  This significantly reduced the acquisition time,
while providing most of the information about the underlying diffusion in the
tissue.  This naturally led to the application of techniques for estimating
functions on a spherical domain.  For example, Q-ball imaging (QBI)
demonstrated a spherical version of the Fourier transform to reconstruct the
probability diffusion as an isosurface \cite{Tuch2002phd}.

To begin studying the microstructure of fibers with these imaging techniques,
we need models to interpret these diffusion measurements.  Such models fall
broadly into two categories:  parametric and nonparametric.


\subsection{Parametric models}

One of the simplest models of diffusion is a Gaussian distribution: an
elliptic (anisotropic) shape indicating a strong diffusion direction while a
more rounded surface (isotropic) indicating less certainty in any particular
direction (see \autoref{fig:1t}).
%
While robust, assuming this Gaussian model is inadequate in cases of mixed
fiber presence or more complex orientations where the signal may indicate a
non-Gaussian pattern \cite{Alexander2002,Frank2002,Tuch2002}.
%
To handle these complex patterns, higher resolution imaging and more flexible
parametric models have been proposed including mixtures of tensors
\cite{Alexander2001,Tuch2002,Behrens2007,Hosey2005,Parker2005,Kreher2005,Peled2006}
and directional functions \cite{McGraw2006,Kaden2007,Rathi2009mia_w}.  While
these typically require the number of components to be fixed or estimated
separately, more continuous mixtures have also been proposed
\cite{Jian2007ni}.
%
Further, biologically inspired models and tailored acquisition schemes have
been proposed to estimate physical tissue microstructure
\cite{Assaf2005,Assaf2008}; see \cite{Alexander2009} for more.


\subsection{Nonparametric models}

Nonparametric models can often provide more information about the diffusion
pattern.  Instead of modeling a discrete number of fibers as in parametric
models, nonparametric techniques estimate a spherical orientation distribution
function indicating potential fiber directions and the relative certainty
thereof.
%
For this estimation, an assortment of surface reconstruction methods have been
introduced: Q-ball imaging to directly transform the signal into a probability
surface \cite{Tuch2004}, spherical harmonic representations
\cite{Frank2002,Anderson2005,Hess2006,Descoteaux2007mrm}, higher-order tensors
\cite{Ozarslan2003,Basser2007,Barmpoutis2009}, diffusion profile transforms
\cite{Jansons2003,Ozarslan2006}, deconvolution with an assumed single-fiber
signal response \cite{Tournier2004,Jian2007dot}, and more.  \autoref{fig:sh}
shows a spherical harmonic reconstruction of the signal.  Compare this to the
original signal in \autoref{fig:signal}.

It is important to keep in mind that there is a often distinction made between
the reconstructed diffusion orientation distribution function and the putative
fiber orientation distribution function; while most techniques estimate the
diffusion function, its relation to the underlying fiber function is still an
open problem.  Spherical convolution is designed to directly transform the
signal into a fiber distribution
\cite{Alexander2005ipmi,Jansons2003,Anderson2005,Tournier2004}, yet diffusion
sharpening strategies have been developed to deal with Q-ball and diffusion
functions \cite{Descoteaux2009tmi}.

While parametric methods directly describe the principal diffusion directions,
interpreting the diffusion pattern from model independent representations
typically involves determining the number and orientation of principal
diffusion directions present.  A common technique is to find them as surface
maxima of the diffusion function
\cite{Hess2006,Tournier2004,Bloy2008,Descoteaux2009tmi}, while another approach
is to decompose a high-order tensor representation of the diffusion function
into a mixture of rank-1 tensors \cite{Schultz2008}.


\subsection{Regularization}

As in all physical systems, the measurement noise plays a nontrivial role, and
so several techniques have been proposed to regularize the estimation.  One
could start by directly regularizing the MRI signal by designing filters based
on the various signal noise models
\cite{Koay2006,Assemlal2008,AjaFernandez2008}.  Alternatively, one could
estimate the diffusion tensor field and then correct these estimated
quantities \cite{Tschumperle2002}.  For spherical harmonic modeling, a
regularization term can be been directly included in the least squares
formulation \cite{Hess2006,Descoteaux2007mrm}.
%
Attempts such as these to manipulate diffusion weighted images or tensor
fields have received considerable attention regarding appropriate algebraic
and numeric treatments
\cite{Tschumperle2002,Batchelor2005,Fletcher2007riem,Kindlmann2007miccai}.

Instead of regularizing signal or model parameters directly, an alternative
approach is to infer the underlying geometry of the vector field
\cite{Savadjiev2006}.
%
Another interesting approach treats each newly acquired diffusion image as a
new system measurement.  Since diffusion tensors and spherical harmonics can
be estimated within a least-squares framework, one can use a Kalman filter to
update the estimate and optionally stop the scan when the model parameters
converge \cite{Poupon2008}.  Further, this online technique can be used to
alter the gradient set so that, were the scan to be stopped early, the
gradients up to that point are optimally spread (active imaging)
\cite{Deriche2009}.


\subsection{Characterizing tissue}

The goal of diffusion imaging is to draw inferences from the diffusion
measurements.
%
As a starting point, one often converts the diffusion weighted image volumes
to a scalar volume much like structural MRI or CT images.
%
Starting with the standard Gaussian diffusion tensor model, an assortment of
scalar measures have been proposed to quantify the size, orientation, and
shape of the diffusion pattern \cite{Basser1996,Westin2002}.  For example,
fractional anisotropy quantifies the deviation from an isotropic tensor, an
appealing quantity because it corresponds to the strength of diffusion while
remaining invariant to orientation.
%
Derivatives of these scalar measures have also been proposed to capture more
information about the local neighborhood
\cite{Kindlmann2007tmi,Savadjiev2009}, and these measures have been extended
to high-order tensors \cite{Ozarslan2005}.
%
Further, a definition of generalized anisotropy has been proposed to directly
characterize anisotropy in terms of variance in the signal, hence avoiding an
assumed model \cite{Tuch2002}.
%
While geometric in nature, studies have shown these to be reasonable proxy
measures for neural myelination \cite{Norris2001,Beaulieu2002,Jones2003}.
Some studies have also examined the sensitivity of such measures against image
acquisition schemes \cite{Whitcher2008,Chung2006}.

Meaningful visualization of diffusion images is difficult because of their
multivariate nature, and much is lost when reducing the spectral signal down
to scalar intensity volumes.
%
Several geometric abstractions have been proposed to convey more information.
%
Since the most common voxel model is still the Gaussian diffusion tensor, most
of the effort has focused on visualizing this basic element.  The most common
glyph is an ellipsoid simultaneously representing the size, shape, and
orientation; however, since tensors have six free parameters, more elaborate
representations have been proposed to visualize these additional dimensions
using color, shading, or subtle variations in shape
\cite{Westin2002,Ennis2005,Vilanova2006,Kindlmann2007miccai}.
%
Apart from tensors, visualization strategies for other models have received
comparatively little attention, the typical approach being to simply to
visualize the diffusion isosurface at each voxel.


\vskip1em
%
A vast literature exists on methods of acquisition, modeling, reconstruction,
and visualization of diffusion images.  For a comprehensive view, we suggest
\cite{Tuch2002phd,Minati2007,Hagmann2005phd,Westin2002,Alexander2005,Descoteaux2008}.


\begin{figure}[t]
  \centering
  \includegraphics[width=.48\columnwidth]{case01045_1T_tc}%
  \hspace{1em}
  \includegraphics[width=.48\columnwidth]{case01045_2T_tc}%
  \caption{Tractography from the center of the corpus callosum (seed region in
    yellow).  The single-tensor model (left) captures only the corona radiata
    and misses the lateral pathways known to exist.  The two-tensor method
    \cite{Malcolm2009ipmi} (right) reveals many of these missing pathways
    (highlighted in blue). (From \cite{Malcolm2009ipmi})}
  \label{fig:filtered}
\end{figure}


\section{Tractography} \label{sec:tractography}

To compliment the wide assortment of techniques for signal modeling and
reconstruction, there is an equally wide range of techniques to infer neural
pathways.
%
At the local level, one may categorize them either as tracing individual
connections between regions or as diffusing out to estimate the probability of
connection between regions.
%
In addition, more global approaches have been developed to consider, not just
the local orientations, but the suitability of entire paths when inferring
connections.

\subsection{Deterministic tractography}

Deterministic tractography involves directly following the diffusion pathways.
Typically, one places several starting points (seeds) in one region of
interest and iteratively traces from one voxel to the next, essentially path
integration in a vector field.  One terminates these fiber bundles when the
local diffusion appears week or upon reaching a target region.
%
\autoref{fig:rh_fa} offers a glimpse from inside the brain using this basic
approach.
%
Often additional regions are used as masks to post-process results, \eg
pathways from region A but not touching region B.

In the single tensor model, standard streamline tractography follows the
principal diffusion direction of the tensor \cite{Mori2002}, while multi-fiber
models often include techniques for determining the number of fibers present
or when pathways branch \cite{Hagmann2004,Kreher2005,Guo2006}.
%
Since individual voxel measurements may be unreliable, several techniques have
been developed for regularization.  For example, using the estimate from the
previous position \cite{Lazar2003,Zhukov2002} as well as filtering
formulations for path regularization \cite{Gossl2002} and model-based
estimation \cite{Malcolm2009ipmi}.
%
The choice of model and optimization mechanism can drastically effect the
final tracts.  To illustrate, \autoref{fig:filtered} shows tractography from
the center of the corpus callosum using a single-tensor model and a two-tensor
model using the filtered technique from \cite{Malcolm2009ipmi}.



\subsection{Probabilistic tractography}

While discrete paths intuitively represent the putative fiber pathways of
interest, they tend to ignore the inherent uncertainty in estimating the
principle diffusion directions in each voxel.
%
Instead of tracing discrete paths to connect voxels, one may instead query the
probability of voxel-to-voxel connections given the diffusion probability
distributions reconstructed in each voxel.

Several approaches have been developed based on sampling.  For example, one
might run streamline tensor tractography treating each as a Monte Carlo
sample; the more particles that take a particular path, the more likely that
particular fiber pathway \cite{Koch2002,Behrens2003,Parker2003,Bjornemo2002}.
Another approach would be to consider more of the continuous diffusion field
from Q-ball or other reconstructions
\cite{Tuch2000,Batchelor2001,Perrin2005,Parker2005,Friman2006,Zhang2009}.  By
making high curvature paths unlikely, path regularization can be naturally
enforced within the probabilistic framework.
%
Another approach is to propagate an arrival-time isosurface from the seed
region out through the diffusion field, the front evolution force being a
function of the local diffusivity
\cite{Batchelor2001,Campbell2005,Tournier2003}.

Using the full diffusion reconstruction to guide particle diffusion has the
advantage of naturally handling uncertainty in diffusion measurements, but for
that same reason it tends toward diffuse tractography and false-positive
connections.  One option is to constrain diffusivity by fitting a model,
thereby ensuring definite diffusion directions yet still taking into account
some uncertainty \cite{Parker2005,Friman2006,Behrens2003}.  A direct extension
is to introduce a model selection mechanism to allow for additional components
where appropriate \cite{Behrens2007,Freidlin2007}.
%
However, one could stay with the nonparametric representations and instead
sharpen the diffusion profile to draw out the underlying fiber orientations
\cite{Tournier2007,Descoteaux2009tmi}.




\subsection{Global tractography}

Despite advances in voxel modeling, discerning the underlying fiber
configuration has proven difficult.  For example, looking at a single voxel,
the symmetry inherent in the diffusion measurements makes it difficult to tell
if the observed pattern represents a fiber curving through the voxel or
instead represents a fanning pattern.
%
Reliable and accurate fiber resolution requires more information than that of
a single voxel.
%
For example, instead of estimating the fiber orientation, one could instead
infer the geometry of the entire neighborhood \cite{Savadjiev2008}.

Going a step further, one could say that reliable and accurate connectivity
resolution requires even more information, beyond simply a voxel neighborhood.
%
In some respects, probabilistic tractography can be seen to take into account
more global information.  By spawning thousands of particles, each attempting
to form an individual connection, probabilistic techniques are able to explore
more possibilities before picking those that are likely \cite{Parker2003}.
However, if these particles still only look at the local signal as they
propagate from one voxel to the next, then they remain susceptible to local
regions of uncertainty.  Even those with resampling schemes are susceptible
since the final result is still a product of the method used in local tracing
\cite{Bjornemo2002,Zhang2009}.

A natural step to address such problems is to introduce global connectivity
information into local optimization procedures of techniques mentioned above.
%
The work of \cite{Jbabdi2007} does this by extending the local Bayesian
formulation in \cite{Behrens2007} with an additional prior that draws upon
global connectivity information in regions of uncertainty.
%
Similarly, one could use an energetic formulation still with data likelihood
and prior terms, but additionally introduce terms governing the number of
components present \cite{Fillard2009}.

Another approach is to treat the entire path as the parameter to be optimized
and use global optimization schemes.
%
For example, one could model pathways as piecewise linear with a data
likelihood term based on signal fit and a prior on spatial coherence of those
linear components \cite{Kreher2008,Reisert2009}.  One advantage of this
path-based approach is that it somewhat obviates the need for a multi-fiber
voxel model; however, such a flexible global model dramatically increases the
computational burden.

An alternative formulation is to find geodesic paths through the volume.
Again using some form of data likelihood term, such methods then employ
techniques for front propagation to find globally optimal paths of connection
\cite{ODonnell2002,Pichon2005,Prados2006,Lenglet2004eccv,Fletcher2007ipmi}.

Tractography is often used in group studies which typically require a common
atlas for inter-subject comparison.  Beginning with the end in mind, one could
determine a reference bundle as a template and use this to drive tractography.
This naturally ensures both the general geometric form of the solution and a
direct correspondence between subjects \cite{Eckstein2009,Goodlett2009}.
%
Alternatively, the tract seeding and other algorithm parameters could be
optimized until the tracts (data driven) approach the reference (data prior)
\cite{Clayden2007}.  Since this requires pre-specifying such a reference
bundle, information that may be unavailable or difficult to obtain, one could
even incorporate the formulation of the reference bundle into the optimization
procedure itself \cite{Clayden2009}.




\subsection{Validation}

In attempting to reconstruct neural pathways virtually, it is important to
keep in mind the inherent uncertainty in such reconstructions.  The resolution
of dMRI scanners is at the level of 1-3mm while physical fiber axons are
often an order of magnitude smaller in diameter--a relationship which leaves
much room for error.
%
Some noise or a complex fiber configuration could simply look like a diffuse
signal and cause probabilistic tractography to stop in its tracks, while a few
inaccurate voxel estimations could easily send the deterministic tracing off
course to produce a false-positive connection.  Even global methods could
produce a tract that fits the signal quite well but incidentally jumps over an
actual boundary in one or two voxels it thinks are noise.  Consequently, a
common question is: Are these pathways really present?

With this in mind, an active area of study is validating such results.  Since
physical dissection often requires weeks of tedious effort, many techniques
have been used for validating these virtual dissections.
%
A common starting point is to employ synthetic and physical phantoms with
known parameters when evaluating new methods \cite{Poupon2008phantom}.
%
When possible, imaging before and after injecting radio-opaque dyes directly
into the tissue can provide some of the best evidence for comparison
\cite{Lin2003,Dauguet2007}.
%
Another powerful approach is to apply bootstrap sampling or other
non-parametric statistical tests to judge the sensitivity and reproducibility
of resulting tractography against algorithm parameters, image acquisition, and
even signal noise
\cite{Lazar2005,Jones2005,Gigandet2008,Whitcher2008,Chung2006,Clayden2007}.


\section{Applications} \label{sec:applications}

Having outlined various models and methods of reconstructing pathways, we now
briefly cover several methods of further analysis.

\subsection{Volume segmentation}

Medical image segmentation has a long history, much of it focused on scalar
intensity-based segmentation of anatomy.  For neural segmentation, structural
MRI easily reveals the boundaries between gray-matter and white-matter, and
anatomic priors have helped further segment some internal structures
\cite{Pohl2007ipmi}; however, the boundaries between many structures in the
brain are remain invisible with structural MRI alone.  The introduction of
dMRI has provided new discriminating evidence in such cases where tissue may
appear homogeneous on structural MRI or CT but contain distinct fiber
populations.

To begin, most work has focused segmentation of the estimated tensor fields.
Using suitable metrics to compare tensors, these techniques often borrow
directly from active contour or graph cut segmentation with the approach of
separating distributions.
%
For example, one could define a Gaussian distribution of tensors to
approximate a structure of interest \cite{Rousson2004,deLuis-Garcia2007}.
%
For tissues with more heterogeneous fiber populations, \eg the corpus callosum
as it bends, such global parametric representations are unsuitable.  For this,
nonparametric approaches are more appropriate at capturing the variation
throughout such structures \cite{Rathi2007,Malcolm2007tc}.
%
Another approach to capture such variation is to limit the parametric
distributions to local regions of support, essentially robust edge detection
\cite{Lankton2008mmbia}.

In \autoref{fig:corpus} we see a graph cut segmentation of the corpus callosum
\cite{Malcolm2007tc}.  The color-coded fractional anisotropy image is shown
for visualization while segmentation was performed on the underlying tensor
data.  If statistics are computing ignoring the tensor manifold (Euclidean
assumption), the final segmentation fails \autoref{fig:euclidean}.  If
statistics are computing via a Riemannian mapping that respects this
structure, the final segmentation is accurate \autoref{fig:riemannian}.  This
highlights the need for appropriate algebraic treatment of tensors and other
non-Euclidean models.
\begin{figure}[t]
  \centering
  \subfloat[Initial seed regions]{%
    \includegraphics[width=.33\columnwidth]{CORPUS_initial}}%
  \subfloat[Euclidean mapping]{ \label{fig:euclidean} %
    \includegraphics[width=.33\columnwidth]{CORPUS_euclidean}}%
  \subfloat[Riemannian mapping]{ \label{fig:riemannian}%
    \includegraphics[width=.33\columnwidth]{CORPUS_final}}%
  \caption{Segmenting the corpus callosum using the graph cut technique from
    \cite{Malcolm2007tc} (side view).  The Euclidean mapping
    \subref{fig:euclidean} does not take into account the structure of the
    underlying tensor manifold.  The Riemannian mapping
    \subref{fig:riemannian} takes this structure into account when computing
    statistics and so produces a correct segmentation. (From
    \cite{Malcolm2007tc})}
  \label{fig:corpus}
\end{figure}

An altogether different approach to segmenting a structure is to divide it up
according to where portions connect elsewhere.  For example, the thalamus
contains several nuclei indistinguishable in standard MR or even with
contrast.  After tracing connections from the thalamus to the cortex, one
study demonstrated that grouping these connections revealed the underlying
nuclei \cite{Behrens2003nn}.




\subsection{Fiber clustering}

The raw output of full-brain tractography can produce hundreds of thousands of
such tracings, an overwhelming amount of information.  One approach to
understanding and visualizing such results is to group individual tracings
into fiber bundles.  Such techniques are typically based around two important
design choices: the method of comparing fibers, and the method of clustering
those fibers.

In comparing two fibers, one often starts by defining a distance measure,
these typically being based on some point-to-point correspondence between the
fibers \cite{Ding2003,Corouge2006,ODonnell2007tmi}.  With this correspondence
in hand, one of the most common distances is then to take the mean closest
point distance between the two fibers (Hausdorff distance).
%
An alternative is to transform each fiber to a new vector space with a natural
norm, \eg a fiber of any length can be encoded with only the mean and
covariance of points along its path and then use the L$_2$ distance
\cite{Brun2004}.
%
An altogether different approach is to consider the spatial overlap between
fibers \cite{Wang2009,Wasserman2009}.  Since full-brain tractography often
contains many small broken fragments as it tries to trace out bundles, such
fragments are often separated from their actual cluster.  Measures of spatial
overlap may be more robust in such cases.
%
In each of these methods, fibers were only considered as sequences of points,
\ie connections and orientations were ignored.  Recent work demonstrates that
incorporating such considerations provides robust descriptors of fiber bundles
\cite{Durrelman2009}.

\begin{figure}[t]
  \centering
  \includegraphics[width=.48\columnwidth]{case01045_1t_rh_ext}%
  \includegraphics[width=.48\columnwidth]{case01045_1t_rh_int}%
  \caption{Full-brain streamline tractography clustered using affinity
    propagation \cite{Malcolm2009cukf_ext}.  Viewed from the outside (left)
    and inside cutting away the left hemisphere (right).  Among the visible
    structures, we see the cingulum bundle (yellow), internal capsule (red),
    and arcuate (purple).}
  \label{fig:rh}
\end{figure}
Based on these distances, several methods have been developed to cluster the
fibers.
%
Spectral methods typically begin with the construction of a Gram matrix
encoding the pairwise affinity between any two fibers.  After which normalized
cuts can be applied to partition the Gram matrix and hence the fibers
\cite{Brun2004}.
%
Affinity propagation has recently been demonstrated as an efficient and robust
alternative which automatically determines the number of clusters to support a
specified cluster size preference \cite{Leemans2009,Malcolm2009cukf_ext}.
%
In \autoref{fig:rh} shows how clustering can automatically reveal known
structures and provide a more coherent view of the brain.  In addition,
clustering can be used to judge outliers.  For example, \autoref{fig:fo}
reveals several streamlines that appear to have gone off track relative to the
cluster centers.
\begin{figure}[t]
  \centering
  \subfloat[Original fibers]{ \label{fig:fo_fl}%
    \begin{minipage}[b]{.48\linewidth}
      \includegraphics[width=\textwidth]{case01045_1t_fo_fl_side} \\
      \includegraphics[width=\textwidth]{case01045_1t_fo_fl_top}
    \end{minipage}}
  \subfloat[Major pathways]{ \label{fig:fo_ex}%
    \begin{minipage}[b]{.48\linewidth}
      \includegraphics[width=\textwidth]{case01045_1t_fo_ex_side} \\
      \includegraphics[width=\textwidth]{case01045_1t_fo_ex_top}
    \end{minipage}}
  \caption{Fronto-occipital fibers from the right hemisphere using streamline
    tractography and clustered into bundles \subref{fig:fo_fl}
    \cite{Malcolm2009cukf_ext}.  Viewing the most representative fiber in each
    bundle \subref{fig:fo_ex} we see a fiber from one cluster (red) that
    appears to have wandered off the pathway.  (From \cite{Malcolm2009cukf_ext})}
  \label{fig:fo}
\end{figure}

Another clustering approach is to use the inner product space itself.  For
example, one can efficiently group directly on the induced manifold by
iteratively joining fibers most similar until the desired clustering emerges
\cite{Wasserman2009}.
%
To avoid construction of the large Gram matrix, variants of expectation
maximization have been demonstrated to iteratively cluster fibers, an approach
that naturally lends itself to incorporating anatomic priors
\cite{Wang2009,ODonnell2007tmi,Maddah2008media}.
%
Alternatively, one can begin with the end in mind by registering a reference
fiber bundle template to patients thus obviating any need for later spatial
normalization or correspondence \cite{Clayden2009}.



\subsection{Connectivity}

While tissue segmentation can provide global cues of neural organization, it
tells little of the contribution of individual elements.
%
Similarly, while clustered tracings are easily visualized, deciphering the
flood of information from full-brain tractography demands more comprehensive
quantitative analysis.
%
For this, much has been borrowed from network analysis to characterize the
neural topology.
%
To start, instead of segmenting fibers into bundles, one can begin by
classifying voxels into hubs or subregions into subnetworks
\cite{Sporns2007,Gong2009,Hagmann2007,Hagmann2008}.

Dividing the brain up into major functional hubs, one can then view it as a
graphical network as in \autoref{fig:connect}.  Each of these edges is then
often weighted as a function of connection strength \cite{Hagmann2007}, but
may also incorporate functional correlation to give further evidence of
connectivity.
\begin{figure}[t]
  \centering
  \includegraphics[width=.7\columnwidth]{case01045_connections_320}
  \caption{The brain viewed as a network of weighted connections.  Each edge
    represents a possible connection and is weighted by the strength of that
    path.  Many techniques from network analysis are applied to reveal hubs
    and subnetworks within this macroscopic view.}
  \label{fig:connect}
\end{figure}

One of the first results of such analysis was the discovery of dense hubs
linked by short pathways, a characteristic observed in many complex physical
systems (small-world phenomena).
%
Another interesting finding came from combining anatomic connections from dMRI
with neuronal activity provided by fMRI \cite{Honey2009}.  They found that
areas which are functionally connected are often not structurally connected,
hence tractography alone does not provide the entire picture.

For a recent review of this emerging field of structural and functional
network analysis, we recommend \cite{Bullmore2009}.






\subsection{Tissue analysis}

\begin{figure}[t]
  \centering
  \includegraphics[width=.8\columnwidth]{case01026_03_ds}
  \caption{Plotting FA as a function of arc-length to examine local
    fluctuations.  Fibers are selected that connect the left and right seed
    regions (green).  Notice how the FA from single-tensor (blue) is lower in
    regions of crossing compared to two-tensor FA (red).  (From \cite{Malcolm2009study})}
  \label{fig:fa_ds}
\end{figure}
In forming population studies, there are several approaches for framing the
analysis among patients.
%
For example, voxel-based studies examine tissue characteristics in regions of
interest \cite{Ashburner2000}.  Discriminant analysis has been applied to
determine such regions \cite{Caan2006}.  Alternatively, one could also perform
regression on the full image volume taking into account not only variation in
diffusion but also in the full anatomy \cite{Rohlfing2009}.
%
In contrast, tract-based studies incorporate the results of tractography to
use fiber pathways as the frame of reference \cite{Ding2003,Smith2006}, and
several studies have demonstrated the importance of taking into account local
fluctuations in estimated diffusion
\cite{Corouge2006,ODonnell2009,Maddah2008media,Goodlett2009,Yushkevich2008}.

A common approach in many of these studies is to focus on characterizing
individual pathways or bundles.  To illustrate this analysis, \autoref{fig:fa_ds}
shows fibers connecting a small region in each hemisphere.  We then average FA
plotted along the bundle as a function of arc-length.  Further, we plot the FA
from both single- and two-tensor models to show how different models often
produce very different tissue properties \cite{Malcolm2009study}.

Several reviews exist documenting the application and findings of using
various methods \cite{Lim2002,Horsfield2002,Kubicki2007}.


\section{Summary}

Diffusion MRI has provided an unprecedented view of neural architecture.  With
each year, we develop better image acquisition schemes, more appropriate
diffusion models, more accurate pathway reconstruction, and more sensitive
analysis.

In this survey, we began with an overview of the various imaging techniques
and diffusion models.  While many acquisition sequences have become widely
distributed for high angular resolution imaging, work continues in developing
sequences and models capable of accurate resolution of biological properties
such as axon diameter and degree of myelination
\cite{Assaf2008,Alexander2009}.  We then reviewed various parametric models
starting with the diffusion tensor on up to various mixture models as well as
high-order tensors.  Work continues to develop more accurate and reliable
model estimation by incorporating information from neighboring voxels
\cite{Savadjiev2008,Malcolm2009ipmi}.  Further, scalar measures derived from
these models similarly benefit from incorporating neighborhood information
\cite{Savadjiev2009}.

Next we outlined various methods of tractography to infer connectivity.
Broadly, these techniques took either a deterministic or probabilistic
approach.  We also documented the recent trend toward global approaches, those
that combine local voxel-to-voxel tracing with a sense of the full path
\cite{Fillard2009,Reisert2009}.  Even with such considerations, tractography
is has proven quite sensitive to image acquisition and initial conditions, so
much work has gone into validation.  Common techniques are the use of physical
phantoms \cite{Poupon2008phantom} or statistical tests like bootstrap analysis
\cite{Lazar2005,Jones2005,Clayden2007}.

Finally, we briefly introduced several machine learning approaches to make
sense of the information found in diffusion imagery.  Starting with
segmentation, several techniques for scalar intensity segmentation have been
extended to dMRI.  With the advent of full-brain tractography providing
hundreds of thousands of fiber paths, the need to cluster connections into
bundles has become increasingly important.  The application of network
analysis to connectivity appears to be an emerging area of research,
especially in combination with alternate imaging modalities
\cite{Bullmore2009}.  Finally, we noted several approaches to the analysis of
neural tissue itself in regions of interest or along pathways.




\chapter{Filtered Tractography}

\section{Summary}

Nearly all approaches to tractography fit the model at each voxel independent
of other voxels; however, tractography is a causal process: we arrive at each
new position along the fiber based upon the diffusion found at the previous
position.

We propose to we treat model estimation and tractography as such by placing
this process within a causal filter.
%
As we examine the signal at each new position, the filter recursively updates
the underlying local model parameters, provides the variance of that estimate,
and indicates the direction in which to propagate tractography.
\autoref{fig:overview} provides an overview of this recursive process.
\begin{figure}[t]
  \centering
  \resizebox{!}{1.8in}{\input{/Users/malcolm/junk/writeups/fibers/figs/ukf/ukf.pstex_t}}
  \caption{System overview illustrating relation between the neural fibers,
    the measured scanner signal, and the unscented Kalman filter as it is used
    to estimate the local model.  At each step, the filter uses its current
    model state ($\hat{\x}_t$) to predict the observed scanner signal
    ($\bar{\y}_{t+1|t}$) and then compares that against the actual measured
    signal ($\y_t$) in order to update its internal model state
    ($\hat{\x}_{t+1}$).}
  \label{fig:overview}
\end{figure}

To begin estimating within a finite dimensional filter, we model the diffusion
signal using a mixture of parametric directional functions.  We choose
parametric models since they provide a compact representation of the signal
parameterized by the principal diffusion direction and scaling parameters
influencing anisotropy, and further allows analytic reconstruction of the
oriented diffusion function from those parameters
\cite{malcolm2010tmi,Rathi2009mia_w}.  This enables estimation directly from
the raw signal without separate preprocessing or regularization.
%
Because the signal reconstruction is nonlinear, we use the unscented Kalman
filter to estimate the model parameters and then propagate in the most
consistent direction.
%
Using causal estimation in this way yields inherent path regularization and
accurate fiber resolution at crossing angles not found with independent
optimization.  In a loop, the filter estimates the model at the current
position, moves a step in the most consistent direction, and then begins
estimation again.
%
Since each iteration begins with a near-optimal solution provided by the
previous estimation, the convergence of model fitting is improved and many
local minima are naturally avoided.
%
This approach generalizes to arbitrary fiber model with finite dimensional
parameter space.  The bulk of this dissertation is found among
\cite{malcolm2010tmi,Malcolm2009ipmi,malcolm2009cukf}.

Several studies have specific relevance to this present work because of their
use of a filtering strategy in either orientation estimation or tractography.
%
In extending standard streamline tractography to enhance path regularization,
\cite{Gossl2002} move curve integration into a linear Kalman filter while
\cite{Zhukov2002} incorporate a moving least squares estimator.
%
Alternatively, one could use a particle filter to place a prior on the
direction of propagation \cite{Zhang2009}.
%
Since these methods model only the position of the fiber, not the local fiber
model, they are inherently focused on path regularization rather than
estimating the underlying fiber structure.
%
Finally, \cite{Poupon2008} proposed using a linear Kalman filter for online,
direct estimation of either single-tensor or harmonic coefficients while
successive diffusion image slices are acquired, while \cite{Deriche2009}
revisited the technique to account for proper regularization and proposed a
method to quickly determine optimal gradient set orderings.

\autoref{sec:model} provides the necessary background on modeling the
measurement signal using directional functions and defines the specific fiber
models employed in this study.  Then, \autoref{sec:estimation} describes how
this model may be estimated using an unscented Kalman filter.



\section{Modeling local fiber orientations} \label{sec:model}

\begin{figure}[t]
  \centering
  \subfloat[Single-tensor streamline]{
    \includegraphics[width=.48\columnwidth]{case01045_1T_tc_sweet}}%
  \subfloat[Two-component model]{
    \includegraphics[width=.48\columnwidth]{case01045_2W_tc_sweet}}%
  \caption{Comparison of tractography using a single-tensor model and the
    proposed two-component model with filtering.  While the single-tensor
    model misses many of the lateral branches from the corpus callosum, the
    filter provides a stable estimate of the two-component model capable of
    revealing the lateral transcallosal pathways.  Seed region indicated with
    yellow.}
  \label{fig:sweetness}
\end{figure}
In diffusion weighted imaging, image contrast is related to the strength of
water diffusion, and our goal is to accurately relate these signals to an
underlying model of fiber orientation.  At each image voxel, diffusion is
measured along a set of distinct gradients, $\u_1,...,\u_n\in\S^2$ (on the
unit sphere), producing the corresponding signal,
$\s=\rv{s_1,...,s_n}^T\in\R^n$.  For voxels containing a mixed diffusion
pattern, a general weighted formulation may be written as,
\begin{equation}
  s_i = s_0 \sum_j w_j e^{ -b \u_i^T D_j \u_i },
\end{equation}
where $s_0$ is a baseline signal intensity, $b$ is an acquisition-specific
constant, $w_j$ are convex weights, and $D_j$ is a tensor matrix describing a
diffusion pattern.

We now provide definition to the two primary models employed in this work:
equally weighted diffusion tensors \cite{Malcolm2009ipmi,malcolm2010tmi} and
Watson directional functions\cite{Rathi2009mia_w}.  Later in
\autoref{ch:weighted}, we return to extend the method to weighted mixtures of
diffusion tensors which requires a different filtering scheme
\cite{malcolm2009cukf}.



\subsection{Diffusion tensors} \label{sec:equal}

From the general mixture model above (\autoref{eq:general_model}), we choose
to start with two components.  This choice is guided by several previous
studies which found two-fiber models to be sufficient at \bone
\cite{Tuch2002,Kreher2005,Guo2006,Zhan2006,Peled2006,Behrens2007}.
%
Also, we assume the shape of each tensor to be ellipsoidal, \ie there is one
dominant principal diffusion direction \m with eigenvalue \lx and the
remaining orthonormal directions have equal eigenvalues $\ly=\lambda_3$ (as in
\cite{Parker2005,Friman2006,Peled2006,Kaden2007}).
%
Last, as in the study of \cite{Zhan2006}, we fix the weights so that each
component contributes equally.  While assuming equally-weighted compartments
may limit flexibility, we found that the eigenvalues adjust to fit the signal
in much the same way a fully weighted model would adjust.  We will revisit
this in the experiments and discussion (\autoref{sec:weights}).

These assumptions then leave us with the following model used in this study:
\begin{equation} \label{eq:2T_model}
  s_i = \tfrac{s_0}{2} e^{ -b \u_i^T D_1 \u_i } + \tfrac{s_0}{2} e^{ -b \u_i^T D_2 \u_i } ,
\end{equation}
where $D_1,D_2$ are each expressible as,
%
$ D = \lx \m \m^T + \ly\left(\v{p}\v{p}^T + \v{q}\v{q}^T\right), $
%
with $\m,\v{p},\v{q} \in \S^2$ forming an orthonormal basis aligned to the
principal diffusion direction \m.  The free model parameters are then $\m_1$,
$\lx_1$, $\ly_1$, $\m_2$, $\lx_2$, and $\ly_2$.  In our current
implementation, we restrict each $\lambda$ to be positive.  Extending off the
two-tensor model, we can directly formulate a three-tensor version:
\begin{equation}   \label{eq:3T_model}
  s_i = \frac{s_0}{3} \sum_{j=1}^3 e^{-b \u_i^T D_j \u_i } ,
\end{equation}
with the additional parameters $\m_3$, $\lx_3$, and $\ly_3$.





\subsection{Watson directional function} \label{sec:watson}

Considering a single tensor, we now follow the formulation of Rathi
\etal\cite{Rathi2009mia_w} to define the Watson directional function which
approximates the apparent diffusion pattern.  We begin by noting that any
diffusion tensor $D$ can be decomposed as $D = U \Lambda U^T$, where $U$ is a
rotation matrix and $\Lambda$ is a diagonal matrix with eigenvalues
$\{\lambda_1, \lambda_2, \lambda_3\}$.  These eigenvalues determine the shape
of the tensor:  ellipsoidal, planar, and spherical.  For example, if
$\lambda_1 > \lambda_2 > \lambda_3$, then the shape is ellipsoidal with the
major axis of the ellipsoid pointing to the eigenvector corresponding to
$\lambda_1$.  Intuitively, it represents strong diffusion along that
particular direction.  When $\lambda_1 = \lambda_2 > \lambda_3$, the shape is
planar indicating diffusion along orthogonal directions, and finally, when
$\lambda_1 = \lambda_2 = \lambda_3$, the diffusion is spherical (isotropic).

The most common of these configurations is ellipsoidal with principal
diffusion direction $\m$ and eigenvalue $\lambda_1$, and hence the first step
in introducing directional functions is to approximate the tensor by its first
eigenvector expansion: $D \approx \lambda_1 \m \m^T$.  Using this, each
exponent in \autoref{eq:general_model} may then be rewritten,
\begin{eqnarray} \label{eq:approx}
  -b \u_i^T D \u_i
  &\approx& -b\lambda_1\u_i^T\left(\m\m^T\right)\u_i \\
  &=& -b \lambda_1 \left(\u_i^T \m\right)^2 \\
  &=& -k \left(\u_i^T \m\right)^2 ,
\end{eqnarray}
where the scalar $k$ concentration parameter determines the degree of
anisotropy.  Finally, the general model may be restated:
\begin{equation} \label{eq:watson_model}
  s_i = A \sum_j w_j e^{-k_j (\u_i^T \m_j)^2} ,
\end{equation}
where $A$ is a normalization constant such that $\norm{\s}=1$.  For purposes
of comparison, this normalization will also be done to signals obtained from
scanner.  Note that, while the diffusion tensor requires six parameters, these
Watson functions require four parameters: three for the orientation vector
$\m$ and one concentration parameter $k$.  Employing a spherical
representation can further reduce the unit vector \m to two spherical
coordinates.  \autoref{fig:watson_k} demonstrates how adjusting the $k$-value
produces different diffusion patterns, and \autoref{fig:watson_multi}
illustrates two multi-fiber configurations.
\begin{figure}[t]
  \centering
  \subfloat[Reconstructed signals for strong diffusivity ($k=2$), weak
  diffusivity ($k=0.5$), and isotropic diffusion patterns ($k=0.01$).]{
    \label{fig:watson_k}
    \includegraphics[width=.19\columnwidth]{watson_strong}%
    \includegraphics[width=.19\columnwidth]{watson_weak}%
    \includegraphics[width=.19\columnwidth]{watson_iso}}%
  \hspace{1em}%
  \subfloat[Signals for two-fiber and three-fiber mixtures.]{
    \label{fig:watson_multi}
    \includegraphics[width=.19\columnwidth]{watson_two}%
    \includegraphics[width=.19\columnwidth]{watson_three}}
  \caption{Watson directional functions are capable of representing various
    diffusion patterns and fiber orientations.}
  \label{fig:watson}
\end{figure}

From this general mixture model, we choose to start with a restricted form
involving two equally-weighted Watson functions.  This choice is guided by
several previous studies.
%
Behrens \etal\cite{Behrens2007} showed that at a $b$-value of 1000 ms/mm$^2$
the maximum number of detectable fibers is two.  Several other studies have
also found two-fiber models to be sufficient
\cite{Tuch2002,Kreher2005,Zhan2006,Peled2006}.  Using this as a practical
guideline, we started with a mixture of two Watson functions as our local
fiber model.
%
\comment{Also, we assume the shape of each tensor to be ellipsoidal, \ie there
  is one dominant principal diffusion direction \m with eigenvalue \lx and the
  remaining orthonormal directions have equal eigenvalues $\ly=\lambda_3$
  \cite{Friman2006}.}
%
Further, following the study of \cite{Zhan2006}, we assume an equal
combination (50\%-50\%) of the two Watson functions.  While the effect of this
second choice appears to have little to no effect on experiments, we have yet
to quantify any potential loss in accuracy.  These assumptions leave us with
the following two-fiber model used in this study:
\begin{equation} \label{eq:2W_model}
  s_i = \frac{A}{2} \left(e^{ -k_1 (\u_i^T \m_1)^2 } + e^{ -k_2 (\u_i^T \m_2)^2 } \right) .
\end{equation}
where $k_1$ and $\m_1$ parameterize the first Watson function, $k_2$ and
$\m_2$ parameterize the second, and $A$ is again a normalization constant such
that $\norm{\s}=1$.  Thus, the equally-weighted two-fiber model is fully
described by the following parameters: $k_1$, $\m_1$, $k_2$, $\m_2$.
%
Extending off the two-Watson model, we can directly formulate the
equally-weighted three-Watson model:
\begin{equation}   \label{eq:3W_model}
  s_i = \frac{A}{3} \sum_{j=1}^3 e^{-k_j (\u_i^T \m_j)^2} ,
\end{equation}
with the additional parameters $k_3$ and $\m_3$.

Finally, from such parameters, Rathi \etal\cite{Rathi2009mia_w} describe how
one may compute the ODF analytically by applying the Funk-Radon transform
directly to \autoref{eq:watson_model}.  The ODF can be reconstructed directly
from the same parameters describing the signal without a separate estimation
process.  For the two-Watson model (\autoref{eq:2W_model}) the ODF is
approximated by,
\begin{equation} \label{eq:2W_model_odf}
  f_i = \frac{B}{2} \left(e^{ -\tfrac{k_1}{2} (1-(\u_i^T \m_1)^2) }
                        + e^{ -\tfrac{k_2}{2} (1-(\u_i^T \m_2)^2) } \right) ,
\end{equation}
\comment{and for the three-Watson model (\autoref{eq:3W_model}) this becomes,
\begin{equation}   \label{eq:3W_model_odf}
  f_i = \frac{B}{3} \sum_{j=1}^3 e^{-\tfrac{k_j}{2} (1-(\u_i^T \m_j)^2) } ,
\end{equation}}
where $B$ is a normalization factor such that $\sum_i f_i = 1$.



\section{Estimating the fiber model} \label{sec:estimation}

Given the measured scanner signal at a particular voxel, we want to estimate
the underlying model parameters that explain this signal.  As in streamline
tractography, we treat the fiber as the trajectory of a particle which we
trace out.  At each step, we examine the measured signal at that position,
estimate the underlying model parameters, and propagate forward in the most
consistent direction.  \autoref{fig:overview} illustrates this filtering
process.

To use a state-space filter for estimating the model parameters, we need the
application-specific definition of four filter components:
\begin{enumerate*}
\item The system state \x: the model parameters
\item The state transition $f[\cdot]$: how the model changes as we trace the fiber
\item The observation $h[\cdot]$: how the signal appears given a particular state
\item The measurement \y: the actual signal obtained from the scanner
\end{enumerate*}
For our state, we directly use the model parameters, thus the two-fiber model
in \autoref{eq:2W_model} has the following state vector:
\begin{equation} \label{eq:state}
  \x = \rv{\m_1 \;\; k_1 \;\; \m_2 \;\; k_2 }^T,
  \quad
  \m \in \S^2, k \in \R .
\end{equation}
While each $\m$ could be represented in a reduced spherical form, the
antipodes of the spherical parameterization would then introduce
nonlinearities which complicate estimation.
%
For the state transition we assume identity dynamics; the local fiber
configuration does not undergo drastic change from one position to the next.
Our observation is the signal reconstruction, $\y=\s=\rv{s_1,...,s_n}^T$ using
$s_i$ from \autoref{eq:2W_model}, and our measurement is the actual signal
interpolated directly from the diffusion weighted images at the current
position.

Since the signal reconstruction is a nonlinear process, we employ an unscented
Kalman filter to perform nonlinear estimation.  Similar to classical linear
Kalman filtering, the unscented version seeks to reconcile the predicted state
of the system with the measured state and addresses the fact that those two
processes (prediction and measurement) may be nonlinear or unknown.  It does
this in two phases:  first it uses the system transition model to predict the
next state and observation, and then it uses the new measurement to correct
that state estimate.  In what follows, we present the algorithmic application
of the filter.  For more thorough treatments, see \cite{Julier2004,Merwe2003}.

It is important to note two alternative techniques for nonlinear estimation.
%
First, particle filters are commonly used to provide a multi-modal estimate of
unknown systems.  With respect to an $n$-dimensional state space, particle
filters require the number of particles to be exponential to properly explore
the state space.  In contrast, the unscented filter requires only $2n+1$
particles (sigma points) for a Gaussian estimate that space.  Further, for
many slowly varying systems, the multi-modal estimate is unnecessary:  from
one voxel to the next, fibers tend not to change direction drastically.
%
Second, an extended Kalman filter may also be used to provide a Gaussian
estimate after linearizing the system; however, the unscented Kalman filter
provides a more accurate estimate with equivalent computational cost and
altogether avoids the attempt at linearization
\cite{Julier2004,Merwe2003,Lefebvre2004}.


Suppose the system of interest is at time $t$ and we have a Gaussian estimate
of its current state with mean, $\x_t \in \R^n$, and covariance, $P_t \in
\R^{n \times n}$.  Prediction begins with the formation of a set
$\X_t=\{\x_i\} \subset \R^n$ of $2n+1$ \textit{sigma point} states with
associated convex weights, $w_i \in \R$, each a perturbed version of the
current state.  We use the covariance, $P_t$, to distribute this set:
\begin{equation*}
  \x_0 = \x_t
  \qquad
  w_0 = \kappa/(n+\kappa)
  \qquad
  w_i = w_{i+n} = \tfrac{1}{2(n+\kappa)}
\end{equation*}
\begin{equation}   \label{eq:sigma_points}
  \x_{i}   = \x_t + \left[\sqrt{(n+\kappa)P_t}\right]_i
  \quad
  \x_{i+n} = \x_t - \left[\sqrt{(n+\kappa)P_t}\right]_i
\end{equation}
where $[A]_i$ denotes the $i^\text{th}$ column of matrix $A$ and $\kappa$ is
an adjustable scaling parameter ($\kappa = 0.01$ in all experiments).  Next,
this set is propagated through the state transition function, $\hat{\x}=f[\x]
\in \R^n$, to obtain a new predicted sigma point set:
$\X_{t+1|t}=\{f[\x_i]\}=\{\hat{\x}_i\}$.  Since in this study we assume the
fiber configuration does not change drastically as we follow it from one voxel
to the next, we may write this identity transition as, $\x_{t+1|t} = f[\x_t] =
\x_t $.  These are then used to calculate the predicted system mean state and
covariance,
\begin{equation*}
  \bar{\x}_{t+1|t} = \sum_i w_i ~ \hat{\x}_i ,
\end{equation*}
\begin{equation} \label{eq:Pxx}
  P_{xx} = \sum_i w_i \left(\hat{\x}_i - \bar{\x}_{t+1|t}\right)
                     \left(\hat{\x}_i - \bar{\x}_{t+1|t}\right)^T
           + Q ,
\end{equation}
where $Q$ is the injected process noise bias used to ensure a non-null spread
of sigma points and a positive-definite covariance.  This procedure comprises
the \textit{unscented transform} used to estimate the behavior of a nonlinear
function: spread sigma points based on your current uncertainty, propagate
those using your transform function, and measure their spread.

To obtain the predicted observation, we again apply the unscented transform
this time using the predicted states, $\X_{t+1|t}$, to estimate what we expect
to observe from the hypothetical measurement of each state:
$\y=h[\hat{\x}] \in \R^m$.  Keep in mind that, for this study, our
observation is the signal reconstruction from \autoref{eq:2W_model}, and the
measurement itself is the diffusion-weighted signal, $\s$, interpolated at the
current position.  From these, we obtain the predicted set of observations,
$\Y_{t+1|t} = \{ h[\hat{\x}_i] \} = \{ \y_i \}$, and may calculate its
weighted mean and covariance,
\begin{equation*}
  \bar{\y}_{t+1|t} = \sum_i w_i ~ \hat{\y}_i ,
\end{equation*}
\begin{equation} \label{eq:Pyy}
  P_{yy} = \sum_i w_i \left(\hat{\y}_i - \bar{\y}_{t+1|t}\right)
                     \left(\hat{\y}_i - \bar{\y}_{t+1|t}\right)^T
           + R ,
\end{equation}
where $R$ is the injected measurement noise bias again used to ensure a
positive-definite covariance.  The cross correlation between the estimated
state and measurement may also be calculated:
\begin{equation} \label{eq:Pxy}
  P_{xy} = \sum_i w_i \left(\hat{\x}_i - \bar{\x}_{t+1|t}\right)
                     \left(\hat{\y}_i - \bar{\y}_{t+1|t}\right)^T .
\end{equation}

\begin{algorithm}[t]
  \setlength\abovedisplayskip{2pt}
  \setlength\belowdisplayskip{2pt}
  \caption{Unscented Kalman Filter}
  \label{alg:ukf}
  \begin{algorithmic}[1]
    \STATE Form weighted sigma points $\X_t=\{w_i, \x_i\}_{i=0}^{2n}$ around
    current mean $\x_t$ and covariance $P_t$ with scaling factor $\zeta$
    \begin{equation*}
      \x_0 = \x_t
      \qquad
      \x_i    = \x_t + [\sqrt{\zeta P_t}]_i
      \qquad
      \x_{i+n} = \x_t - [\sqrt{\zeta P_t}]_i
    \end{equation*}
    \STATE Predict the new sigma points and observations
    \begin{equation*}
      \X_{t+1|t} = f[\X_t]   \qquad   \Y_{t+1|t} = h[\X_{t+1|t}]
    \end{equation*}
    \STATE Compute weighted means and covariances, \eg
    \begin{equation*}
      \bar{\x}_{t+1|t} = \sum_i w_i ~ \x_i
      \qquad
      P_{xy} = \sum_i w_i (\x_i - \bar{\x}_{t+1|t})(\y_i - \bar{\y}_{t+1|t})^T
    \end{equation*}
    \STATE Update estimate using Kalman gain $K$ and scanner measurement
    $\y_t$
    \begin{equation*}
      K = P_{xy}P_{yy}^{-1}
      \qquad
      \x_{t+1} = \bar{\x}_{t+1|t} + K(\y_t - \bar{\y}_{t+1|t})
      \qquad
      P_{t+1} = P_{xx} - K P_{yy} K^T
    \end{equation*}
  \end{algorithmic}
\end{algorithm}
As is done in the classic linear Kalman filter, the final step is to use the
Kalman gain, $K = P_{xy}P_{yy}^{-1}$, to correct our prediction and provide us
with the final estimated system mean and covariance,
\begin{equation} \label{eq:x_}
  \x_{t+1} = \bar{\x}_{t+1|t} + K(\y_t - \bar{\y}_{t+1|t})
\end{equation}
\begin{equation} \label{eq:P_}
  P_{t+1} = P_{xx} - K P_{yy} K^T ,
\end{equation}
where $\y_t \in \R^m$ is the actual signal measurement taken at this time.
\autoref{alg:ukf} summarizes this algorithm for unscented Kalman filtering.



\section{The algorithm} \label{sec:alg}

To summarize the proposed technique, we are using the unscented Kalman filter
to estimate the local model parameters as we trace out each fiber.  For each
fiber, we maintain the position at which we are currently tracing it and the
current estimate of its model parameters (mean and covariance).  At each
iteration of the algorithm, we predict the new state, which in this case is
simply identity ($\x_{t+1|t}=\x_t$) as we assume the fiber does not change
drastically in character from one position to the next.  Our actual
measurement $\y_t$ in \autoref{eq:x_} is the diffusion-weighted signal, \s,
recorded by the scanner at this position.  At subvoxel positions we
interpolate directly on the diffusion-weighted images.  With these, we step
through the equations above to find the new estimated model parameters,
$\x_{t+1}$.  Last, we use path integration to move a small step in the most
consistent of principal diffusion directions, and then we repeat these steps
from that new location.  \autoref{alg:loop} outlines the feedback loop between
filtering and tractography.
\begin{algorithm}[t]
  \setlength\abovedisplayskip{2pt}
  \setlength\belowdisplayskip{2pt}
  \caption{Main loop repeated for each fiber}
  \label{alg:loop}
  \begin{algorithmic}[1]
    \REPEAT
      \STATE Form the sigma points $\X_t$ around $\x_t$
      \STATE Predict the new sigma points $\X_{t+1|t}$ and observations
      $\Y_{t+1|t}$
      \STATE Compute weighted means and covariances, \eg $\bar{\x}_{t+1|t}$, $P_{xy}$
      \STATE Update estimate ($\x_{t+1}$, $P_{t+1}$) using scanner
      measurement ($\y_t$)
      \STATE Proceed in most consistent direction $\m_j$
    \UNTIL{estimated model appears isotropic}
  \end{algorithmic}
\end{algorithm}






\chapter{Results}

\begin{figure}[t]
  \centering
  \subfloat[Fiber passing through 60\deg synthetic crossing.]{
    \label{fig:crossing}
    \includegraphics[width=.4\columnwidth]{zoom_2T_b1000_fiber}}%
  \subfloat[Estimated ODFs along fiber.]{
    \label{fig:zooms}
    \includegraphics[width=.4\columnwidth]{zoom_2T_b1000_zoom}}
  \caption{One fiber passing through an example synthetic field and the
    estimated ODFs within crossing region \textit{(blue box)} using
    (unsharpened) spherical harmonics (SH) and the filter.  The filter
    provides consistent angular resolution while SH modeling at those same
    locations sometimes misses or is off.  Above and below the crossing
    region, the filter aligns both tensor components to fit the single-tensor
    signal.}
\end{figure}
The supporting results of this dissertation are divided into three sections.
First, we focus on the equally-weighted diffusion tensor model treating both
two and three component models.  Second, we switch to the Watson directional
function as a model and repeat several of these experiments.  Last, we extend
this technique to weighted mixtures and employ a constrained filter.
%
Also, a Python implementation of filtered tractography is made available in
Slicer 3.6.

\section{Tensors}

We begin with synthetic experiments to validate our technique against ground
truth.  After constructing a set of crossing fiber fields, we perform
tractography and examine the underlying orientations and branchings
(\autoref{sec:tracts}).  We then turn to more quantitative experiments over a
broad range of angles and component weightings, and we confirm that our
approach accurately recognizes crossing fibers (\autoref{sec:angle}) and
provides a superior estimate of the diffusion process
(\autoref{sec:quantities}).
%
We then demonstrate estimation of three-fiber crossings (\autoref{sec:3T}).
%
Lastly, we examine a real dataset to demonstrate how causal estimation is able
to pick up fibers and branchings known to exist \invivo yet absent using an
assortment of other techniques (\autoref{sec:2T_real}).  This section follows our
work found in \cite{malcolm2010tmi}.

\begin{figure}[t]
  \centering
  \subfloat[\textbf{Angular error}]{ \label{fig:2T_angle}
    \begin{minipage}[b]{\linewidth}
      \includegraphics[width=.32\textwidth]{angle_w_2T_b1000_1}
      \includegraphics[width=.32\textwidth]{angle_w_2T_b1000_2}
      \includegraphics[width=.32\textwidth]{angle_w_2T_b1000_3}
    \end{minipage}}

  \subfloat[\textbf{FA error, no crossing}]{ \label{fig:lambda_t00}
    \begin{minipage}[b]{0.32\linewidth}
      \includegraphics[width=\textwidth]{lambda_b1000_w50_t00} \\
      \includegraphics[width=\textwidth]{lambda_b1000_w60_t00} \\
      \includegraphics[width=\textwidth]{lambda_b1000_w70_t00}
    \end{minipage}}%
  \subfloat[\textbf{FA error, 45\deg crossing}]{ \label{fig:lambda_t45}
    \begin{minipage}[b]{0.32\linewidth}
      \includegraphics[width=\textwidth]{lambda_b1000_w50_t45} \\
      \includegraphics[width=\textwidth]{lambda_b1000_w60_t45} \\
      \includegraphics[width=\textwidth]{lambda_b1000_w70_t45}
    \end{minipage}}%
  \subfloat[\textbf{FA error, 90\deg crossing}]{ \label{fig:lambda_t90}
    \begin{minipage}[b]{0.32\linewidth}
      \includegraphics[width=\textwidth]{lambda_b1000_w50_t90} \\
      \includegraphics[width=\textwidth]{lambda_b1000_w60_t90} \\
      \includegraphics[width=\textwidth]{lambda_b1000_w70_t90}
    \end{minipage}}
  \caption{The error in estimated crossing angle \subref{fig:2T_angle} and
    estimated fractional anisotropy
    \subref{fig:lambda_t00}-\subref{fig:lambda_t90} at various volume
    fractions: 50\%-50\% (top row), 60\%-40\% (middle row), 70\%-30\% (bottom
    row).  Where appropriate, we compare among single-tensor least-squares
    \textit{(green)}, sharpened spherical harmonics \textit{(red)}, two-tensor
    Levenberg-Marquardt \textit{(blue)}, and the proposed filtered two-tensor
    \textit{(black)}.  Through all examples, the filtered technique provides
    the most consistent and accurate results.}
  \label{fig:T_synthetic}
\end{figure}

Following the experimental method of generating synthetic data found in
\cite{Tournier2004,Descoteaux2009tmi,Schultz2008}, we pulled from our real
data set the 300 voxels with highest fractional anisotropy (FA) and compute
the average eigenvalues among these voxels:  $\{1200, 100, 100\}\mu$m$^2$/msec
(FA=0.91).  We generated synthetic MR signals according to
\autoref{eq:general_model} using these eigenvalues to form an anisotropic
tensor at \bone s/mm$^2$ using 81 gradient directions uniformly spread on the
hemisphere.  We assume $s_0=1$ and introduce Rician noise (\snr{5}).  For
extra experiments at \bthree and alternative noise levels, one can refer to an
earlier conference version of this work \cite{Malcolm2009ipmi}.

In these experiments, we compare against several techniques selected to
represent standard alternative models and fitting procedures.
%
First, we include direct (least-squares) single-tensor estimation and
streamline tractography \cite{Basser2000}.  Despite its limited utility in
crossing and branching regions, this technique is widely used in both the
clinical and neuroscience communities and provides a baseline for comparison.
%
Second, we use the same two-tensor model from \autoref{sec:model} but estimate
the model parameters using a Levenberg-Marquardt nonlinear least-squares
solver.  This shows the effect of filtered estimation versus a standard
alternative scheme.  Such techniques depend largely on the initialization, and
so we employ several variants to remove any uncertainty in initialization.
For the synthetic experiments, we initialize with the ground truth.  For the
\invivo experiments, we initialize with the single-tensor estimate
\cite{Peled2006} which we loosely refer to as ``independent'' and we
initialize with the estimate at the previous position which we refer to as
``causal''.
%
Third, we use spherical harmonics for modeling and fiber-ODF sharpening for
peak detection (order $l=8$, regularization $L=0.006$)
\cite{Tournier2004,Descoteaux2009tmi}.  This provides a comparison with an
independently estimated, model-free representation.  Note that this technique
is very similar to spherical deconvolution.  We will often refer to this
method as ``sharpened spherical harmonics''.


The unscented Kalman filter conveniently requires few parameters.
Specifically, of importance are the matrices for injecting model noise $Q$ and
injecting measurement noise $R$ (see \autoref{eq:Pxx} and \autoref{eq:Pyy}).
Fortunately, the relative magnitude of each can be determined off-line from
the data itself, and typically these are formulated as diagonal matrices
(zeros off the diagonal).
%
The injected model noise governs how much variance is allowed in the model:
higher values allow more variation but, pushed too far, could lead to
inaccurate estimation.  We found roughly $q_\m \in [0.0015, 0.0030]$ (roughly
2-4\deg) and $q_\lambda \in [25,100]$ to allow an appropriate amount of
angular and diffusive flexibility among our synthetic and \invivo experiments
as well as among various other patients and across other scanner protocols we
have encountered.
%
The injected measurement noise governs how much variance is expected in the
measurement: higher values mean we expect more variance and hence trust our
measurement less.  This value depends on the level of physical noise present
which varies depending on the scanner, protocol, or pre-processing, and so
some experimentation may be necessary.  However, in all our experiments thus
far, we have found $r_s \in [0.01, 0.03]$ to quite robust.



\subsection{Synthetic tractography}  \label{sec:tracts}

While the independent optimization techniques can be run on individually
generated voxels, care must be taken in constructing reasonable scenarios to
test the causal filter.  For this purpose, we constructed an actual 2D field
through which to navigate.  In the middle is one long fiber pathway where the
filter begins estimating a single tensor but then runs into a field of voxels
with two crossed fibers at a fixed angle.  We generated several similar
fields, each at a different fixed angle and component weighting.
%
In \autoref{fig:crossing} we show one such field with a 60\deg crossing.  In
our experiments, fibers start from the bottom and propagate upward where they
encounter the crossing region.
%
We found that in regions with only one true fiber present (those outside the
crossing here), the second component consistently aligned with the first.

\begin{figure}[t]
  \centering
  \subfloat[Low-noise (\snr{10})]{
    \includegraphics[width=.32\columnwidth]{angle_3T_b1000_SH_clean}}%
  \subfloat[High-noise (\snr{5})]{
    \includegraphics[width=.32\columnwidth]{angle_3T_b1000_SH_dirty}}
  \caption{The filtered approach \textit{(black)} is able to resolve
    three-fiber crossings with improved accuracy and at sharper angles
    compared to using sharpened spherical harmonics \textit{(red)}.  Both
    low-noise and high-noise experiments are shown.}
  \label{fig:angle_3T}
\end{figure}
In \autoref{fig:zooms} we take a closer look at several points along this
single fiber as it passes through the crossing region.  We examine the
reconstructed ODFs produced by the filter as well as those produced by
spherical harmonic modeling at those same positions.  As reported in
\cite{Descoteaux2009tmi,Schultz2008}, spherical harmonics at \bone begin to
not detect the second component at around 50\deg-60\deg, but instead report a
single angle as seen in one of the middle samples in \autoref{fig:zooms}.  As
reported in \cite{Zhan2006,Tournier2007,Schultz2008}, a close examination of
the reported axes shows this bias toward a single averaged axis.  In contrast,
the filtered results are consistent and accurate.  One can note a slight
deflection upon entry of the crossing region as the filter attempts to
maintain smooth estimates.  The deflection is lessened upon exit since the
single component allows for the most stable model fit.


\begin{figure}[t]
  \centering
  \subfloat[Single-tensor streamline]{ \label{fig:T_1T_tc}%
    \includegraphics[width=.49\columnwidth]{case01045_1T_tc}}%
  \subfloat[Spherical harmonics]{ \label{fig:T_SH_tc}%
    \includegraphics[width=.49\columnwidth]{case01045_SH_tc}}

  \subfloat[Independent Levenberg-Marquardt two-tensor]{ \label{fig:LMi_tc}%
    \includegraphics[width=.49\columnwidth]{case01045_LMi_tc}}%
  \subfloat[Causal Levenberg-Marquardt two-tensor]{ \label{fig:LMc_tc}%
    \includegraphics[width=.49\columnwidth]{case01045_LMc_tc}}

  \subfloat[Filtered two-tensor]{ \label{fig:2T_tc}%
    \includegraphics[width=.49\columnwidth]{case01045_2T_tc}}%
  \subfloat[Filtered three-tensor]{ \label{fig:3T_tc}%
    \includegraphics[width=.49\columnwidth]{case01045_3T_tc}}
  \caption{Tractography using various methods, seeded at the center of the
    corpus callosum.  Single-tensor reconstructs only the dominate callosal
    radiata \subref{fig:T_1T_tc}.  Spherical harmonics pick up some of the
    lateral branches \subref{fig:T_SH_tc}.  Initializing a Levenberg-Marquardt
    solver with an independent single tensor estimate finds only the radiata
    \subref{fig:LMi_tc}, while initializing instead with its previous estimate
    \subref{fig:LMc_tc} finds little more than spherical harmonics.  Only
    filtered tractography picks up the lateral paths consistent with the
    underlying anatomy.  Fibers exiting $\pm22\,\text{mm}$ around the
    mid-sagittal plane are indicated in \textit{blue}.  Seed region indicated
    in \textit{yellow}.}
  \label{fig:T_tc}
\end{figure}


\subsection{Angular resolution}  \label{sec:angle}

Having verified the underlying behavior, we then began a more comprehensive
evaluation and quantified the estimated angle within the crossing regions.
Synthetic crossing fields were constructed with a range of crossing angles and
weighting combinations.  In \autoref{fig:2T_angle} each row is a different
weighting: top 50-50, middle 60-40, bottom 70-30.  Each graph then plots the
angular error as a function of crossing angle, from 15\deg to 90\deg.  Within
the crossing regions, we compared the performance among direct single-tensor
estimation, sharpened spherical harmonics, a nonlinear solver
(Levenberg-Marquardt), and the proposed filter.
%
By varying the size of the crossing regions in \autoref{fig:crossing} or the
number of fibers run, we ensured that each technique performed estimation on
at least 500 voxels to produce consistent trendlines across this wide range of
angles.

\autoref{fig:2T_angle} shows the estimated separating angle reported using
spherical harmonics \textit{(red)}, the nonlinear solver \textit{(blue)}, and
the proposed filter \textit{(black)}.  Over each technique's series of
estimations, the trendlines indicate the mean error while the bars indicate
one standard deviation.
%
Consistent with the synthetic results reported in
\cite{Descoteaux2009tmi,Descoteaux2007mrm}, spherical harmonics are generally
unable to detect and resolve angles below 50\deg for \bone or below 40\deg for
\bthree.  With a perfect initialization, the Levenberg-Marquardt solver is
closer to the solution but shows significant variance from the noise.  In
contrast, the filtered approach statistically estimates the true underlying
signal and so is capable of resolving angles down to the range of 20-30\deg
with 5\deg error in the equally weighted field (50-50) \textit{(top row)} with
performance degrading little in the 60-40 field \textit{(middle)}.  In the
70-30 field \textit{(bottom)}, the trendlines begin to show an asymptotic
limit to performance, yet the filter is the only method capable of reasonable
estimates at large angles.  See \citet{Malcolm2009ipmi} for additional
experiments at both \bone and \bthree.  From these trendlines, one can
conclude that the filtered approach provides accurate and consistent angular
resolution at crossing angles far below independent estimation.




\subsection{Estimated quantities} \label{sec:quantities}

While angular resolution is important in accurately resolving paths, the
underlying estimated quantities are important in the analysis of these
pathways.
%
We focus on three scenarios: a single fiber (no crossing), a 45\deg crossing,
and an orthogonal 90\deg crossing.  In each scenario we hold constant the
primary eigenvalue and adjust the minor eigenvalues to produce fields with a
range of diffusion properties.  Further, we adjust the crossing weights as in
the earlier experiments.
%
We compare among the three techniques estimating diffusion quantities: direct
single-tensor estimation \textit{(green)}, the two-tensor model with the
nonlinear solver (Levenberg-Marquardt initialized with ground truth)
\textit{(blue)}, and the two-tensor model with the proposed filter
\textit{(black)}.
%
And finally we judge these techniques using the error in estimated fractional
anisotropy (FA), one of the most common proxy measures for diffusion.

The fields with a single fiber (no crossing) serve as another baseline.
\autoref{fig:lambda_t00} shows that both direct estimation and the filter
provide tight estimates of the true FA.  Note the increased variance incurred
in the nonlinear solver due to noise.
%
In \autoref{fig:lambda_t45}, single-tensor estimation begins to show a bias.
Levenberg-Marquardt begins to deteriorate slightly, but the filter continues
to provide accurate and consistent estimates.
%
Lastly, in \autoref{fig:lambda_t90}, single-tensor estimation provides
erroneous estimates while only the multi-component techniques are able to
maintain accuracy.  Note again how the filter provides tight estimates.

\begin{figure}[th]
  \centering
  \subfloat[Streamline single-tensor]{ \label{fig:T_1T_cc}
    \begin{minipage}[b]{0.32\linewidth}
      \includegraphics[width=\textwidth]{case01045_DT_cc_top} \\
      \includegraphics[width=\textwidth]{case01045_DT_cc_front}
    \end{minipage}}%
  \subfloat[Streamline spherical harmonics]{ \label{fig:T_SH_cc}
    \begin{minipage}[b]{0.32\linewidth}
      \includegraphics[width=\textwidth]{case01045_SH_cc_top}
      \includegraphics[width=\textwidth]{case01045_SH_cc_front}%
    \end{minipage}}%
  \subfloat[Filtered two-tensor]{ \label{fig:2T_cc}
    \begin{minipage}[b]{0.32\linewidth}
      \includegraphics[width=\textwidth]{case01045_2T_cc_top}
      \includegraphics[width=\textwidth]{case01045_2T_cc_front}
    \end{minipage}}
  \caption{Tracing fibers originating from the center of the entire corpus
    callosum with views from above \textit{(top rows)} and front-to-back
    \textit{(bottom rows)}.  The proposed filtered tractography is able to
    find many of the lateral projections \textit{(blue)} while single-tensor
    is unable to find any and few are found with sharpened spherical
    harmonics.  Seed region indicated in \textit{yellow}.}
  \label{fig:T_cc}
\end{figure}



\subsection{Volume fractions}  \label{sec:weights}

In the current implementation, we have chosen a model without weights
(\autoref{eq:2T_model}).  To examine this assumption, we included experiments
over weighted fields to see the effect on estimation.
%
Each row of \autoref{fig:T_synthetic} shows a different weighting combination:
equal in the top row to most asymmetric in the bottom row.

Despite the equally-weighted assumption, \autoref{fig:2T_angle} shows that the
filter is capable of correctly resolving angles in the range of 60-40 and only
the most orthogonal angles at 70-30.  In all of these runs, the filter
maintained tracing of the dominant fiber, drifting little in the most
asymmetric cases.  At 80-20 no technique was able to reliably detect the minor
component.

While the filter had trouble picking up the minor component in the more
asymmetric cases, \autoref{fig:lambda_t00}-\ref{fig:lambda_t90} shows that it
maintained accurate estimates of the diffusion processes, perhaps the most
important consideration.
%
In these regions where the model does not explicitly fit the data, we found
that the filter compensates by adjusting the eigenvalues.  For example,
changes in the eigenvalues could be interpreted as weights:
%
$
 e^{ -b \u_i^T D \u_i }
   = e^{ -b \u_i^T (D_a + D_b) \u_i }
   = e^{ -b \u_i^T D_a \u_i } e^{ -b \u_i^T D_b \u_i }
   = w e^{ -b \u_i^T D_b \u_i } .
$
%
In each of the unequally weighted crossings, the filter increases the
eigenvalues of the primary component and decreases those of the minor
component in order to fit the signal with much the same effect as increasing
or decreasing weights.  Since all techniques appear to deteriorate beyond the
70-30 weighting, it appears that components of less than 30\% contribution
will not be reliably detected despite incorporating weights.

The unscented Kalman filter itself places no constraint on the state space
except what is statistically likely given the current estimate:  all values in
the state vector are free to take any value in \R.  In our current
implementation, we restricted $\m$ to $\S^2$ and $\lx,\ly > 0$.  However, this
is an ad hoc solution and we have experimented with applying causal filters
capable of estimating under model constraints, \eg positive eigenvalues,
convex weights \cite{malcolm2009cukf}.




\subsection{Three-fiber crossings} \label{sec:3T}

Resolving three-fiber crossings has proven difficult for many techniques,
especially at the lower $b$-values typically used in clinical scans.  For
example, \citet{Tuch2002} found the general multi-tensor model to be unstable
for three or more components using data at \bn{1077} with 126 gradients.
\citet{Bergmann2006} only reported results for up to two-tensors (\bn{700}, 30
gradients).  In simulation, \citet{Behrens2007} found that $b$-values at
upwards of 3000-4000 s/mm$^2$ were required for detecting more than two fibers
and none were found \invivo (\bn{1000}, 60 gradients).  Further, many studies
specifically use at most two orientations
\cite{Alexander2001,Kreher2005,Peled2006}.
%
However, spherical shell techniques have had some success in resolving
three-fiber crossings.  \citet{Tournier2004} reported such crossings using
spherical deconvolution (\bn{2971}, 60 gradients).  Most recently,
\citet{Schultz2008} demonstrated tensor decomposition as a promising technique
for resolving such configurations (\bn{1000}, 60 gradients).


Following the synthetic experimental setup of \citet{Schultz2008}, we
constructed an additional set of synthetic fields this time with three
equally-weighted Gaussian components.  Similar to the synthetic fields shown
in \autoref{fig:crossing}, one fiber is angled up and is the intended
orientation to track through the region, but here the other two orientations
were set so that the endpoints of the three principal axes formed an
equilateral triangle with any two separated by the specified angle.  With this
setup, \autoref{fig:angle_3T} shows that the filtered approach provides an
accurate estimate that reaches to roughly 45\deg compared to 60-65\deg for
spherical harmonics.  A significant bias is apparent at more acute angles
using either technique.




\begin{figure}[t]
  \centering
  \includegraphics[width=.19\textwidth]{000018}
  \includegraphics[width=.19\textwidth]{000021}
  \includegraphics[width=.19\textwidth]{000024}
  \includegraphics[width=.19\textwidth]{000027}
  \includegraphics[width=.19\textwidth]{000035}
  \caption{Successive snapshots of filtered two-tensor tracing from both the
    corpus callosum \textit{(red)} and internal capsule \textit{(yellow)},
    viewed from right hemisphere.  Here we see the lateral pathways from the
    corpus cross the motor tracts from the internal capsule.}
  \label{fig:cc_ic}
\end{figure}
\begin{figure}[t]
  \centering
  \includegraphics[width=.49\columnwidth]{3cross_context}
  \includegraphics[width=.49\columnwidth]{3cross}
  \caption{The three-tensor filtered approach is able to trace through this
    intersection of the corticospinal tract \textit{(blue)}, corpus callosum
    \textit{(red)}, and superior longitudinal fasciculus \textit{(green)}.}
  \label{fig:intersection_3T}
\end{figure}





\subsection{\Invivo tractography}  \label{sec:2T_real}

We next test our approach on a real human brain scan acquired on a 3-Tesla GE
system using an echo planar imaging (EPI) diffusion weighted image sequence.
A double echo option was used to reduce eddy-current related distortions.  To
reduce impact of EPI spatial distortion, an eight channel coil was used to
perform parallel imaging using Array Spatial Sensitivity Encoding Techniques
(GE) with a SENSE-factor (speed-up) of 2.  Acquisitions have 51 gradient
directions with \bn{900} and eight baseline scans with \bn{0}.  The original
GE sequence was modified to increase spatial resolution, and to further
minimize image artifacts.  The following scan parameters were used: TR 17000
ms, TE 78 ms, FOV 24 cm, 144x144 encoding steps, 1.7 mm slice thickness. All
scans had 85 axial slices parallel to the AC-PC line covering the whole brain.
In addition, \bn{0} field inhomogeneity maps were collected and calculated.

We first focus on fibers originating in the corpus callosum.  Specifically, we
seek to trace out the lateral transcallosal fibers that run through the corpus
callosum out to the lateral gyri.  It is known that single-tensor streamline
tractography only traces out the dominant pathways forming the U-shaped
callosal radiation (\autoref{fig:sweetness}, \ref{fig:T_1T_tc}, and
\ref{fig:T_1T_cc}).  Several studies document this phenomena, among them the
works of Descoteaux, Schultz, \etal \cite{Descoteaux2009tmi,Schultz2008} have
side-by-side comparisons.  These fibers have been reported in using diffusion
spectrum imaging \cite{Hagmann2004}, probabilistic tractography
\cite{Kaden2007,Anwander2007,Descoteaux2009tmi}, and more recently with tensor
decomposition \cite{Schultz2008}.

We start with two basic experiments:  first examining the tracts surrounding a
single coronal slice and second looking at all tracts passing through the
corpus callosum.  We seed each algorithm multiple times in voxels at the
intersection of the mid-sagital plane and the corpus callosum.  We terminate
fibers when the generalized fractional anisotropy of the estimated signal
(std/rms) fell below 0.1.  To explore branchings found using the proposed
technique, we consider a component to be branching if it was separated from
the primary component by less than 40\deg with $FA\ge0.15$.  Similarly, with
sharpened spherical harmonics, we consider it a branch if we find additional
maxima over the same range.  Unless otherwise stated, in all experiments we
follow the primary fiber from the seeding and its branches, \ie one level of
branching.  We do not go on to follow the branches on those secondary fibers.
While such heuristics are somewhat arbitrary, we found little qualitative
difference in adjusting these thresholds.

To demonstrate the flexibility of the proposed filtering strategy with respect
to model choice, we use both the two-tensor fiber model
(\autoref{eq:2T_model}) and the three-tensor fiber model
(\autoref{eq:3T_model}).  While this introduced differences in the quantity of
branchings detected, we found that using either model resulted in generally
the same pathways.  This suggests that the filtering itself accounted for most
of the differences compared to the other techniques, more so than the choice
of two or three components in the fiber model.  Further, it is important to
note that despite the more complicated multi-fiber models, the filter provides
stable and consistent estimates of the appropriate number of compartments.
For example, when tracing using the two-tensor fiber through a one-tensor
region, the filter overlaps both tensors to fit the signal (above and below
the crossing region in \autoref{fig:crossing}).

For the first experiment, \autoref{fig:T_tc} shows tracts originating from
within a few voxels intersecting a particular coronal slice.  As a reference,
we use a coronal slice showing the intensity of fractional anisotropy (FA)
placed a few voxels behind the seeded coronal position.  Keeping in mind that
these fibers are intersecting or are in front of the image plane, this roughly
shows how the fibers navigate the areas of high anisotropy (bright regions).
Comparable to the results in \cite{Descoteaux2009tmi,Schultz2008},
\autoref{fig:T_SH_tc} shows that sharpened spherical harmonics only pick up a
few fibers intersecting the U-shaped callosal radiata.  In contrast, our
proposed method traces out many pathways consistent with the apparent anatomy
using either the two-fiber or three-fiber model.  To emphasize transcallosal
tracts, we color as blue those fibers exiting a corridor of $\pm22\,\text{mm}$
around the mid-sagittal plane.  To explore the maximum connectivity found
using each approach, this is the one experiment where we followed secondary
branches, \ie primary fibers, their branches, and finally branches off those
fibers.  \autoref{fig:T_cc} shows a view of the whole brain to see the overall
difference between the different methods.

\begin{figure}[t]
  \centering
  \subfloat[Levenberg-Marquardt (causal) two-tensor]{ \label{fig:slf_LM} %
    \includegraphics[width=.49\columnwidth]{case01045_slf_LM_roi} %
    \includegraphics[width=.49\columnwidth]{case01045_slf_LM_roi_front}}
  \subfloat[Filtered two-tensor]{ \label{fig:slf_2T} %
    \includegraphics[width=.49\columnwidth]{case01045_slf_2T_roi} %
    \includegraphics[width=.49\columnwidth]{case01045_slf_2T_roi_front}} %
  \caption{A portion of the superior longitudinal fasciculus as it crosses a
    section of the lateral pathways emanating from the corpus callosum.  These
    fibers were extracted using two regions of interest \textit{(yellow)}.
    While LM reconstructs several of these fibers, filtered estimation
    produces a more consistent bundle with deeper, uniform penetration into
    the frontal and occipital lobes.  No such fibers were found using the
    comparison methods.}
  \label{fig:slf}
\end{figure}




\begin{figure}[t]
  \centering
  \subfloat[Streamline single-tensor]{ \label{fig:cing_DT}%
    \includegraphics[width=.32\textwidth]{case01045_cing_DT}}%
  \subfloat[Levenberg-Marquardt (causal) two-tensor]{ \label{fig:cing_LM}%
    \includegraphics[width=.32\textwidth]{case01045_cing_LM}}%
  \subfloat[Spherical harmonics]{ \label{fig:cing_SH}%
    \includegraphics[width=.32\textwidth]{case01045_cing_SH}}

  \subfloat[Filtered single-tensor]{ \label{fig:cing_1T}%
    \includegraphics[width=.32\textwidth]{case01045_cing_1T}}
  \subfloat[Filtered two-tensor]{ \label{fig:cing_2T}%
    \includegraphics[width=.32\textwidth]{case01045_cing_2T}}
  \caption{Tracing of the right cingulum bundle using various methods.  From
    full-brain seeding, fibers were selected that passed through any two gates
    \textit{(yellow)}.  While all methods here produce some false-positives
    (\eg partial voluming onto the splenium, genu, or tapetum), the filtered
    methods appear to produce the fullest cingulum reconstructions.}
  \label{fig:cing}
\end{figure}




\begin{figure}[t]
  \centering
  \subfloat[Streamline single-tensor]{ \label{fig:ioff_DT}%
    \includegraphics[width=.32\textwidth]{case01045_ioff_DT}}%
  \subfloat[Levenberg-Marquardt (causal) two-tensor]{ \label{fig:ioff_LM}%
    \includegraphics[width=.32\textwidth]{case01045_ioff_LM}}%
  \subfloat[Spherical harmonics]{ \label{fig:ioff_SH}%
    \includegraphics[width=.32\textwidth]{case01045_ioff_SH}}%

  \subfloat[Filtered single-tensor]{ \label{fig:ioff_1T}%
    \includegraphics[width=.32\textwidth]{case01045_ioff_1T}}%
  \subfloat[Filtered two-tensor]{ \label{fig:ioff_2T}%
    \includegraphics[width=.32\textwidth]{case01045_ioff_2T}}%

  \caption{Tracing of the right inferior fronto-occipital fasciculus using
    various methods.  From full-brain seeding, fibers were selected that
    passed through any two gates \textit{(yellow)}.  Spherical harmonics and
    Levenberg-Marquardt show sparse and irregular connectivity.  Filtered
    single- and two-tensor results appear to have deeper and more uniform
    penetration into the occipital lobe.}
  \label{fig:ioff}
\end{figure}



\begin{figure}[t]
  \centering
  \subfloat[Streamline single-tensor]{ \label{fig:ioff_DT_top}%
    \includegraphics[width=.32\textwidth]{case01045_ioff_DT_top}}%
  \subfloat[Levenberg-Marquardt (causal) two-tensor]{ \label{fig:ioff_LM_top}%
    \includegraphics[width=.32\textwidth]{case01045_ioff_LM_top}}%
  \subfloat[Spherical harmonics]{ \label{fig:ioff_SH_top}%
    \includegraphics[width=.32\textwidth]{case01045_ioff_SH_top}}%

  \subfloat[Filtered single-tensor]{ \label{fig:ioff_1T_top}%
    \includegraphics[width=.32\textwidth]{case01045_ioff_1T_top}}%
  \subfloat[Filtered two-tensor]{ \label{fig:ioff_2T_top}%
    \includegraphics[width=.32\textwidth]{case01045_ioff_2T_top}}%

  \caption{A view of the previous figure from above.  Note the consistent
    deeper penetration of the filtered techniques.  The two-tensor method
    infiltrates many minor pathways into the temporal lobe.}
  \label{fig:ioff_top}
\end{figure}



We next combined the filtered tracings from the corpus callosum with those
from the internal capsule to demonstrate practically how the filter is able to
push through areas of crossing.  \autoref{fig:cc_ic} shows successive
snapshots as two-tensor filtered tracings from the corpus callosum
\textit{(red)} infiltrate those originating in the internal capsule
\textit{(yellow)}.  Since other methods were unable to reconstruct such dense
penetration, it is our hope that this method of multi-tensor tractography will
provide rich information in connectivity studies.
%
\autoref{fig:intersection_3T} takes a closer look at one area where these
fiber pathways merge and shows fragments from the fibers traversing through
this region using the three-tensor model.

In the experiments thus far, the decision to branch is somewhat ad hoc.  An
alternative approach to avoid such thresholds is to perform full-brain
tractography following only one path and then select fibers using masks.
Along these lines we seeded every voxel with $FA\ge0.15$ using each technique
discussed so far.  In addition, we included filtered single-tensor
tractography to contrast to the baseline single-tensor streamline.

We select out three well-studied pathways.
%
\autoref{fig:slf} shows the first pathway, a portion of the superior
longitudinal fasciculus as it crosses the lateral pathways exposed in earlier
figures.  Only two techniques were able to produce this portion of the tract:
the causal variant of Levenberg-Marquardt (initializes each step with previous
solution) and filtered two-tensor tractography.  Both share the same general
shape, but the filtered version appears to show a superior reconstruction,
especially where the endpoints insert.  Spherical harmonics were unable to
traverse through the lateral crossings here likely due to ambiguous peaks
along this corridor, and the single-tensor models are inherently incapable of
modeling the crossing.

\autoref{fig:cing} shows the second pathway, the cingulum bundle as it passes
through several gates \textit{(yellow)}.  Each technique incurs false-positive
fibers at the ends of the corpus--the genu (anterior) and splenium
(posterior)--where partial voluming often leads to fibers straying onto the
corpus callosum.  While all techniques have difficulty making the posterior
bend, they begin to differ in their reconstructions of the main body.  Going
from streamline single-tensor to filtered single-tensor, we see a fuller
reconstruction.  Levenberg-Marquardt shows many false positives as it easily
finds incorrect minima.  Sharpened spherical harmonics provide an accurate
although sparse reconstruction.  Filtered two-tensor provides a full
reconstruction although it produces many false positives.

Last, \autoref{fig:ioff} shows tracing of the interior occipital-fronto
fasciculus as it spreads and inserts into the occipital and temporal lobes.
Streamline single tractography reconstructs the central pathway.  Only the
filtered techniques provide consistently deeper penetration into the gray
matter while retaining coherent paths.  Further, only the filtered version of
the two-tensor model reconstructs the known minor insertions into the temporal
lobe.  This is even more pronounced in the views from above in
\autoref{fig:ioff_top}.






\section{Watson directional functions}

We now treat the evaluation of the Watson directional function separately.
%
Since this model is an approximation of the tensor diffusion model, we begin
with experiments examining reconstruction error as well as those measuring
angular error before moving quickly to \invivo experiments.
%
This section follows our work in \citet{malcolm2010mia}.

\begin{figure}[t]
  \centering
  \subfloat[\bone]{
    \includegraphics[width=.24\columnwidth]{mse_2W_b1000_clean}%
    \includegraphics[width=.24\columnwidth]{mse_2W_b1000_dirty}}
  \subfloat[\bthree]{
    \includegraphics[width=.24\columnwidth]{mse_2W_b3000_clean}%
    \includegraphics[width=.24\columnwidth]{mse_2W_b3000_dirty}}
  \caption{Mean squared error (MSE) between reconstructed signal and ground
    truth signal at various crossing angles \textit{(low-noise on left,
      high-noise on right)}.  Notice how the increased noise has little effect
    on the filter \textit{(black)} compared to using matching pursuit
    \textit{(blue)} or sharpened spherical harmonics \textit{(red)}.}
  \label{fig:mse_2W}
\end{figure}
Throughout these experiments, we draw comparison to three alternative
techniques.
%
First, we use the same two-Watson model from \autoref{sec:model} with a
variant of matching pursuit for brute force, dictionary-based optimization
\cite{Mallat1993}.  In our implementation, we construct a finite dictionary of
two-Watson signals at a range of various $k$-values, essentially discretizing
the search space across orientations and $k$-values.  Given a new measured
signal, the signal from the dictionary with highest inner product provides an
estimate of orientation and concentration.  While our signal is produced at 81
gradient directions, we use 341 directions to construct the dictionary, thus
any error is due to the method's sensitivity to noise and discretization.
Note that by using 341 orientation directions there is roughly an 8\deg
angular difference between offset orientations, hence we see that the angular
error is often at most 8\deg.  This approach highlights the effect of using
the same model but changing the optimization technique to one that treats each
voxel independently.
%
Second, as in the tensor experiments above, we use spherical harmonics for
modeling \cite{Tournier2004} and fiber-ODF sharpening for peak detection as
described in \cite{Descoteaux2009tmi} (order $l=8$, regularization $L=0.006$).
This provides a comparison with an independently estimated, model-free
representation.
%
Last, when performing tractography on real data, we again include
single-tensor streamline tractography as a baseline.

\begin{figure}[t]
  \centering
  \subfloat[Matching pursuit, \bone]{ \label{fig:angle_2W_MP_1000}
    \includegraphics[width=.24\columnwidth]{angle_2W_b1000_PP_clean}
    \includegraphics[width=.24\columnwidth]{angle_2W_b1000_PP_dirty}}%
  \subfloat[Matching pursuit, \bthree]{ \label{fig:angle_2W_MP_3000}
    \includegraphics[width=.24\columnwidth]{angle_2W_b3000_PP_clean}
    \includegraphics[width=.24\columnwidth]{angle_2W_b3000_PP_dirty}}

  \subfloat[Spherical harmonics, \bone]{ \label{fig:angle_2W_SH_1000}
    \includegraphics[width=.24\columnwidth]{angle_2W_b1000_SH_clean}
    \includegraphics[width=.24\columnwidth]{angle_2W_b1000_SH_dirty}}%
  \subfloat[Spherical harmonics, \bthree]{ \label{fig:angle_2W_SH_3000}
    \includegraphics[width=.24\columnwidth]{angle_2W_b3000_SH_clean}
    \includegraphics[width=.24\columnwidth]{angle_2W_b3000_SH_dirty}}
  \caption{Average angle error at various crossing angles comparing all three
    techniques: matching pursuit \textit{(blue)}, sharpened spherical
    harmonics \textit{(red)}, and the proposed filter \textit{(black)}.  The
    filter provides stable and consistent estimation compared to either
    alternative technique.  Each subfigure shows both the low-noise and
    high-noise experiments \textit{(left, right)}.}
  \label{fig:2W_angle}
\end{figure}
For the Watson model, we found that values on the order of $q_\m = 0.001$
(roughly 2\deg), $q_\lambda = 10$, and $r_s=0.02$ were quite robust for the
appropriate diagonal entries of $Q$ and $R$ (see \autoref{eq:Pxx} and
\autoref{eq:Pyy}).  Off-diagonal entries were left at zero.

\begin{figure}[t]
  \centering
  \subfloat[Single-tensor]{ \label{fig:W_1T_tc}%
    \includegraphics[width=.49\columnwidth]{case01045_1T_tc}}%
  \subfloat[Spherical harmonics]{ \label{fig:W_SH_tc}%
    \includegraphics[width=.49\columnwidth]{case01045_SH_tc}}

  \subfloat[Filtered two-Watson]{ \label{fig:2W_tc}%
    \includegraphics[width=.49\columnwidth]{case01045_2W_tc}}%
  \subfloat[Filtered three-Watson]{ \label{fig:3W_tc}%
    \includegraphics[width=.49\columnwidth]{case01045_3W_tc}}
  \caption{Filtered tractography picks up many fiber paths consistent with the
    underlying structures.  Both single-tensor streamline and sharpened
    spherical harmonics are unable to find the majority of these pathways.
    Fibers existing $\pm22\,\text{mm}$ around the mid-sagittal plane are
    indicated in blue.  Seed region indicated in yellow.}
  \label{fig:W_tc}
\end{figure}
\begin{figure}[t]
  \centering
  \subfloat[Filtered two-Watson]{ \label{fig:2W_tc_zoom}%
    \includegraphics[width=.49\columnwidth]{case01045_2W_tc_zoom}}%
  \subfloat[Filtered three-Watson]{ \label{fig:3W_tc_zoom}%
    \includegraphics[width=.49\columnwidth]{case01045_3W_tc_zoom}}
  \caption{Closeup of upper right in \autoref{fig:2W_tc} and
    \autoref{fig:3W_tc}.}
  \label{fig:W_tc_zoom}
\end{figure}
\begin{figure}[t]
  \centering
  \subfloat[Filtered two-Watson]{\label{fig:2W_frontal}%
    \includegraphics[width=.49\columnwidth]{case01045_2W_cc_zoom}}%
  \subfloat[Filtered three-Watson]{\label{fig:3W_frontal}%
    \includegraphics[width=.49\columnwidth]{case01045_3W_cc_zoom}}%
  \caption{Closeup of frontal fibers in \autoref{fig:2W_cc} and
    \autoref{fig:3W_cc} viewed from above.}
  \label{fig:frontal}
\end{figure}
\begin{figure}[t]
  \centering
  \subfloat[Single-tensor]{ \label{fig:1T_ic_front}%
    \includegraphics[width=.49\columnwidth]{case01045_1T_ic_front}}%
  \subfloat[Spherical harmonics]{ \label{fig:SH_ic_front}%
    \includegraphics[width=.49\columnwidth]{case01045_SH_ic_front}}
  \subfloat[Filtered two-Watson]{ \label{fig:2W_ic_front}%
    \includegraphics[width=.49\columnwidth]{case01045_2W_ic_front}}%
  \subfloat[Filtered three-Watson]{ \label{fig:3W_ic_front}%
    \includegraphics[width=.49\columnwidth]{case01045_3W_ic_front}}
  \caption{Frontal view with seeding in the internal capsule \textit{(blue)}.
    While both single-tensor and spherical harmonics tend to follow the
    dominant corticospinal tract to the primary motor cortex, the filtered
    approach follows many more pathways.  Seed region indicated in yellow.}
  \label{fig:ic_front}
\end{figure}
\begin{figure}[t]
  \centering
  \subfloat[Single-tensor]{ \label{fig:1T_ic_top}%
    \includegraphics[width=.49\columnwidth]{case01045_1T_ic_top}}%
  \subfloat[Spherical harmonics]{ \label{fig:SH_ic_top}%
    \includegraphics[width=.49\columnwidth]{case01045_SH_ic_top}}
  \subfloat[Filtered two-Watson]{ \label{fig:2W_ic_top}%
    \includegraphics[width=.49\columnwidth]{case01045_2W_ic_top}}%
  \subfloat[Filtered three-Watson]{ \label{fig:3W_ic_top}%
    \includegraphics[width=.49\columnwidth]{case01045_3W_ic_top}}
  \caption{View from above showing cortical insertion points for each method.
    FA backdrop is taken near the top of the brain.  The filtered approach
    shows more lateral insertions compared to single-tensor and spherical
    harmonic tracts.}
  \label{fig:ic_top}
\end{figure}
\begin{figure}[t]
  \centering
  \subfloat[Single-tensor]{ \label{fig:1T_ic_side}%
    \includegraphics[width=.48\textwidth]{case01045_1T_ic_left}}%
  \subfloat[Spherical harmonics]{ \label{fig:SH_ic_side}%
    \includegraphics[width=.48\textwidth]{case01045_SH_ic_left}}

  \subfloat[Filtered two-Watson]{ \label{fig:2W_ic_side}%
    \includegraphics[width=.48\textwidth]{case01045_2W_ic_left}}%
  \subfloat[Filtered three-Watson]{ \label{fig:3W_ic_side}%
    \includegraphics[width=.48\textwidth]{case01045_3W_ic_left}}
  \caption{Side view with seeding in the internal capsule \textit{(yellow)}.
    Filtered tractography finds many insertions into cortical regions of the
    parietal and occipital lobes.  Seed region indicated in yellow.}
  \label{fig:ic_side}
\end{figure}


\subsection{Signal reconstruction and angular resolution}  \label{sec:mse_angle}

In the first experiment, we look at signal reconstruction error.  We calculate
the mean squared error of the reconstructed signal, \s, against the ground
truth signal, $\hat{\s}$ (pure, no noise): $ \norm{\s - \hat{\s}}^2 /
\norm{\hat{\s}}^2$.  In essence, this is exactly what the filter is trying to
minimize: the error between the reconstructed signal and the measured signal.
\autoref{fig:mse_2W} shows the results of using the proposed filter, matching
pursuit, and spherical harmonics.  Over each technique's series of
estimations, the trendlines indicate the mean error while the bars indicate
one standard deviation.  Spherical harmonics \textit{(red)} appear to produce
a smooth fit to the given noisy data, while matching pursuit \textit{(blue)}
shows the effect of discretization and sensitivity to noise.  The two raised
areas are a result of the dictionary being constructed with an 8\deg minimum
separation between any pair of orientations.  This experiment demonstrates
that the proposed filter \textit{(black)} accurately and reliably estimates
the true underlying signal.

In the second experiment, we looked at the error in angular resolution by
comparing the filtered approach to matching pursuit and sharpened spherical
harmonics.  \autoref{fig:angle_2W_MP_1000} and \autoref{fig:angle_2W_MP_3000}
show the sensitivity of matching pursuit.  Consistent with the results
reported in \cite{Descoteaux2009tmi,Descoteaux2007mrm}, spherical harmonics are
generally unable to detect and resolve angles below 50\deg for \bone or below
40\deg for \bthree.  \autoref{fig:angle_2W_SH_1000} and
\autoref{fig:angle_2W_SH_3000} confirm this, respectively.  This experiment
demonstrates that for \bone, the filtered approach consistently resolves
angles down to 20-30\deg with 5\deg error compared to independent optimization
which fails to reliably resolve below 60\deg with as much as 15\deg error.
For \bthree, the filtered approach consistently resolves down to 20-30\deg
with 2-3\deg error compared to independent optimization which cannot resolve
below 50\deg with 5\deg error.






\subsection{\textit{In vivo} tractography}  \label{sec:W_real}

As in \autoref{sec:W_real}, we first focused on fibers originating in the corpus
callosum.  We proceed with two basic experiments:  first examining the tracts
surrounding a single coronal slice and second looking at all tracts passing
through the corpus callosum.  We seed each algorithm multiple times in voxels
at the intersection of the mid-sagital plane and the corpus callosum.  To
explore branchings found using the proposed technique, we considered a
component to be branching if it was separated from the primary component by
less than 40\deg with $k\ge0.6$.  Similarly, with sharpened spherical
harmonics, we considered it a branch if we found additional maxima over the
same range.  We terminated fibers when the general fractional anisotropy of
the estimated signal (std/rms) fell below 0.1.  While such heuristics are
somewhat arbitrary, we found little qualitative difference in adjusting these
values.

To demonstrate the flexibility of the proposed filtering strategy with respect
to model choice, we use both the two-Watson fiber model
(\autoref{eq:2W_model}) and the three-Watson fiber model
(\autoref{eq:3W_model}).  While this introduced differences in the quantity of
branchings detected, we found that using either model resulted in generally
finding the same pathways.

For the first experiment, \autoref{fig:W_tc} shows tracts originating from
within a few voxels intersecting a particular coronal slice.  For a reference
backdrop, we use a coronal slice showing the intensity of fractional
anisotropy (FA) placed a few voxels behind the seeded coronal position.
Keeping in mind that these fibers are intersecting or are in front of the
image plane, this roughly shows how the fibers navigate the areas of high
anisotropy (bright regions).  Our proposed method traces out many pathways
consistent with the apparent anatomy using either the two-fiber or three-fiber
model.  To emphasize transcallosal tracts, we color as blue those fibers
exiting a corridor of $\pm22\,\text{mm}$ around the mid-sagittal plane.
\autoref{fig:W_tc_zoom} provides a closer inspection of \autoref{fig:2W_tc} and
\autoref{fig:3W_tc} where, to emphasize the underlying anatomy influencing the
fibers, we use as a backdrop the actual coronal slice passing through the
voxels used to seed this run.  Such results are obtained in minutes in either
our MATLAB or Python implementations.  At each step, the cost of
reconstructing the signal for few sigma points approaches the cost of a few
iterations of weighted least-squares estimation of a single tensor.

For the second experiment, \autoref{fig:W_cc} shows a view of the whole brain to
see the overall difference between the different methods.  Here again we
emphasize with blue the transcallosal fibers found using the proposed filter.
Comparing \autoref{fig:2W_cc} and \autoref{fig:3W_cc} we see several regions
that appear to have different fiber density using the two models.  This
suggests that incorporating model selection into filtered approaches may have
a significant effect.  To show the various pathways infiltrating the gyri,
\autoref{fig:frontal} provides a closeup of the frontal lobe from above
(without blue emphasis).

Next we examined fibers passing through the internal capsule to trace out the
pathways reaching up into the primary motor cortex at the top of the brain as
well as down into the hippocampal regions near the brain stem.
\autoref{fig:ic_front} shows frontal views for each technique with seeding
near the cerebral peduncles \textit{(blue)}.  \autoref{fig:ic_side} shows this
same result from a side view where we can see that the filtered approach picks
up the corticospinal pathways.  Notice that the two-Watson model picks up the
temporopontine and parietopontine tracts and the three-Watson model further
reveals the occipitopontine pathways, another indication that the chosen fiber
model often affects the results.  As reported elsewhere
\cite{Behrens2007,Qazi2008ni}, single-tensor tractography follows the dominant
corticospinal tract to the primary motor cortex.  The same pathways were also
found with spherical harmonics.
%
\autoref{fig:ic_top} shows a view from above where we use a transverse FA
image slice near the top of the brain as a backdrop so we can focus on the
fiber endpoints.  From this we can see how each method infiltrates the sulci
grooves, and specifically we see that the filtered method is able to
infiltrate sulci more lateral compared to single-tensor tractography.

Note that in the region of intersection between the transcallosal fibers, the
corticospinal, and the superior longitudinal fasciculus, the partial voluming
of each of these pathways leads the filter to report several end-to-end
connections that are not necessarily present, e.g. fibers originating in the
left internal capsule do not pass through this region, through the corpus
callosum and then insert into the right motor cortex.  Many of the lateral
extensions are callosal fibers that are picked up while passing through this
juncture.  It is our hope that such connections may be avoided with the
introduction of weighted mixtures, alternative filter formulations, or
different heuristic choices in the algorithm.

\clearpage
\begin{figure}[p]
  \centering
  \subfloat[Single-tensor]{ \label{fig:W_1T_cc}
    \begin{minipage}[b]{0.32\linewidth}
      \includegraphics[width=\textwidth]{case01045_1T_cc_top} \\
      \includegraphics[width=\textwidth]{case01045_1T_cc_front}
    \end{minipage}}%
  \subfloat[Spherical harmonics]{ \label{fig:W_SH_cc}
    \begin{minipage}[b]{0.32\linewidth}
      \includegraphics[width=\textwidth]{case01045_SH_cc_top}
      \includegraphics[width=\textwidth]{case01045_SH_cc_front}%
    \end{minipage}}%
  \subfloat[Filtered two-Watson]{ \label{fig:2W_cc}
    \begin{minipage}[b]{0.32\linewidth}
      \includegraphics[width=\textwidth]{case01045_2W_cc_top}
      \includegraphics[width=\textwidth]{case01045_2W_cc_front}
    \end{minipage}}

  \subfloat[Filtered three-Watson]{ \label{fig:3W_cc}
    \begin{minipage}[b]{0.7\linewidth}
      \includegraphics[width=.49\textwidth]{case01045_3W_cc_top}%
      \includegraphics[width=.49\textwidth]{case01045_3W_cc_front}
    \end{minipage}}
  \caption{Tracing fibers originating from the center of the entire corpus
    callosum with views from above \textit{(top rows)} and front-to-back
    \textit{(bottom rows)}.  The proposed filtered tractography is able to
    find many of the lateral projections \textit{(blue)} while single-tensor
    is unable to find any and few are found with sharpened spherical
    harmonics.  Seed region indicated in yellow.}
  \label{fig:W_cc}
\end{figure}


\chapter{Weighted Mixtures}  \label{ch:weighted}

\section{Summary}

Until now we have made use of two-component and three-component models, all
with fixed volume fractions in the general mixture equation
(\autoref{eq:general_model},\autoref{eq:2T_model}).
%
This assumption is forced upon us because update equations of the standard
Kalman filter are unable to enforce the convex relationship between mixture
coefficients.  We began in \autoref{sec:weights} to explore the impact of this
assumption in the presence of mixed fiber populations.  A similar problem
arises with the eigenvalues needing to be positive; however, for this a
practical solution was to clamp these values at small positive values after
each update.
%
Our initial efforts at using the log domain to ensure positivity showed
promise as an indirect method of influencing the positivity of eigenvalues,
but still did not address the convexity constraint \cite{rathi2010mrm}.

To treat the problem more directly, we began exploring constrained filtering
where the state estimate is constrained to a subspace of allowable solutions
\cite{Simon2006}.  In our situation, this ensures that tensor eigenvalues are
positive and the mixture weights are non-negative and convex.

\autoref{sec:2TW} defines the weighted two-fiber model employed in
this section, and \autoref{sec:cukf} describes how this
model fits into the unscented Kalman filter and more importantly how the
constraints are enforced.  This chapter follows our work in
\cite{malcolm2009cukf,Malcolm2009cukf_ext}.


\subsection{The weighted model} \label{sec:2TW}

Following the equally-weighted two-tensor model from earlier
(\autoref{eq:2T_model}), we use it here in its more general form:
\begin{equation} \label{eq:2TW_model}
  s_i = s_0 w_1 e^{ -b \u_i^T D_1 \u_i } + s_0 w_2 e^{ -b \u_i^T D_2 \u_i } ,
\end{equation}
where $w_1,w_2$ are convex weights and $D_1,D_2$ are each expressible as
%
$ D = \lx \m \m^T + \ly\left(\v{p}\v{p}^T + \v{q}\v{q}^T\right), $
%
with $\m,\v{p},\v{q} \in \S^2$ forming an orthonormal basis aligned to the
principal diffusion direction \m.  The free model parameters are then $\m_1$,
$\lx_1$, $\ly_1$, $w_1$, $\m_2$, $\lx_2$, $\ly_2$, and $w_2$.  Lastly, we wish
to constrain this model to have positive eigenvalues and convex weights
($w_1,w_2\ge0$ and $w_1+w_2=1$).



\subsection{Constrained estimation} \label{sec:cukf}

As in the standard estimation process, we begin with the application-specific
definition of four filter components.  The only difference here is for that we
directly include the weighting parameters for the two-tensor model in
\autoref{eq:2TW_model}:
\begin{equation} \label{eq:2TW_state}
  \x = \rv{ \m_1 \;\; \lx_1 \;\; \ly_1 \;\; w_1 \;\;
            \m_2 \;\; \lx_2 \;\; \ly_2 \;\; w_2 }^T ,
  \quad
  \m\in\S^2, \lambda\in\R^{+}, w\in[0,1] .
\end{equation}
For the state transition we again assume identity dynamics, and our
observation is the signal reconstruction, $\y=h[\x]=\s=\rv{s_1,...,s_m}^T$
using $s_i$ described by the model in \autoref{eq:2TW_model}, and our
measurement is the actual signal interpolated directly on the diffusion
weighted images at the current position.

\begin{algorithm}[t]
  \setlength\abovedisplayskip{2pt}
  \setlength\belowdisplayskip{2pt}
  \caption{Constrained Unscented Kalman Filter}
  \label{alg:cukf}
  \begin{algorithmic}[1]
    \STATE Form weighted sigma points $\X_t=\{w_i, \x_i\}_{i=0}^{2n}$ around
    current mean $\x_t$ and covariance $P_t$ with scaling factor $\zeta$
    \begin{equation*}
      \x_0 = \x_t
      \qquad
      \x_i    = \x_t + [\sqrt{\zeta P_t}]_i
      \qquad
      \x_{i+n} = \x_t - [\sqrt{\zeta P_t}]_i
    \end{equation*}
    \STATE Project onto the constrained subspace (\autoref{eq:constraints})
    \STATE Predict the new sigma points and observations
    \begin{equation*}
      \X_{t+1|t} = f[\X_t]   \qquad   \Y_{t+1|t} = h[\X_{t+1|t}]
    \end{equation*}
    \STATE Project onto the constrained subspace (\autoref{eq:constraints})
    \STATE Compute weighted means and covariances, \eg
    \begin{equation*}
      \bar{\x}_{t+1|t} = \sum_i w_i ~ \x_i
      \qquad
      P_{xy} = \sum_i w_i (\x_i - \bar{\x}_{t+1|t})(\y_i - \bar{\y}_{t+1|t})^T
    \end{equation*}
    \STATE Update estimate using Kalman gain $K$ and scanner measurement
    $\y_t$
    \STATE Project onto the constrained subspace (\autoref{eq:constraints})
    \begin{equation*}
      \x_{t+1} = \bar{\x}_{t+1|t} + K(\y_t - \bar{\y}_{t+1|t})
      \qquad
      P_{t+1} = P_{xx} - K P_{yy} K^T
      \qquad
      K = P_{xy}P_{yy}^{-1}
    \end{equation*}
  \end{algorithmic}
\end{algorithm}

In the standard formulation (\autoref{sec:estimation}), we ignored the
constraints on our model.  This results in instabilities:  the diffusion
tensors may become degenerate with zero or negative eigenvalues, or the
weights may become negative.  To enforce appropriate constraints, one can
directly project any unconstrained state $\x$ onto the constrained subspace
\cite{Simon2006}.  In other words, we wish to find the state $\hat{\x}$
closest to the unconstrained state $\x$ which still obeys the constraints,
$A\hat{\x} \le \v{b}$.  Using $P_t$ as a weighting matrix, this becomes a
quadratic programming problem:
\begin{equation} \label{eq:constrained}
  \min_{\hat{\x}} \ (\x - \hat{\x})^T P_t^{-1} (\x - \hat{\x})
  \quad
  \text{subject to}
  \quad
  A\hat{\x} \le \v{b} .
\end{equation}
This projection procedure is applied within unscented Kalman filter procedure
to correct at every place where we move in the state-space: after spreading
the sigma points $\X_t$, after transforming the sigma points $\X_{t+1|t}$, and
after the final estimate $\x_{t+1}$.  See \autoref{alg:cukf}.


In this study, for voxels that can be modeled with only one tensor, we found
it preferable to have both the tensor components similarly oriented.  Upon
encountering a region of dispersion, the second component is poised and ready
to begin branching instead of having zero weight and arbitrary orientation.
To favor such solutions, we require the weights of each of the components to
be not just non-negative but also greater than 0.2, and so, in our current
implementation, $A$ and \v{b} are constructed to encode the following state
constraints:
\begin{equation} \label{eq:constraints}
  \lx_1, \ly_1, \lx_2, \ly_2 > 0
  \qquad
  w_1, w_2 \ge 0.2
  \qquad
  w_1 + w_2 = 1 .
\end{equation}



\section{Experiments}

We first use experiments with synthetic data to validate our technique against
ground truth.  We confirm that our approach accurately recognizes crossing
fibers over a broad range of angles and consistently estimates the partial
volumes (\autoref{sec:w_synthetic}).  We then examine a real dataset to
demonstrate how causal estimation is able to pick up fibers and branchings
known to exist \textit{in vivo} yet absent using other techniques
(\autoref{sec:w_real}).

In these experiments, we compare against two alternative techniques.
%
% First, we use the same weighted two-tensor model from \autoref{sec:model} with
% a variant matching pursuit for brute force, dictionary-based optimization
% \cite{Mallat1993}.  In our implementation, we simply project against a single
% dictionary populated with the same pure two-tensor signals used to generate
% the synthetic data (same weights and eigenvalues) only oriented at a higher
% tessellation, thus any error is due to the method's sensitivity to noise.
% This approach highlights the effect of using the same model but changing the
% optimization technique to one that treats each voxel independently.  Since we
% use the same eigenvalues and weights as used to construct the data, this is in
% effect the theoretical limit on performance for independent optimization
% techniques.  We will demonstrate that the filtered approach consistently beats
% this theoretical limit.
%
First, we use sharpened spherical harmonics with peak detection as described
in \cite{Descoteaux2007} (order $l=8$, regularization $L=0.006$).  This
provides a comparison with an independently estimated nonparametric
representation.
%
Second, when performing tractography on real data, we also compare against
single-tensor streamline tractography for a baseline.
% \footnote{Using the freely available Slicer 2.7
%   (\url{http://www.slicer.org}).}.


\subsection{Synthetic validation} \label{sec:w_synthetic}

Following the experimental method of generating multi-compartment synthetic
data found in \cite{Tuch2002,Descoteaux2007,Schultz2008}, we averaged the
eigenvalues of the 300 voxels with highest fractional anisotropy (FA) in our
real data set:  $\{1200, 100, 100\}\mu$m$^2$/msec.  We used these eigenvalues
to generate synthetic MR signals according to \autoref{eq:general_model} at
\bone with 81 gradient directions on the hemisphere and introduced Rician
noise (\snr{5}).

\begin{figure}[t]
  \centering
  \subfloat[50\%-50\%]{
    \begin{minipage}[b]{0.24\linewidth}
      \includegraphics[width=\columnwidth]{detection_2TW_b1000_dirty_1} \\
      \includegraphics[width=\columnwidth]{angle_2TW_b1000_dirty_SH_1} \\
      \includegraphics[width=\columnwidth]{weight_2TW_b1000_dirty_1}
    \end{minipage}}%
  \subfloat[60\%-40\%]{
    \begin{minipage}[b]{0.24\linewidth}
      \includegraphics[width=\columnwidth]{detection_2TW_b1000_dirty_2} \\
      \includegraphics[width=\columnwidth]{angle_2TW_b1000_dirty_SH_2} \\
      \includegraphics[width=\columnwidth]{weight_2TW_b1000_dirty_2}
    \end{minipage}}%
  \subfloat[70\%-30\%]{
    \begin{minipage}[b]{0.24\linewidth}
      \includegraphics[width=\columnwidth]{detection_2TW_b1000_dirty_3} \\
      \includegraphics[width=\columnwidth]{angle_2TW_b1000_dirty_SH_3} \\
      \includegraphics[width=\columnwidth]{weight_2TW_b1000_dirty_3}
    \end{minipage}}%
  \subfloat[80\%-20\%]{
    \begin{minipage}[b]{0.24\linewidth}
      \includegraphics[width=\columnwidth]{detection_2TW_b1000_dirty_4} \\
      \includegraphics[width=\columnwidth]{angle_2TW_b1000_dirty_SH_4} \\
      \includegraphics[width=\columnwidth]{weight_2TW_b1000_dirty_4}
    \end{minipage}}%
  \caption{Comparison of sharpened spherical harmonics \textit{(red)} against
    filtered approach \textit{(black)} over several different metrics:
    detection rate, angular resolution, estimated primary fiber weight
    \textit{(rows, top to bottom)}.  Each column is a different primary fiber
    weighting.  The filter provides superior detection rates, accurate angular
    resolution, and consistent weight estimation.  Trendlines indicate mean
    while dashed bars indicate one standard deviation.}
  \label{fig:w_synthetic}
\end{figure}
As before, we constructed a set of two-dimensional fields through which to
navigate (\autoref{sec:tracts}).  In the middle is one long pathway where the
filter starts at one end estimating a single tensor but then runs into voxels
with two crossed fibers at a fixed angle and weighting.  In this crossing
region we calculated error statistics to compare against sharpened spherical
harmonics.
% By varying the size of the crossing region, we ensured that both techniques
% performed about 500 estimations.

From these synthetic sets, we examined detection rate, angular resolution, and
estimated volume fractions and we plot the results in \autoref{fig:w_synthetic}.
Each column looks at a different primary-secondary weighting combination, and
each row looks at a different metric.
%
In the top row, we count how many times each technique distinguishes two
separate fibers.  The filtered approach \textit{(black)} is able to detect two
distinct fibers at crossing angles far below that using spherical harmonics
\textit{(red)}.  Further, the filtered approach maintains such relatively high
detection rates even at 80/20 partial voluming \textit{(far right column)}.
%
In the middle row, we look at where each technique reported two fibers and we
record the error in estimated angles.  From this, we see that spherical
harmonics result in an angular error of roughly 15\deg at best and fails to
detect a second component at angles below 60\deg.  In contrast, the filtered
approach has an error between 5-10\deg and is able to accurately estimate down
to crossing angles of 30\deg.
%
In the bottom row, we look at the primary fiber weight estimated by the
filter.  As expected, this estimate is most accurate closer to 90\deg
\textit{(blue line indicates true weight)}.


\subsection{\textit{In vivo} tractography}  \label{sec:w_real}

\begin{figure}[t]
  \centering
  \subfloat[Single-tensor]{
    \includegraphics[width=.33\columnwidth]{case01045_1T_tc}}%
  \subfloat[Spherical harmonics]{
    \includegraphics[width=.33\columnwidth]{case01045_SH_tc}}%
  \subfloat[Filtered]{
    \includegraphics[width=.33\columnwidth]{case01045_2TW_tc}}
  \caption{Filtered tractography picks up many fiber paths consistent with the
    underlying structures.  Both single-tensor streamline and sharpened
    spherical harmonics are unable to find the majority of these
    pathways. Seed region indicated in yellow.}
  \label{fig:TW_tc}
\end{figure}
\begin{figure}[t]
  \centering
  \subfloat[Single-tensor]{
    \includegraphics[width=0.32\columnwidth]{case01045_1T_cc}}%
  \subfloat[Spherical harmonics]{
    \includegraphics[width=0.32\columnwidth]{case01045_SH_cc}}%
  \subfloat[Filtered]{
    \includegraphics[width=0.32\columnwidth]{case01045_2TW_cc}}
  \caption{Tracing fibers originating from the center of the entire corpus
    callosum viewed from above.  The proposed filtered tractography is able to
    find many of the lateral projections \textit{(blue)} while single-tensor
    is unable to find any and few are found with sharpened spherical
    harmonics. Seed region indicated in yellow.}
  \label{fig:TW_cc}
\end{figure}
Our work focuses on fibers originating in the corpus callosum.
Specifically, we sought to trace out the lateral transcallosal fibers that run
through the corpus callosum out to the lateral gyri.  It is known that
single-tensor streamline tractography only traces out the dominant pathways
forming the U-shaped callosal radiation while spherical harmonics only capture
some of these pathways \cite{Descoteaux2007,Schultz2008}.
%
These fibers have been reported in using diffusion spectrum imaging
\cite{Hagmann2004}, probabilistic tractography
\cite{Kaden2007,Anwander2007,Descoteaux2007}, and more recently with tensor
decomposition \cite{Schultz2008}; however, all of these techniques require
lengthy scans at high \textit{b}-values along with significantly more
processing time than streamline techniques.

We begin by seeding each algorithm up to thirty times in voxels at the
intersection of the mid-sagital plane and the corpus callosum.  To explore
branchings found using the proposed technique, we considered a component to be
branching if it was separated from the primary component by less than 40\deg
with FA$\ge$0.15 and weight above 0.3.  Similarly, with sharpened spherical
harmonics, we considered it a branch if we found additional maxima over the
same range.  We terminated fibers when either the generalized fractional
anisotropy \cite{Tuch2002} of the estimated signal fell below 0.1 or the
primary component FA fell below 0.15 or weight below 0.3.

We tested our approach on a human brain scan using a 3-Tesla magnet to collect
51 diffusion weighted images on the hemisphere at $b=900\,\text{s/mm}^2$, a
scan sequence comparable those of \cite{Descoteaux2007,Schultz2008}.
\comment{with a voxel size of $1.66\times1.66\times1.7\,\text{mm}^3$ and}
%
\autoref{fig:TW_tc} shows tracts originating from within a few voxels
intersecting a chosen coronal slice.  Confirming the results in
\cite{Descoteaux2007,Schultz2008}, sharpened spherical harmonics only pick up
a few fibers intersecting the U-shaped callosal radiata.  In contrast, our
proposed algorithm traces out many pathways consistent with the apparent
anatomy.  \autoref{fig:TW_cc} shows a view of the whole corpus callosum from
above.  The filtered approach is able to pick up many transcallosal fibers
throughout the corpus callosum as well as infiltrating the frontal gyri to a
greater degree than either alternate technique.  To emphasize transcallosal
tracts, we color as blue those fibers exiting a corridor of $\pm
22\,\text{mm}$ around the mid-sagittal plane.

\section{Summary}

In this section, we demonstrated that the constrained approach gives
significantly lower angular error (5-10\deg) in regions with fiber crossings
than using sharpened spherical harmonics (15-20\deg), and it reliably
estimates the partial volume fractions.

While these initial results are promising, there are several aspects to be
improved.
%
First, without a prior, the weights tend toward the middle of their range,
each tending to hover around 0.6 (halfway between 0.2 and 1.0).  The effect is
that both components show some presence, even in strongly coherent fields.
The solution might be to induce priors over their distribution, hopefully
forcing them to zero without signal evidence of a second component.
%
Second, solving the quadratic constraint is expensive.  A practical
alternative might be to simply project as close to the subspace as possible
\cite{Simon2006}; although, this allows solutions not necessarily on the
subspace and may again open the possibility for non-convex weights and
negative eigenvalues.  In practice, this may still be sufficient.




\chapter{Validation}

\section{Summary}
In attempting to reconstruct neural pathways virtually, it is important to
keep in mind the inherent uncertainty in such reconstructions.  The resolution
of dMRI scanners is at the level of 3-10mm$^3$ while physical fiber axons are
often an order of magnitude smaller in diameter--a relationship which leaves
much room for error.
%
Some noise or a complex fiber configuration could simply look like a diffuse
signal and cause probabilistic tractography to stop in its tracks, while a few
inaccurate voxel estimations could easily send the deterministic tracing off
course to produce a false-positive connection.  Even global methods could
produce a tract that fits the signal quite well but incidentally jumps over an
actual boundary in one or two voxels it thinks are noise.  Consequently, a
common question is: Are these pathways really present?

With this in mind, an active area of study is validating such results.  Since
physical dissection often requires weeks of tedious effort, many techniques
have been used for validating these virtual dissections.
%
A common starting point is to employ synthetic and physical phantoms with
known parameters when evaluating new methods \cite{Poupon2008phantom}.
%
When possible, imaging before and after injecting radio-opaque dyes directly
into the tissue can provide some of the best evidence for comparison
\cite{Lin2003,Dauguet2007}.
%
Another powerful approach is to apply bootstrap sampling or other
non-parametric statistical tests to judge the sensitivity and reproducibility
of resulting tractography against algorithm parameters, image acquisition, and
even signal noise
\cite{Lazar2005,Jones2005,Gigandet2008,Whitcher2008,Chung2006,Clayden2007}.

To begin validating these filtered approaches, we applied this technique to a
phantom simulating several complex pathway interactions and highlight tracts
passing through several prescribed seed positions.  This section follows our
work in \citet{Malcolm2009fc}.

\section{Results and Discussion}

\begin{figure}[t]
  \centering
  \subfloat[Ground truth]{\includegraphics[width=.49\textwidth]{ground_truth}}
  \subfloat[Filtered tractography]{\includegraphics[width=.49\textwidth]{phantom_3mm}}
  \caption{Baseline image from the synthetic phantom (3mm, \bn{1500})
    overlayed with ground truth and selected fiber tracts (colored) and seed
    points (white).  The filter is capable of tracing through regions of
    crossing, branching, and fanning.}
  \label{fig:phantom}
\end{figure}

Among the phantoms provided in the 2009 MICCAI Fiber Cup, we present results
using the 3mm version at \bn{1500} \cite{Poupon2008phantom}.  Instead of
initializing tractography from the prescribed seed points, we begin by seeding
in voxels with nonzero baseline intensity and terminating tractography when
the estimate becomes isotropic, essentially ``full-brain'' tractography.  From
these potential pathways, \autoref{fig:phantom} shows a representative fiber
for each seed point.  We further restricted movement to the image plane.

With the explosion of techniques for mapping connectivity, it is often
difficult to assess the relative merits among various approaches, and each
application has its particular goals.
%
For connectivity studies, one may only be interested in the final resolved
pathway; however, questions arise such as whether to branch or whether to
represent connectivity as a discrete path or a voxel-to-voxel connectivity
matrix possibly telling more about the relative certainty of connectivity.
%
For tissue studies, one may be primarily concerned with the estimated
microstructure at each position.  Filtered approaches like this provide not
only an estimate of these quantities (mean) but also an additional measure of
uncertainty (variance).

The approach presented here may be considered a local method: at the current
position we estimate a direction and take a step.  With such approaches, one
mistake can send the subsequent trajectory off track.  We believe that more
global approaches should be considered, ones that take into account larger
portions of the fiber pathway.  Further, we believe that anatomical priors
should be incorporated.  Such a progression of techniques may be considered
analogous to how level set methods developed from local edge-based
computations, to more global region-based approaches, and further with
integration of shape priors.



\chapter{Tract-based Statistical Analysis}

\section{Summary}

Diffusion tensor imaging has made it possible to evaluate the organization and
coherence of white matter fiber tracts.  Hence, it has been used in many
population studies, most notably, to find abnormalities in schizophrenia.  To
date, most population studies analyzing fiber tracts have used a single tensor
as the local fiber model.  While robust, this model is known to be a poor fit
in regions of crossing or branching pathways.  Nevertheless, the effect of
using better alternative models on population studies has not been studied.
%
The goal of this section is to compare white matter abnormalities as revealed
by two-tensor and single-tensor models.  To this end, we compare three
different regions of the brain from two populations: schizophrenics and normal
controls.  Preliminary results demonstrate that regions with significant
statistical difference indicated using one-tensor model do not necessarily
match those using the two-tensor model and vice-versa.  We demonstrate this
effect using various tensor measures.  This chapter follows our work in
\citet{Malcolm2009study}.


\section{Introduction}

Diffusion tensor imaging (DTI) has become an established tool for
investigating tissue structure, and many studies have used it to understand
the effects of aging or disease.  Using this imaging technique,
neuroscientists wish to ask how regions of tissue compare or how well-defined
various connections may be.
%
For example, several DTI studies have indicated a disturbance in connectivity
between different brain regions, rather than abnormalities within any specific
region as responsible for the cognitive dysfunctions observed in schizophrenia
\cite{Kubicki2007}.  For such studies, the quality of the results depends on
the accuracy of the underlying model.

The most common local fiber model used in population studies is a single
diffusion tensor which provides a Gaussian estimate of diffusion orientation
and strength.  While robust, this model can be inadequate in cases of mixed
fiber presence or more complex orientations, and so various alternatives have
been introduced including mixture models
\cite{Alexander2001,Tuch2002,Peled2006,Hall2008} as well as nonparametric
approaches \cite{Tuch2002,Descoteaux2007,Jansons2003,Tournier2004,Jian2007ni}.
Probabilistic techniques have also been developed in connectivity studies
\cite{Parker2003,Behrens2007}.

Despite this wide selection of available models, nearly all population studies
thus far have been based on the single-tensor model, and as such it is
important to examine the limits of this model and potential impact this has.
%
To start, some works have focused on the effects of noise and acquisition
schemes and have found nontrivial effects on estimated fractional anisotropy
(FA) and other quantities \cite{Basser2000noise,Jones2004,Goodlett2007}.
%
Beyond such factors, it is known that in regions containing crossing or
fanning pathways the single-tensor model itself provides a poor fit that
results in lower FA \cite{Alexander2002,Tuch2002}.  It is estimated
that as much as one third of white matter may contain such putative fiber
populations \cite{Behrens2007}.
%
It is here that this present study focuses.

In forming studies, there are several approaches for comparing patient
populations.
%
For example, voxel-based studies examine tissue characteristics in regions of
interest \cite{Ashburner2000}.
%
In contrast, tract-based studies incorporate the results of tractography to
use fiber pathways as the frame of reference \cite{Ding2003,Smith2006}, and
several studies have demonstrated the importance of taking into account local
fluctuations in estimated diffusion
\cite{Corouge2006,Fletcher2007ipmi,ODonnell2007,Maddah2008,Goodlett2009}.

To date, many studies have focused on schizophrenia, but the findings vary.
For example, the review by Kubicki \etal \cite{Kubicki2007} cites one
voxel-based study where the genu of the corpus callosum has significant
differences and another that finds nothing.  A recent tract-based study showed
statistical differences not only in the genu of the corpus callosum but also
in regions known to have fiber crossings and branchings \cite{Maddah2008}.
The question then arises, as to whether the same effect can be seen by better
models able to resolve crossings and branchings?  Is the population difference
simply a result of poor modeling?  We seek to answer such questions in this
present work.

\comment{For example, changes in the genu of corpus callosum have been
  reported in \cite{Foong2000} but not confirmed in \cite{Sun2003}}



\section{Our contributions}

In this chapter, we make a first attempt towards confirming or creating doubts
regarding the results reported using the single-tensor model.
%
Specifically, we compare various tensor metrics as estimated using two-tensor
and single-tensor models on a number of fiber tracts generated using our
recently proposed method for deterministic two-tensor tractography
\cite{Malcolm2009ipmi}.  Hence we are continuing this work with a focus
on how such techniques will begin affecting clinical studies.  It is important
to note that this is the first time the same population study has been
performed using two different underlying models.

We begin with synthetic experiments to examine the difference in reported FA
using single- and two-tensor models.  We find that the single-tensor model
consistently underestimates the FA by as much as 30\% in crossing regions, a
difference considered statistically significant in many studies.
%
Then, we look at connections between three different cortical regions of the
brain and show that statistical group differences reported using the
single-tensor model do not necessarily show differences using the two-tensor
model and vice-versa.  Specifically, the regions known to have branchings and
crossings reports significant differences in the single tensor study, but not
in the two-tensor study, and conversely, certain regions which show subtle
abnormalities using the two-tensor method are lost in the single-tensor model.
Thus, model error may have contributed to the statistical differences found in
previous DTI studies.


\section{Method} \label{sec:method}

In this chapter, we form a tract-based study using the two-tensor tractography
method described in \autoref{sec:equal}, and we compare this against the
results from a single-tensor model typically used in such studies.
%
\autoref{sec:model} provides the necessary background on modeling the measured
signal using tensors and defines the specific two-fiber model employed in this
study.
%
\autoref{sec:fibers} looks at the seed regions and the resulting fibers
connecting each hemisphere and finally how these fibers are compared within a
tract-based coordinate system.

Several scalar measures have been proposed for quantifying various aspects of
tensors, and in this study we focus on fractional anisotropy (FA), trace, and
the ratio between eigenvalues ($\lambda_2/\lambda_1$).  Since these measures
are defined for the single-tensor model, when computing their value on the
two-tensor model, we record the value from the tensor most aligned with the
local fiber tangent.


\section{Tractography and fiber comparison} \label{sec:fibers}

\begin{figure}[t]
  \centering
  \subfloat[Seed regions]{
    \includegraphics[width=.40\columnwidth]{case01009_roi_front}}%
  \subfloat[\Green]{\includegraphics[width=.40\columnwidth]{case01009_03}}

  \subfloat[\Blue] {\includegraphics[width=.40\columnwidth]{case01009_24}}%
  \subfloat[\Red]  {\includegraphics[width=.40\columnwidth]{case01009_28}}%
  \caption{For each patient, we seed in three different cortical regions and
    select only those fibers that connect the hemispheres.}
  \label{fig:roi_fibers}
\end{figure}
For each patient, we have manually delineated cortical regions from which we
choose three regions covering the frontal and parietal lobe.  For one patient,
\autoref{fig:roi_fibers} shows these seed regions and the resulting fibers
that connect each hemisphere.  Specifically, the regions are the \red, \blue,
and \green.

We followed the deterministic fiber tracking procedure in
\cite{Malcolm2009ipmi}.  We begin by seeding each algorithm several times
in each voxel of the seed regions.  To explore branchings found using the
proposed technique, we considered a component to be branching if it was
separated from the primary component by less than 40\deg with FA$\ge$0.15.  We
terminated fibers when either the general fractional anisotropy
\cite{Tuch2002} of the estimated signal fell below 0.1 or the primary
component FA fell below 0.15.  While such parameters are heuristic in nature
and could be examined in their own right, we found the resulting tractography
to be sufficient for the purposes of this work.

\begin{figure}[t]
  \centering
  \subfloat[Single-tensor.]{ \label{fig:03_1T}
    \includegraphics[width=.49\textwidth]{case01026_03_single}}%
  \quad
  \subfloat[Two-tensor and FA curves]{ \label{fig:03}
    \includegraphics[width=.49\textwidth]{case01026_03_ds}}%
  \caption{Single-tensor tractography finds no connections.  Two-tensor passes
    through the region of crossing \textit{(red/yellow boundary)}.  FA curves
    show drop in single-tensor FA \textit{(blue)} in this region
    \textit{(indicated by dashed white line)}.}
  \label{fig:ds_1T}
\end{figure}
It is known that single-tensor streamline tractography is only able to trace
out the dominant pathways forming the U-shaped callosal radiation, so previous
tract-based studies looking at the corpus callosum have been restricted to
only studying portions of the corpus radiata, typically focusing on the
splenum and genu \cite{Corouge2006,Fletcher2007ipmi,Goodlett2008,Maddah2008}.
%
One of the main advantages of using the multi-tensor filtering approach in
\cite{Malcolm2009ipmi} is that it is one of the few techniques capable of
following, not only these dominate pathways, but also the transcallosal
pathways out to the lateral gyri.  For example, \autoref{fig:03_1T} shows that
for the \green region, using single-tensor tractography leads to nearly all
fibers looping back into another sulci instead of finding any connection to
the opposite hemisphere.  In contrast, \autoref{fig:03} demonstrates that the
filtered two-tensor approach finds several connecting pathways.  Therefore,
this is the first attempt at tract-oriented analysis along such pathways.

\begin{figure}[t]
  \centering
  \subfloat[Single-tensor FA.]{ \label{fig:24_1T}
    \includegraphics[width=.49\textwidth]{case01009_24_fa_1t}}%
  \subfloat[Two-tensor FA.]{ \label{fig:24_2T}
    \includegraphics[width=.49\textwidth]{case01009_24_fa}}%
  \caption{Two-tensor fibers overlayed with FA intensity using both models
    \textit{(red to yellow is low to high)}.  Both methods show high FA in the
    corpus callosum but single-tensor FA drops as fibers enter the gray
    matter.}
  \label{fig:fa}
\end{figure}
Since this study focuses on comparing fiber models, we simply perform direct
single-tensor estimation at the same locations as we have two-tensor
estimates, thus providing an exact correspondence for averaging within each
patient.  \autoref{fig:fa} shows the \blue region and colors the same fibers
with FA intensity to show the relative FA differences reported by either
model.

After performing tractography, we placed fibers from each patient within a
common coordinate system by registering the seed regions to a template.  Each
seed region was registered separately.
%
The mid-sagittal plane was automatically determined and provided a common
reference point for each fiber as it passed through the corpus callosum.  From
this reference point, we record arc-length moving outward to the cortical
regions.  The fibers of each patient being registered with a template and
having an origin, we can plot the various scalar tensor measures along this
arc-length as in other tract-based studies.  For example, \autoref{fig:03}
shows FA as a function of arc-length for the \green region using the
single-tensor model \textit{(blue)} and two-tensor model \textit{(red)}.
Since the single- and two-tensor estimates line up exactly, any correspondence
error is confined to the matching within fiber bundles and among patients--not
across models which was our focus here.  As in \cite{ODonnell2007,Maddah2008},
statistical significance was tested as a function of arc-length.


\section{Results}

We first use experiments with synthetic data to validate our technique against
ground truth.
%
By varying crossing angle or eigenvalues used to construct voxels, we confirm
that our approach accurately estimates the fractional anisotropy while using
the single-tensor model can under-estimate FA by as much as 30-40\%.
(\autoref{sec:synthetic}).
%
Then, we examine our real dataset to demonstrate the different results
reported using either model (\autoref{sec:study_real}).


\subsection{Synthetic validation} \label{sec:synthetic}

\begin{figure}[t]
  \centering
  \subfloat[Fixed eigenvalues]{ \label{fig:theta}
    \includegraphics[width=.24\columnwidth]{fa_2T_b1000_theta}}%
  \subfloat[No crossing]{ \label{fig:theta00}
    \includegraphics[width=.24\columnwidth]{fa_2T_b1000_theta_00}}%
  \subfloat[Crossing at 45\deg]{ \label{fig:theta45}
    \includegraphics[width=.24\columnwidth]{fa_2T_b1000_theta_45}}%
  \subfloat[Crossing at 90\deg]{ \label{fig:theta90}
    \includegraphics[width=.24\columnwidth]{fa_2T_b1000_theta_90}}
  \caption{Estimated fractional anisotropy (FA) using single-tensor
    \textit{(blue)} and two-tensor \textit{(red)} models on synthetic data
    with known FA \textit{(dashed black)}.  The two-tensor model accurately
    captures the FA across a wide range of angles and eigenvalues.}
  \label{fig:study_synthetic}
\end{figure}
Following the experimental method of generating synthetic data found in
\cite{Tournier2004,Descoteaux2007}, we generated synthetic MR signals
according to \autoref{eq:2T_model} using eigenvalues determined from out
\textit{in vivo} data.  We use 81 gradient directions uniformly spread on the
hemisphere and Rician noise (\snr{10}) based on the unweighted signal.  Using
these we constructed a set of two-dimensional fields through which to navigate
while estimating FA.

\autoref{fig:study_synthetic} shows the resulting FA estimates using direct
single-tensor estimation \textit{(blue)} and filtered two-tensor estimation
\textit{(red)}.  As expected, \autoref{fig:theta} demonstrates that crossing
regions can lead to single-tensor FA estimates as much as 0.3 lower than
expected.  In \autoref{fig:theta00} we look at both techniques accurately
estimating the FA of a single tensor as we adjust the second eigenvalue
($\lambda_2$ in \autoref{eq:2T_model}).  However, \autoref{fig:theta45} and
\ref{fig:theta90} demonstrate a consistent drop in FA under the same range of
eigenvalues.
%
This experiment demonstrates that we may expect as much as 0.2 to 0.3 drop in
FA in regions of crossing and branching.


\subsection{\textit{In vivo} model comparison}  \label{sec:study_real}

\begin{figure}[t]
  \centering
  \subfloat[\Green]{ \label{fig:study03}
    \begin{minipage}[b]{0.3\linewidth}
      \includegraphics[width=\columnwidth]{study_stats_1_FA} \\
      \includegraphics[width=\columnwidth]{study_stats_1_trace} \\
      \includegraphics[width=\columnwidth]{study_stats_1_RD}
    \end{minipage}}%
  \subfloat[\Blue]{ \label{fig:study24}
    \begin{minipage}[b]{0.3\linewidth}
      \includegraphics[width=\columnwidth]{study_stats_2_FA} \\
      \includegraphics[width=\columnwidth]{study_stats_2_trace} \\
      \includegraphics[width=\columnwidth]{study_stats_2_RD}
    \end{minipage}}%
  \subfloat[\Red]{ \label{fig:study28}
    \begin{minipage}[b]{0.3\linewidth}
      \includegraphics[width=\columnwidth]{study_stats_3_FA} \\
      \includegraphics[width=\columnwidth]{study_stats_3_trace} \\
      \includegraphics[width=\columnwidth]{study_stats_3_RD}
    \end{minipage}}%
  \caption{Average of various tensor metrics as a function of arc-length using
    single-tensor \textit{(blue)} and two-tensor \textit{(red)} models
    comparing normal patients \textit{(solid lines)} with schizophrenic
    patients \textit{(dotted lines)}.  Rows show FA, trace, eigenvalue ratio.
    Areas of statistical significance are indicated along the bottom
    \textit{(black dashes indicate 95\% confidence)}.  While each metric
    generally indicates the same area of significance \textit{(looking down
      columns}), the areas of significance vary with each model
    \textit{(comparing red and blue)}}.
  \label{fig:study}
\end{figure}
We tested our approach on a human brain scans using a 3-Tesla magnet to
collect 51 diffusion weighted images on the hemisphere at a voxel size of
$1.66\times1.66\times1.7\,\text{mm}^3$ and with $b=900\,\text{s/mm}^2$ in
addition to eight baseline scans.
%
Included in this study are 17 normal controls and 22 schizophrenics; however,
since not connecting fibers were not found in all patients for all regions,
not all patients were represented in each regional group.  Below are the sizes
for each region and group.
\begin{center}
\begin{tabular}{l@{\hspace{.5cm}}c@{\hspace{.5cm}}c@{\hspace{.5cm}}c}
                   & \Green & \Blue & \Red \\
  Normals          & 16     & 17    & 8    \\
  Schizophrenics   & 20     & 22    & 17
\end{tabular}
\end{center}
For each region and patient group, \autoref{fig:study} shows the resulting
average curves using the single-tensor \textit{(blue)} and two-tensor
\textit{(red)} models.  Along the bottom of each plot we indicate local
regions of statistical significance between groups.
%
In \autoref{fig:study03} we see that the two-tensor model detected a region of
significance across all three measures whereas the single-tensor model found
only one small portion of that in the trace.  As \autoref{fig:03} indicates,
this a region known to contain branching and crossing, hence the single-tensor
FA drop.  Thus, we suspect that the single-tensor was unable to provide a
tight enough fit in this region to detect the difference found using the
two-tensor model.
%
In \autoref{fig:study24} we see that the single-tensor model found a slight
area again in a region of known branching, yet the two-tensor model found
nothing.
%
In \autoref{fig:study28} we see the most reported differences and further we
see those differences reported using both models.  We note that these
differences may in part be due to the relative size of each population
supporting this region.
%
In summary, among these three regions, we see areas where each model either
confirmed or denied the findings of the other.

\section{Conclusion}

There are many challenges in building automatic frameworks for detecting
population differences.  By repeating the same tract-based study and changing
only the model, we have demonstrated that the ultimate findings may vary.
Specifically, our results indicate that some areas reported as significant
using the single-tensor model may in fact be due to poor modeling at
branchings or crossings.

While our results are preliminary, we believe that exploring both alternative
models and methods of reconstructing pathways will provide more accurate and
comprehensive information about neural pathways and ultimately enhance
non-invasive diagnosis and treatment of brain disease.

Considering future work, we expect that further model discrepancies may be
revealed with more accurate fiber and patient correspondences
\cite{ODonnell2007,Maddah2008} or functional representations
\cite{Goodlett2008}.









\chapter{Summary}

Studies involving deterministic tractography rely on the underlying model
estimated at each voxel as well as the reconstructed pathways.
%
Most of the work on deterministic and probabilistic tractography has involved
estimating the fiber-ODF independently at each voxel and performing
tractography as a post-processing step with path regularization.
%
In this work, we proposed a method for simultaneous estimation and
tractography using two- and three-tensor mixtures.  Using an unscented Kalman
filter provided robust parameter estimation and demonstrated significantly
higher angular accuracy compared to various nonparametric and independent
optimization techniques.  Specifically we found an angular error of 5-10\deg
in regions with fiber crossings compared to 15-20\deg using a common spherical
harmonic technique, and it is able to reliably resolve crossings down to
20-30\deg compared to spherical harmonics which reaches only down to
50-60\deg.
%
\Invivo experiments demonstrate the ability of the proposed method to reveal
fibers known to exist anatomically, \eg lateral transcallosal fibers or
temporal insertions along the fronto-occipital fasciculus, yet absent using
the comparison techniques.

Nevertheless, there are several aspects of this technique that could be
improved in future work.
%
First, we believe improvements can certainly be made with the local model.
For example, switching filters may provide a natural method for reduced
parameter estimation where appropriate.  Further, we believe it is still
important to explore constrained models (convex weights, positive eigenvalues)
\cite{malcolm2009cukf} as well as approaches at model selection
\cite{Behrens2007}.

Second, as is often the case in tractography, reported connections may be
invalid (false-positives).  For example, the partial voluming along the
cingulum bundle in \autoref{fig:cing} causes many spurious traces.  There are
several possible approaches to suppressing such false positives (\eg higher
resolution scans, more appropriate local models, alternative filter
formulations), but we believe that incorporating global information will
ultimately have more effect than more local choices in the model or filter.

In summary, we believe that filtering techniques offer significant increases
in sensitivity over traditional independent optimization methods, but care
must be taken ensuring anatomically correct results.  We believe that
exploring both alternative models and filtering techniques will provide more
accurate and comprehensive information about neural pathways and ultimately
enhance non-invasive diagnosis and treatment of brain disease.



\begin{postliminary}
\references
\end{postliminary}

\end{document}
